\documentclass[aspectratio=1610,english,12pt]{beamer}

\usepackage{../../shared/tex/mystyle}

\author[]{Chris Braissant}
\title[]{Spécialiste Equipement}
\institute{Ban's Diving Resort}

%---------------------------------------------------------------------
%										DOCUMENT
%---------------------------------------------------------------------
\begin{document}

\begin{frame}[plain]
	\maketitle
\end{frame}

\section{Introduction}

\subsection{But}
\begin{frame}{\insertsubsection}
	Le but de ce cours est de voir en détail et comprendre comment fonctionnent les différentes pièces d'un équipement de plongée.\\
	Le cours abordera divers sujets tels que les réparations et la maintenance de détendeurs, valves, BCDs, bouteilles et combinaison de plongée.
\end{frame}

\subsection{Administration}
\begin{frame}{\insertsubsection}
	Liability Release and Express Assumption of Risk\\
	Medical Statement
\end{frame}

\subsection{Liability Release}
\begin{frame}{\insertsubsection}
	\begin{columns}[onlytextwidth]
		\column{0.6\linewidth}
			\mypict{../pdf/misc/Liability}
		\column{0.4\linewidth}
			\begin{enumerate}\itemsep0em 
				\item Equipment Specialist
				\item Prénom et Nom
				\item Instr.: Chris Braissant
				\item Facility: Ban's diving
				\item Prénom et Nom
				\item Instr.: Chris Braissant
				\item Facility: Ban's Diving
				\item - blanc -
				\item Training agency:\\TDI/SDI
				\item Signature et date
				\item Signature et date d'un témoin
				\item Initiales
			\end{enumerate}
	\end{columns}
\end{frame}

%---------------------------------------------------------------------
%										THEORY
%---------------------------------------------------------------------
\section{Théorie de base}

% ----- UNITS -----	
\subsection{Conversion d'unités}

\begin{frame}{Système d'unités}  
	Les deux système employés dans le monde sont:
	\begin{itemize}
		\item Le Système International d'Unités "SI" (métrique)
		\item Le Système Impérial Britanique (USA, UK, Myanmar)
	\end{itemize}
\end{frame}

% SI Units
\begin{frame}{Système International d'Unités}  
	Ce système est basé sur sept unités de base.\\
	Toutes les autres unités peuvent être dérivés de ces sept unités SI.\\
	\vspace{1em}
	\centering
	\begin{tabular}{| l | l | l |}
		\hline
		Définition & Unité & Symbole\\ \hline
		Longueur & mètre & [m]\\
		Masse & kilogramme & [kg]\\
		Temps & seconde & [s]\\
		Courant électrique & Ampère & [A]\\
		Température & Kelvin & [K]\\
		Quantité de matière & mole & [mol]\\
		Intensité lumineuse & candela & [cd]\\
		\hline
	\end{tabular}
\end{frame}

% Longeur
\begin{frame}{Longeur}
	SI:
	\[
		1[mm] \xrightarrow{x10} 1[cm] \xrightarrow{x10} 1[dm] \xrightarrow{x10} 1[m]
	\]
	Imperial:
	\[
		1[inch] \xrightarrow{x12} 1[foot] \xrightarrow{x3} 1[yard] \xrightarrow{x1760} 1[mile]
	\]
	Conversion:
	\begin{align*}
				1[inch] &= 2.54[cm]\\
				1[foot] &= 30.48[cm]\\
				1[yard] &= 0.914[m]\\
				1[mile] &= 1.6[km]
	\end{align*}
\end{frame}

\begin{frame}{SI vs Imperial}  
	\mypict{../img/imperial-units}
\end{frame}

% Area
\begin{frame}{Area}
	SI:
	\begin{align*}
		1m \times 1m &= 1[m^2]\\
		1cm \times 1cm &= 1[cm^2]
	\end{align*}
	Imperial:
	\begin{align*}
		1" \times 1" &= 1[sq\ in]
	\end{align*}
	Conversion:
	\begin{alignat*}{2}
				1[sq\ in] &= 2.54 \times 2.54	&	&= 6.45[cm^2]\\
	\end{alignat*}
\end{frame}

% Volume
\begin{frame}{Volume}
	SI Units:
	\begin{alignat*}{2}
				1[m] \times 1[m] \times 1[m] &= 1[m^3] \\
				1[dm] \times 1[dm] \times 1[dm] &= 1[dm^3] = 1 liter\\
				1[cm] \times 1[cm] \times 1[cm] &= 1[cm^3]\\
	\end{alignat*}
	Imperial:
	\begin{alignat*}{3}
				1[inch] \times 1[inch] \times 1[inch] &= 1" \times 1"	 \times 1"	&	&= 1[cu\ in]\\
				1[foot] \times 1[foot] \times 1[foot] &= 1' \times 1'	 \times 1'	&	&= 1[cu\ ft]\\
	\end{alignat*}
	Conversion:
	\begin{alignat*}{2}
				1[cu\ ft] &= 30.48 \times 30.48 \times 30.48 &	&= 28.3 [dm^3] = 28.3 liter\\
	\end{alignat*}
\end{frame}

% Poids et masse
\begin{frame}{Masse et poids}
	Masse\\
	\begin{tabular}{p{0.3\linewidth} l}
		$1 kg$ & $\approx 2.2 lbs$ \\
		$1 lbs$ & $\approx 0.45 kg$ \\
		$1 m^3$ & $1000 kg$ d'eau douce\\
		& $1030 kg$ d'eau de mer\\
		& $ 1.239 kg$ d'air
	\end{tabular}\\
	Poids\\
	\begin{tabular}{p{0.3\linewidth} l}
		$1kg$ & $\approx9.81N$
	\end{tabular}
\end{frame}

% Pression
\begin{frame}{Pression}
	Pascal\\
	\begin{tabular}{p{0.2\linewidth} p{0.4\linewidth} l}
		$1 Pa$ & $1 Newton/m^2$ & $\approx0.1kg/m^2$
	\end{tabular}
	Bar\\
	\begin{tabular}{p{0.2\linewidth} p{0.4\linewidth} l}
		$1 bar$ & $100'000Pa$ & $0.1MPa$\\
		$1 bar$ & $0.981kg/cm^2$ & $\approx1kg/cm^2$
	\end{tabular}
	PSI\\
	\begin{tabular}{p{0.2\linewidth} p{0.4\linewidth} l}
		$1 PSI$ & $1lbs/sqi$&\\
		$1 PSI$ & $0.45kg/6.45cm^2$ & $0.068bar$\\
		$\rightarrow$ & $1bar$ & $14.5PSI$
	\end{tabular}
\end{frame}

% Temperature
\begin{frame}{Température}
		\mypict{../pdf/misc/Temperatures_Units.pdf} \par
	Formules:
		\begin{align*}
			[^{\circ}C] &= [K]-273 & [K] &= [^{\circ}C]+273\\
			[^{\circ}C] &= \frac{5}{9}*([^{\circ}F]-32) & [^{\circ}F] &= \frac{9}{5}*[^{\circ}C]+32 &
		\end{align*}
\end{frame}

%------ PROPERTIES OF GASES -----
\subsection{Propriétés des gaz}

% Air
\subsubsection{Air}
\begin{frame}{Composition de l'air}
	\begin{description}[lalign=parleft]
		\item[$78.084\%$]Azote ($N_2$)
		\item[$20.946\%$]Oxygène ($O_2$)
		\item[$0.934\%$]Argon ($Ar$)
		\item[$0.033\%$]Monoxide de Carbone ($CO$)
		\item[$0.003\%$]Autre gaz
		\item[]
		\item[$18.18ppm$]Neon ($Ne$)
		\item[$5.24ppm$]Helium ($He$)
		\item[$1.14ppm$]Krypton ($Kr$)
		\item[$0.09ppm$]Xenon ($Xe$)
	\end{description}
\end{frame}

\begin{frame}{Gaz respirables}
	\begin{description}[lalign=parleft]
		\item[Nitrogen ($N_2$)]Narcotique
		\item[Oxygen ($O_2$)]Toxique à haute $ppO_2$
		\item[Helium ($He$)]Distortion de la voix
		\item[Neon ($Ne$)] Dense, longue décompression, cher 
		\item[Hydrogen ($H$)] Explosif avec plus de $4\%$ d'oxygène
		\item[Argon ($Ar$)]Narcotique, dense, difficile à réspirer
		\item[Xenon ($Xe$)]Narcotique, anesthésique, cher
	\end{description}
\end{frame}

% Mix
\begin{frame}{Mélange gazeux}
	\begin{description}
		\item[Nitrox]Azote et Oxygène
		\item[Heliox]Helium et Oxygène
		\item[Trimix]Oxygène, Helium et Azote (ex; TMX 18/45)
		\item[Heliair]Helium et Air ("Trimix des pauvres")
		\item[Hydrox]Hydrogène et Oxygène
		\item[Hydreliox]Hydrogène, Helium et Oxygène
	\end{description}
\end{frame}

% N2
\subsubsection{Azote ($N_2$)}
\begin{frame}{\insertsubsubsection}
	\begin{itemize}
		\item Diatomique ($N_2$)
		\item Physiologiquement inerte
		\item Densité élevée
		\begin{itemize}\item Augmente l'effort réspiratoire en profondeur\end{itemize}
		\item Effet narcotique
		\begin{itemize}\item A pression partielle élevée ($ppN_2 > 3.2$)\end{itemize}
		\item Maladie de décompression
		\begin{itemize}\item Non métabolisé - formation de bulles\end{itemize}
		\item Ne se dissout pas dans l'eau
		\begin{itemize}\item Mais facilement dans l'huile et la graisse\end{itemize}
	\end{itemize}
\end{frame}

\begin{frame}{\insertsubsubsection}
	\begin{itemize}
		\item Maladie de décompression
		\begin{enumerate}[I]
			\item Douleurs
			\item CNS (Système Nerveux Central) et douleurs
			\item Sévère type I et II \\ Suffocations \\ Contre diffusion isobare (Trimix)
			\item Exposition à long terme \\ "Osteonecrosis" = mort des os
		\end{enumerate}
	\end{itemize}
\end{frame}

% O2
\subsubsection{Oxygène ($O_2$)}
\begin{frame}{\insertsubsubsection}
	\begin{itemize}
		\item Diatomique ($O_2$)
		\item Supporte la combustion
		\begin{itemize}
			\item Violemment à haute pression
		\end{itemize}
		\item Essentiel pour la vie
		\begin{itemize}
			\item Pas assez = hypoxie
			\item Trop = hyperoxie
		\end{itemize}
		\item Inodore, incolore
		\item Ininflammable (ne brûle pas)
	\end{itemize}
\end{frame}

\begin{frame}{\insertsubsubsection}
	\begin{description}
		\item[Anoxique]Sans oxygène
		\item[Hypoxique]Pression partielle d'oxygène plus faible
		\item[Normoxique]Pression partielle d'oxygène normale
		\item[Hyperoxique]Pression partielle d'oxygène plus élevée
	\end{description}
\end{frame}

\begin{frame}{Exposition à l'oxygène}
	\mypict{../pdf/misc/Graph_O2_Exposure.pdf}
\end{frame}

% CO2
\subsubsection{Dioxide de Carbone ($CO_2$)}
\begin{frame}{\insertsubsubsection}
	\begin{itemize}
		\item Déchet produit par le métabolisme
		\item 5\% problème respiratoire
		\item 10\% détresse respiratoire
		\item Favorisé par les "espace morts"
		\item 1\% de $CO_2$ dans l'air double le taux de $CO_2$ dans le sang à 40m
	\end{itemize}
\end{frame}

\begin{frame}{\insertsubsubsection}
	\mypict{../pdf/misc/Graph_CO2.pdf}
\end{frame}

% CO
\subsubsection{Monoxide de Carbone ($CO$)}
\begin{frame}{\insertsubsubsection}
	\begin{itemize}
		\item Extrêmement toxique
		\item Inodore, incolore
		\item Produit par combustion
		\begin{itemize}
			\item Lubrification du compresseur
			\item Mauvais placement du compresseur
		\end{itemize}
		\item Hemoglobine a 300x plus d'affinité avec le $CO$ que le $O_2$
	\end{itemize}
\end{frame}

% Helium
\subsubsection{Helium ($He$)}
\begin{frame}{\insertsubsubsection}
	\begin{itemize}
		\item Substitut à l'azote $N_2$ en plongée
		\item Diminue l'effort réspiratoire
		\item Réduit la narcose
		\item Gaz rare = cher!
		\item Conductivité thermique élevée ($\ne$combi étanche)
		\item HPNS:\\Syndrome Neurologique des Hautes Pression
		\item Tables spéciales nécessaires
		\item Limité dans la plongée récréative (prix)
		\item Recycleurs plus économiques
	\end{itemize}
\end{frame}

% COMEX
\subsubsection{Essais humains}
\begin{frame}{COMEX Hydra X (1992)}
	\begin{itemize}
		\item Theo Mavrostomos
		\item Profondeur de -701m en caisson hyperbare
		\item Hydreliox $0.5\% O_2$ / $71.5\% He$ / $28\% H2$
		\item Durée
		\begin{itemize}
			\item 4 semaines préparation pré-plongée
			\item 2 jours à 10m
			\item 13 jours de compression jusqu'à 675m
			\item 3 jours entre 650 et 675m avec une pointe à 701m
			\item 24 jours de décompression
		\end{itemize}
	\end{itemize}
\end{frame}

\begin{frame}{COMEX Hydra X (1992)}
	\mypict{../img/ComexHydraX.png}
\end{frame}

% Temperature 
\begin{frame}{Températures}
	\begin{description}[lalign=parleft]
		\item[$-183^{\circ}C$]Température d'ébullition de l'oxygène
		\item[$-196^{\circ}C$]Température d'ébullition de l'azote	
		\item[$-220^{\circ}C$]Oxygène et Azote sont solides
		\item[$-273^{\circ}C$]Zéro Abosolu (Aucune particule en mouvement)
	\end{description}
\end{frame}

\begin{frame}{Températures}
		\mypict{../pdf/misc/Graph_Temperatures.pdf}
\end{frame}

% Lois
\subsection{Lois des gaz}
\begin{frame}{\insertsubsection}
	Loi de Boyle-Mariotte
	\[	P \propto \frac{1}{V} \qquad (T=constant) \]
	Loi de Charles
	\[	V \propto T \qquad (P=constant)\]
	Loi de Gay-Lussac:
	\[ P \propto T \qquad (V=constant\]
	Loi des gaz idéaux
	\[	\frac{P_1 x V_1}{T_1} =\frac{P_2 x V_2}{T_2} = constante\]
\end{frame}

\begin{frame}{\insertsubsection}
	Loi de Dalton
	\[ P = PP1 + PP2 + PP3 +...+P \]
	Effet Joule-Thomson
	\begin{itemize}
		\item Phénomène lors duquel la température d'un gaz augmente lorsqu'il subit une compression adiabatique
	\end{itemize}
	Loi d'Henry
	\begin{itemize}
		\item A température constante et à saturation, la quantité de gaz dissous dans un liquide est proportionnelle à la pression partielle qu'exerce ce gaz sur le liquide.
	\end{itemize}
\end{frame}

%---------------------------------------------------------------------
%										REGULATORS
%---------------------------------------------------------------------

\section{Détendeurs}

%----- Design requirements -----
\subsection{Exigences de conception}

% Variables
\begin{frame}{Variables}
	\begin{itemize}
		\item Changement de pression ambiante (profondeur)
		\item Changement de volume respiratoire
		\item Augmentation du rythme respiratoire
		\item Augmentation de la densité de l'air
		\item Pertes de frottement
		\item Tendence à geler (effet Joules-Thompson)
	\end{itemize}
\end{frame}

% EN250
\begin{frame}{Standard EN250}
	\begin{tabular}{l l}
		Profondeur: 				& 0-65m\\
		Rythme réspiratoire:			& 0-25 inspirations/min\\
		Volume inspiratoire:			& 2.5 litres\\
		Cycle réspiratoire:			& Sinusoïdale\\
		Résistance inhalatoire:		& 25mbar (10 pouces d'eau)\\
		Résistance expiratoire:		& 25mbar (10 pouces d'eau)\\
		Effort réspiratoire:			& 3.0 joules par litre\\
		Temp. de stockage:			& $-40^{\circ}C$ à $80^{\circ}C$ \\
		Temp. de fonctionnement:	&\\
		1. Eau froide					& $0^{\circ}C$ à $50^{\circ}C$\\
		2. Eau chaude					& $10^{\circ}C$ à $50^{\circ}C$\\
		Pression:						& $35-200 bar$ (A-Clamp)\\
											& $35-300 bar$ (DIN)
	\end{tabular}
\end{frame}

% WOB
\begin{frame}{Effort respiratoire}
	\mypict{../pdf/misc/Breathing_Resistance_FR}
\end{frame}

% Compensation
\begin{frame}{Types de compensations}
	\begin{itemize}
		\item Compensation de pression ambiante
		\begin{itemize}
			\item Lié à la profondeur
			\item Augmentation de la pression intermédiaire
			\item Tous les détendeurs ont ce principe
		\end{itemize}
		\item Compensation de pression dans la bouteille
		\begin{itemize}
			\item Lié à la pression de la bouteille
			\item Diminution de la pression dans la bouteille entraine une diminution de la pression intermédiaire
			\item Compensation pour fournir une pression stable
			\item Seulement certains détendeurs (moyenne et haute gamme)
		\end{itemize}		
	\end{itemize}
\end{frame}

%----- PRINCIPLE -----
\subsection{Principes}

\begin{frame}{\insertsubsection}
	\begin{columns}[onlytextwidth]
		\column{0.6\linewidth}
			La force exercée sur un objet dépend de la surface de contact et de la pression appliquée\par
			Afin de créer un système stable, les forces exercées à chaque extrémités du piston doivent être identiques.
		\column{0.4\linewidth}
			\mypict{../pdf/1st/Piston_1} 
	\end{columns}
\end{frame}

\begin{frame}{\insertsubsection}
	\begin{columns}[onlytextwidth]
		\column{0.6\linewidth}
			Lorsque la pression intermédiaire est suffisamment élevée, le piston va fermer l'orifice et le système sera en équilibre.
		\column{0.4\linewidth}
			\mypict{../pdf/1st/Piston_1} 
	\end{columns}
\end{frame}

\begin{frame}{\insertsubsection}
	\begin{columns}[onlytextwidth]
		\column{0.6\linewidth}
			Definitions:
			\begin{align*}
				D &= Diametre [cm] \\
				S &= Surface [cm^2] \\
				  &= \tfrac{\pi*D^2}{4} \\
				\\
				P &= Pression [bar] \\
				F &= Force [\sfrac{kg}{cm^2}]\\
				  &= P*S \\		
			\end{align*}
		\column{0.4\linewidth}
			\mypict{../pdf/1st/Piston_1} 
	\end{columns}
\end{frame}

\begin{frame}{\insertsubsection}
	\begin{columns}[onlytextwidth]
		\column{0.6\linewidth}
			Orifice:
			\begin{align*}
				D_{orifice} &= 0.447[cm] \\
				S_{orifice} &= \tfrac{\pi*0.447^2}{4}\\
				  &= 0.157[cm^2] \\
				F_{orifice} &= 200*0.157\\
				  &= 31.4 [\sfrac{kg}{cm^2}]
			\end{align*}
		\column{0.4\linewidth}
			\mypict{../pdf/1st/Piston_1} 
	\end{columns}
\end{frame}

\begin{frame}{\insertsubsection}
	\begin{columns}[onlytextwidth]
		\column{0.6\linewidth}
			Piston:
			\begin{align*}
				D_{piston} &= 2[cm] \\
				S_{piston} &= \tfrac{20*{D_{p}}^2}{4}\\
				  &= 3.14[cm^2] \\
				F_{piston} &= F_{orifice}\\
				P_{piston} &= \tfrac{F_p}{S_p}\\
				  &= \tfrac{31.4}{3.14}\\
				  &= 10 [bar]
			\end{align*}
		\column{0.4\linewidth}
			\mypict{../pdf/1st/Piston_1} 
	\end{columns}
\end{frame}

\begin{frame}{\insertsubsection}
	\begin{columns}[onlytextwidth]
		\column{0.6\linewidth}
			Problème:
			\begin{itemize}
				\item Lorsque la pression dans la bouteille va diminuer, la pression intermédiaire va elle aussi diminuer.
				\item Le système créé est un diviseur de pression (rapport 1:20) et non un régulateur.
			\end{itemize}
		\column{0.4\linewidth}
			\mypict{../pdf/1st/Piston_1} 
	\end{columns}
\end{frame}

\begin{frame}{\insertsubsection}
	\begin{columns}[onlytextwidth]
		\column{0.6\linewidth}
			Afin de compenser la chute de pression dans la bouteille, un ressort est ajouté.\\
			$$F_{piston}=F_{orifice}+F_{ressort}$$
		\column{0.4\linewidth}
			\mypict{../pdf/1st/Piston_2} 
	\end{columns}
\end{frame}

\begin{frame}{\insertsubsection}
	\begin{columns}[onlytextwidth]
		\column{0.6\linewidth}
			La pression intermédiaire peut alors être ajustée en modifiant les caractéristiques du ressort.	
		\column{0.4\linewidth}
			\mypict{../pdf/1st/Piston_2} 
	\end{columns}
\end{frame}

\begin{frame}{\insertsubsection}
	\begin{columns}[onlytextwidth]
		\column{0.6\linewidth}
			Finalement, le piston est percé pour permettre à l'air de s'écouler de l'autre côté.
		\column{0.4\linewidth}
			\mypict{../pdf/1st/Piston_3} 
	\end{columns}
\end{frame}

\begin{frame}{\insertsubsection}
			\mypict{../pdf/1st/Bal_vs_Unbal}
\end{frame}

%---------------------------------------------------------------------
%										1st STAGES
%---------------------------------------------------------------------
\subsection{Premiers étage}

\begin{frame}{Différents types de conception }
	\begin{itemize}
		\item Piston non-compensé
		\item Piston compensé
		\item Diaphragme non-compensé
		\item Diaphragme compensé
	\end{itemize}
\end{frame}

% Unbalanced Piston
\subsubsection{Piston non-compensé}
\begin{frame}{\insertsubsubsection}
	\mypict{../pdf/1st/1st_piston_unbal}
\end{frame}

\begin{frame}{\insertsubsubsection}
	\mypict{../img/Scubapro_MK2}
\end{frame}

\begin{frame}{\insertsubsubsection}
	\begin{itemize}
		\item Pour
		\begin{itemize}
			\item Bon marché
			\item Facilité de réparation
			\item Solide et durable
			\item Parfait pour les centres de plongées
		\end{itemize}
		\item Contre
		\begin{itemize}
			\item Performances limitées
		\end{itemize}
	\end{itemize}
\end{frame}

\begin{frame}{\insertsubsubsection}
	\begin{itemize}
		\item Modèles
		\begin{itemize}
			\item Aqualung Calypso
			\item Scubapro MK2
			\item US Divers Conshelf
		\end{itemize}
	\end{itemize}
\end{frame}

% Balanced Piston
\subsubsection{Piston compensé}
\begin{frame}{\insertsubsubsection}
	\mypict{../pdf/1st/1st_piston_bal}
\end{frame}

\begin{frame}{\insertsubsubsection}
	\mypict{../img/Scubapro_MK25}
\end{frame}

\begin{frame}{\insertsubsubsection}
	\mypict{../img/Scubapro_MK25_exploded}
\end{frame}

\begin{frame}{\insertsubsubsection}
	\begin{itemize}
		\item Pour
		\begin{itemize}
			\item Sûr et efficace
			\item Excellentes performances
			\item Solide
		\end{itemize}
		\item Contre
		\begin{itemize}
			\item Cher
			\item Sujet au givrage ("flow-through" piston)
			\item Compliqué
		\end{itemize}
	\end{itemize}
\end{frame}

\begin{frame}{\insertsubsubsection}
	\begin{itemize}
		\item Modèles
		\begin{itemize}
			\item Halcyon H-75P
			\item Scubapro MK20, MK25
		\end{itemize}
	\end{itemize}
\end{frame}

% Unbalanced Diaphragm
\subsubsection{Diaphragme non-compensé}
\begin{frame}{\insertsubsubsection}
	\mypict{../pdf/1st/1st_dia_unbal}
\end{frame}

\begin{frame}{\insertsubsubsection}
	\begin{itemize}
		\item Peu commun
		\item Pour
		\begin{itemize}
			\item Augmentation de la pression intermédiaire lorsque la pression dans la bouteille diminue
		\end{itemize}
		\item Contre
		\begin{itemize}
			\item Nécessite un deuxième étage pouvant gérer une pression plus élevée
			\item Plus difficile à respirer en début de plongée
		\end{itemize}
	\end{itemize}
\end{frame}

\begin{frame}{\insertsubsubsection}
	\begin{itemize}
		\item Modèles
		\begin{itemize}
			\item Aucun sur le marché actuel
		\end{itemize}
	\end{itemize}
\end{frame}

% Balanced Diaphragm
\subsubsection{Diaphragme compensé}
\begin{frame}{\insertsubsubsection}
	\mypict{../pdf/1st/1st_dia_bal}
\end{frame}

\begin{frame}{\insertsubsubsection}
	\mypict{../img/Scubapro_MK17}
\end{frame}

\begin{frame}{\insertsubsubsection}
	\mypict{../img/Scubapro_MK17_exploded}
\end{frame}

\begin{frame}{\insertsubsubsection}
	\begin{itemize}
		\item Détendeur le plus commun
		\item Pour
		\begin{itemize}
			\item Simple mais efficace
			\item Facilement utilisable en eau froide
		\end{itemize}
		\item Contre
		\begin{itemize}
			\item Ne supporte pas être noyés
		\end{itemize}
	\end{itemize}
\end{frame}

\begin{frame}{\insertsubsubsection}
	\begin{itemize}
		\item Modèles
		\begin{itemize}
			\item Apeks (tous les modèles)
			\item Aqualung Legend, Glacia,...
			\item Halcyon H-50D
			\item Mares Abyss, Prestige,...
			\item Scubapro MK11, MK17
			\item Poseidon Jetstream
		\end{itemize}
	\end{itemize}
\end{frame}

\begin{frame}{\insertsubsubsection}
	\begin{columns}[onlytextwidth]
		\column{0.6\linewidth}
			\begin{itemize}
				\item Poseidon Xstream\\
				\begin{itemize}
					\item Le siège est remplacé par une bille fermant l'orifice
					\item L'aiguille agit sur la bille de la même manière que sur le siège
					\item Un ressort maintien la bille en place
				\end{itemize}
			\end{itemize}
		\column{0.4\linewidth}
			\mypict{../img/Poseidon_Xstream}
	\end{columns}
\end{frame}

% Options
\subsubsection{Options des premiers étages}
\begin{frame}{\insertsubsubsection}
	\begin{itemize}
		\item Système anti-givrage
		\begin{itemize}
			\item Chambre à pression ambiante remplie de graisse
			\item Capuchon rempli de graisse
			\item Diaphragme rempli d'huile
			\item Diaphragme avec piston ou "load transmitter" \\ (le plus commun)
			\item Chambre à pression ambiante sèche "Dry bleed" (fuite contrôlée du premier étage)
			\item "Préservatif et vodka"!
		\end{itemize}
	\end{itemize}
\end{frame}

%---------------------------------------------------------------------
%										2nd STAGES
%---------------------------------------------------------------------
\subsection{Deuxièmes étages}

% Differents types
\begin{frame}{Différents types de conception }
	\begin{itemize}
		\item Amont à vanne d'inclinaison
		\item Aval à piston non-compensé
		\item Aval à piston compensé
		\item Servo ou valve pilotée
	\end{itemize}
\end{frame}

% Upstream
\subsubsection{Amont à vanne d'inclinaison}
\begin{frame}{\insertsubsubsection}
	\mypict{../pdf/2nd/2nd_upstream}
\end{frame}

\begin{frame}{\insertsubsubsection}
	\begin{itemize}
		\item Pour
		\begin{itemize}
			\item Simple
		\end{itemize}
		\item Contre
		\begin{itemize}
			\item Conception en amont \\ Nécessite une valve de surpression (PRV)
			\item Performance limitées
		\end{itemize}
	\end{itemize}
\end{frame}

\begin{frame}{\insertsubsubsection}
	\begin{itemize}
		\item Modèles
		\begin{itemize}
			\item Aucun sur le marché actuel
		\end{itemize}
	\end{itemize}
\end{frame}


% Downstream unbalanced
\subsubsection{Aval à piston non-compensé}
\begin{frame}{\insertsubsubsection}
	\mypict{../pdf/2nd/2nd_DS_unbal}
\end{frame}

\begin{frame}{\insertsubsubsection}
	\begin{itemize}
		\item Pour
		\begin{itemize}
			\item Bon marché
			\item Facilité d'entretien
			\item Parfait pour la location
			\item Conception en aval \\ (Joue le rôle de valve de surpression)
		\end{itemize}
		\item Contre
		\begin{itemize}
			\item Performances limitées
		\end{itemize}
	\end{itemize}
\end{frame}

\begin{frame}{\insertsubsubsection}
	\begin{itemize}
		\item Modèles
		\begin{itemize}
			\item Aqualung Calypso
			\item Mares Prestige
			\item Scubapro R095, R195,...
			\item US Divers Conshelf
		\end{itemize}
	\end{itemize}
\end{frame}


% Downstream balanced
\subsubsection{Aval à piston compensé}
\begin{frame}{\insertsubsubsection}
	\mypict{../pdf/2nd/2nd_DS_bal}
\end{frame}

\begin{frame}{\insertsubsubsection}
	\mypict{../img/Scubapro_S600_exploded}
\end{frame}

\begin{frame}{\insertsubsubsection}
	\begin{itemize}
		\item Pour
		\begin{itemize}
			\item Sûr et efficace
			\item Bonnes performances
			\item Facilement ajustable
		\end{itemize}
		\item Contre
		\begin{itemize}
			\item Cher
			\item Compliqué à entretenir
		\end{itemize}
	\end{itemize}
\end{frame}

\begin{frame}{\insertsubsubsection}
	\begin{itemize}
		\item Modèles
		\begin{itemize}
			\item Apeks (tous les modèles)
			\item Aqualung Legend, Glacia
			\item Scubapro G250, S600, A700,...
		\end{itemize}
	\end{itemize}
\end{frame}

% Servo
\subsubsection{Servo ou valve pilotée}
\begin{frame}{\insertsubsubsection}
	\mypict{../img/Poseidon}
\end{frame}

\begin{frame}{\insertsubsubsection}
	\begin{itemize}
		\item Pour
		\begin{itemize}
			\item Sûr et efficace
			\item Bonnes performances
			\item Ne givrent pas \\ (peu de pièces en métal)
		\end{itemize}
		\item Contre
		\begin{itemize}
			\item Compliqués
			\item Conception en amont \\ Nécessite une valve de surpression (PRV)
			\item Résistance à l'inspiration élevée
		\end{itemize}
	\end{itemize}
\end{frame}

\begin{frame}{\insertsubsubsection}
	\begin{itemize}
		\item Modèles
		\begin{itemize}
			\item Poseidon Jetstream et Xstream
		\end{itemize}
	\end{itemize}
\end{frame}

% Options
\subsubsection{Options des deuxièmes étages}
\begin{frame}{\insertsubsubsection}
	\begin{itemize}
		\item Effet Venturi
		\begin{itemize}
			\item Réduit l'effort respiratoire
			\item Tendence à créer un débit continu
			\item Déflecteur dans le boitier du deuxième étage modifiant le trajet du flux d'air
		\end{itemize}
	\end{itemize}
\end{frame}

%---------------------------------------------------------------------
%										INFLATORS
%---------------------------------------------------------------------
\section{Inflateurs}

% Tuyaux
\subsection{Valve Shrader}
\begin{frame}{\insertsubsection}
	\begin{columns}[onlytextwidth]
		\column{0.3\linewidth}
			\mypict{../img/Schrader_valve}
		\column{0.7\linewidth}
			Les tuyaux d'inflateurs utilisent un système de valve Schrader.\\
			Une petite valve amont qui s'ouvre lorsque l'on presse dessus.\\
			Lorsque le tuyau est deconnecté, la pression ferme la valve et empêche l'air de sortir.\\
			Aussi utilisé pour les pneus de vélo!
	\end{columns}
\end{frame}

% Drysuit
\subsection{Combinaison étanche}
\begin{frame}{\insertsubsection}
	\mypict{../img/Drysuit-valve}
\end{frame}

\begin{frame}{\insertsubsection}
	\mypict{../pdf/misc/DrySuit-Valve}
\end{frame}

%---------------------------------------------------------------------
%										VALVES
%---------------------------------------------------------------------
\section{Valves}

%----- CONCEPTION -----
\subsection{Conception}

% K-Valve
\begin{frame}{K-Valve avec disque de rupture}
	\mypict{../img/K-Valve}
\end{frame}

\begin{frame}{K-Valve avec disque de rupture}
	\mypict{../pdf/misc/Valve}
\end{frame}

\begin{frame}{K-Valve avec disque de rupture}
	\mypict{../img/Valve_exploded}
\end{frame}

% Isolateur
\begin{frame}{Isolateur}
	\mypict{../img/Manifold}
\end{frame}

\begin{frame}{Isolateur}
	\mypict{../pdf/misc/Valve_Manifold}
\end{frame}

% J-Valve
\begin{frame}{J-Valve}
	\mypict{../img/J-Valve}
\end{frame}

\begin{frame}{J-Valve}
	\mypict{../pdf/misc/J-Valve}
\end{frame}

% H-Valve
\begin{frame}{H-Valve}
	\mypict{../img/H-Valve}
\end{frame}

% Y-Valve
\begin{frame}{Y-Valve}
	\mypict{../img/Y-Valve}
\end{frame}

% J-Valve
\begin{frame}{J-Valve avec isolateur}
	\mypict{../img/Manifold-old}
\end{frame}

%----- CONNEXION -----
\subsection{Sorties}

\begin{frame}{Etrier}
	\begin{columns}[onlytextwidth]
		\column{0.4\linewidth}
			\mypict{../img/k_outlet}
		\column{0.6\linewidth}
			\begin{itemize}
			 	\item Etrier, Yoke, Int, A-Clamp
				\item Joint torique pincé entre la robinetterie et le détendeur
				\item Pression maximale de 232bar.
			\end{itemize}
	\end{columns}
\end{frame}

\begin{frame}{DIN}
	\begin{columns}[onlytextwidth]
		\column{0.4\linewidth}
			\mypict{../img/din_outlet}
		\column{0.6\linewidth}
			\begin{itemize}
			 	\item DIN $G\sfrac{5}{8}"$ ($22.9mm$)
			 	\item Joint torique compressé dans la robinetterie
			 	\item Détendeur vissé dans la robinetterie, connexion extrêment solide.
			 	\item 5 filletage pour 232 bar
			 	\item 7 filletage pour 300 bar
			\end{itemize}	 
	\end{columns}
\end{frame}

\begin{frame}{Insert}
	\begin{columns}[onlytextwidth]
		\column{0.4\linewidth}
			\mypict{../img/pro_outlet}
		\column{0.6\linewidth}
			\begin{itemize}
			 	\item Converti une robinettrie DIN en yoke
			 	\item Insert vissé dans la  robinetterie
			 	\item Plusieurs types d'insert et de valve\\Compatibilité non garantie!
			\end{itemize}
	\end{columns}
\end{frame}

\begin{frame}{DIN M26}
	\begin{columns}[onlytextwidth]
		\column{0.4\linewidth}
			\mypict{../img/din_m26}
		\column{0.6\linewidth}
			\begin{itemize}
				\item Robinetterie DIN dont le diamètre est plus important ($26mm$)
				\item Utilisé pour des bouteilles contenant de l'oxygène ou du nitrox.
			\end{itemize}
	\end{columns}
\end{frame}

\begin{frame}{CGA540}
	\begin{itemize}
		\item Robinetterie utilisée pour les bouteilles d'oxygène pur
		\item Connexion métal sur métal (laiton)
		\item Pas de joint torique
		\item Outil nécessaire pour serrer correctement et assurer l'etanchéité
	\end{itemize}
	\begin{columns}
		\column{0.4\linewidth}
			\mypict{../img/CGA540_fem}
		\column{0.4\linewidth}
			\mypict{../img/CGA540_mal}
	\end{columns}
\end{frame}

%----- Threads -----
\subsection{Filetages}

\begin{frame}{Filetage bouteille-valve}
	Europe:
	\begin{itemize}
		\item $M25*2mm$ DIN (Deutsche Industrial Norm)\\
			{\o}$=25mm$, $2mm$ entre chaque filetage\\
			\hfill
		\item $M18*1.5mm$ DIN (Deutsche Industrial Norm)\\
			{\o}$=18mm$, $1.5mm$ entre chaque filetage\\
			\hfill
		\item $G \sfrac{3}{4}*14$ BSP (British Standard Pipe)\\
			{\o}$=26.4mm$, 14 filetages par pouce\\
			\hfill
	\end{itemize}
	USA (et autre...):
	\begin{itemize}
		\item $\sfrac{3}{4}"*14$ NPSM\\
			(National Pipe Straight Mecanical)\\
			{\o}$=26.4mm$, 14 filetages par pouce\\
			\hfill
	\end{itemize}
\end{frame}

\begin{frame}{Filetage bouteille-valve}
	\begin{block}{ATTENTION}
		\begin{itemize}
			\item British $G \sfrac{3}{4}"$ et American $\sfrac{3}{4}"$ NPSM sont très proches, mais 	pas compatible.
			\item	La forme du filetage n'est pas la même.
		\end{itemize}
	\end{block}
\end{frame}

%---------------------------------------------------------------------
%										CYLINDERS
%---------------------------------------------------------------------
\section{Bouteilles}

%----- MATERIAL -----
\subsection{Matériaux}

% ALU
\begin{frame}{Alliage d'aluminium}
	\begin{itemize}
		\item Avantages:
			\begin{itemize}
				\item Ne rouille pas!
				\item Pas besoin d'êtres peintes
			\end{itemize}
		\item Inconvénients
			\begin{itemize}
				\item Flottabilité positive à vide
				\item Corrosion galvanique avec le laiton (robinetterie) et l'acier (cerclages)
				\item Température max: $150^\circ C$
			\end{itemize}
	\end{itemize}
\end{frame}

% acier
\begin{frame}{Acier}
	\begin{itemize}
		\item Avantages:
			\begin{itemize}
				\item Solide
				\item Flottabilité négative à vide (moins de lest nécessaire)
				\item Pression d'usage plus élevée ($232bar$)
				\item Plus légère
				\item Température max: $300^\circ C$
			\end{itemize}
		\item Inconvénients
			\begin{itemize}
				\item Rouille!!!
				\item Doivent être peintes
			\end{itemize}
	\end{itemize}
\end{frame}

% acier-carbone
\begin{frame}{Fibre de carbone}
	\begin{itemize}
		\item Avantages:
			\begin{itemize}
				\item Haute pression ($300bar$)
				\item Léger
			\end{itemize}
		\item Inconvénients
			\begin{itemize}
				\item Fragile!
				\item Inspection visuelle et test hydrostatique spéciaux
			\end{itemize}
	\end{itemize}
\end{frame}

%----- COMPARISON -----
\subsection{Comparaison}
\begin{frame}{\insertsubsection}
	\footnotesize
	\begin{multicols}{2}
		\textbf{Aluminium}\\
		Worthington AL80\\
		\begin{tabularx}{\linewidth}{X r}
			Pression:		& 	$207 bar$	\\
			Volume:			&	$11.1L$		\\
			Poids à vide:	&	$14.5kg$	\\
			Flotta.	à vide:	&	$+1.45kg$	\\
			Flotta. à $200bar$:	&	$-0.81kg$
		\end{tabularx}

		\textbf{Acier}\\
		Worthington 80\\
		\begin{tabularx}{\linewidth}{X r}
			Pression:		& 	$237 bar$	\\
			Volume:			&	$11.1L$		\\
			Poids à vide:	&	$12.7kg$	\\
			Flotta. à vide:	&	$-1.36kg$	\\
			Flotta. à $200bar$:	&	$-4.08kg$
		\end{tabularx}
	\end{multicols}

	\vspace{1em}

	\begin{multicols}{2}
		\textbf{Aluminium}\\
		Catalina S80\\
		\begin{tabularx}{\linewidth}{X r}
			Pression:		& 	$207 bar$	\\
			Volume:			&	$11.1L$		\\
			Poids à vide:	&	$14.2kg$	\\
			Flotta.	à vide:	&	$+1.85kg$	\\
			Flotta. à $200bar$:	&	$-0.72kg$
		\end{tabularx}

		\textbf{Aluminium}\\
		Luxfer AL80\\
		\begin{tabularx}{\linewidth}{X r}
			Pression:		& 	$207 bar$	\\
			Volume:			&	$11.1L$		\\
			Poids à vide:	&	$14.2kg$	\\
			Flotta. à vide:	&	$+1.54kg$	\\
			Flotta. à $200bar$:	&	$-0.63kg$
		\end{tabularx}
	\end{multicols}
\end{frame}

\end{document}