\documentclass[aspectratio=1610,english,12pt]{beamer}

\usepackage{../../shared/tex/mystyle}

\author[]{Chris Braissant}
\title[]{SDI Equipment Specialist}
\institute{Ban's Diving Resort}

%---------------------------------------------------------------------
%										DOCUMENT
%---------------------------------------------------------------------
\begin{document}

\begin{frame}[plain]
	\maketitle
\end{frame}

\section{Introduction}

\subsection{Overview}
\begin{frame}{\insertsubsection}
	The aim of the course is to see and understand the principles of operation of the diving equipment\\
	The course will cover various subject as maintenance and repair of regulators, valves, BCD's, cylinders and dry suit.
\end{frame}

\subsection{Paperwork}
\begin{frame}{\insertsubsection}
	Liability Release and Express Assumption of Risk\\
	Medical Statement
\end{frame}

\subsection{Liability Release}
\begin{frame}{\insertsubsection}
	\begin{columns}[onlytextwidth]
		\column{0.6\linewidth}
			\mypict{../pdf/misc/Liability}
		\column{0.4\linewidth}
			\begin{enumerate}\itemsep0em 
				\item Equipment Specialist
				\item Your name
				\item Instr.: Chris Braissant
				\item Facility: Ban's diving
				\item Your name
				\item Instr.: Chris Braissant
				\item Facility: Ban's Diving
				\item - blank -
				\item Training agency:\\TDI/SDI
				\item Signature and date
				\item Witness signature and date
				\item Your initials
			\end{enumerate}
	\end{columns}
\end{frame}

%---------------------------------------------------------------------
%										THEORY
%---------------------------------------------------------------------
\section{Basic Theory}

% ----- UNITS -----	
\subsection{Units Conversion}

\begin{frame}{Units systems}  
	The two most common units system used worldwide are:
	\begin{itemize}
		\item The International System of Units (or metric system)
		\item The British Imperial System (USA, UK, Myanmar)
	\end{itemize} 
\end{frame}

% SI Units
\begin{frame}{International System of Units}
	The metric system is based on seven independant SI units.\\
	Every other quantities are derivated from those seven SI units.\\
	\vspace{1em}
	\centering
	\begin{tabular}{| l | l | l |}
		\hline
		Base quantity & Unit & Symbol\\ \hline
		Length & meter & [m]\\
		Mass & kilogram & [kg]\\
		Time & second & [s]\\
		Electric current & Ampere & [A]\\
		Temperature & Kelvin & [K]\\
		Amount of substance & mole & [mol]\\
		Luminous intenstiy & candela & [cd]\\
		\hline
	\end{tabular}
\end{frame}

% Longeur
\begin{frame}{Length}
	SI Units:
	\[
		1[mm] \xrightarrow{x10} 1[cm] \xrightarrow{x10} 1[dm] \xrightarrow{x10} 1[m]
	\]
	Imperial Units:
	\[
		1[inch] \xrightarrow{x12} 1[foot] \xrightarrow{x3} 1[yard] \xrightarrow{x1760} 1[mile]
	\]
	Conversion:
	\begin{align*}
				1[inch] &= 2.54[cm]\\
				1[foot] &= 30.48[cm]\\
				1[yard] &= 0.914[cm]\\
				1[mile] &= 1.6[km]
	\end{align*}			
\end{frame}

\begin{frame}{SI vs Imperial}  
	\mypict{../img/imperial-units}
\end{frame}

% Area
\begin{frame}{Area}
	SI:
	\begin{align*}
		1m \times 1m &= 1[m^2]\\
		1cm \times 1cm &= 1[cm^2]
	\end{align*}
	Imperial:
	\begin{align*}
		1" \times 1" &= 1[sq\ in]
	\end{align*}
	Conversion:
	\begin{alignat*}{2}
				1[sq\ in] &= 2.54 \times 2.54	&	&= 6.45[cm^2]\\
	\end{alignat*}
\end{frame}

% Volume
\begin{frame}{Volume}
	SI Units:
	\begin{alignat*}{2}
				1[m] \times 1[m] \times 1[m] &= 1[m^3] \\
				1[dm] \times 1[dm] \times 1[dm] &= 1[dm^3] = 1 liter\\
				1[cm] \times 1[cm] \times 1[cm] &= 1[cm^3]\\
	\end{alignat*}
	Imperial:
	\begin{alignat*}{3}
				1[inch] \times 1[inch] \times 1[inch] &= 1" \times 1"	 \times 1"	&	&= 1[cu\ in]\\
				1[foot] \times 1[foot] \times 1[foot] &= 1' \times 1'	 \times 1'	&	&= 1[cu\ ft]\\
	\end{alignat*}
	Conversion:
	\begin{alignat*}{2}
				1[cu\ ft] &= 30.48 \times 30.48 \times 30.48 &	&= 28.3 [dm^3] = 28.3 liter\\
	\end{alignat*}
\end{frame}

% Poids et masse
\begin{frame}{Mass and weight}
	Mass\\
	\begin{tabular}{p{0.3\linewidth} l}
		$1 kg$ & $\approx 2.2 lbs$ \\
		$1 lbs$ & $\approx 0.45 kg$ \\
		$1 m^3$ & $1000 kg$ of fresh water\\
		& $1030 kg$ of salt water\\
		& $ 1.239 kg$ of air
	\end{tabular}
	\\
	\begin{tabular}{p{0.3\linewidth} l}
		$1kg$ & $\approx9.81N$
	\end{tabular}
\end{frame}

% Pression
\begin{frame}{Pressure}
	Pascal\\
	\begin{tabular}{p{0.2\linewidth} p{0.4\linewidth} l}
		$1 Pa$ & $1 Newton/m^2$ & $0.1kg/m^2$
	\end{tabular}\\
	Bar\\
	\begin{tabular}{p{0.2\linewidth} p{0.4\linewidth} l}
		$1 bar$ & $100'000Pa$ & $0.1MPa$\\
		$1 bar$ & $0.981kg/cm^2$ & $\approx1kg/cm^2$
	\end{tabular}\\
	PSI\\
	\begin{tabular}{p{0.2\linewidth} p{0.4\linewidth} l}
		$1 PSI$ & $1lbs/sqi$&\\
		$1 PSI$ & $0.45kg/6.45cm^2$ & $0.068bar$\\
		$\rightarrow$ & $1bar$ & $14.5PSI$
	\end{tabular}
\end{frame}

% Temperature
\begin{frame}{Temperature}
	\mypict{../pdf/misc/Temperatures_Units.pdf} \par
	Formulas:
		\begin{align*}
			[^{\circ}C] &= [K]-273 & [K] &= [^{\circ}C]+273\\
			[^{\circ}C] &= \frac{5}{9}*([^{\circ}F]-32) & [^{\circ}F] &= \frac{9}{5}*[^{\circ}C]+32 &
		\end{align*}
\end{frame}

%------ PROPERTIES OF GASES -----
\subsection{Properties of gases}

% Air
\subsubsection{Air}
\begin{frame}{Composition of the air}
	\begin{description}[lalign=parleft]
		\item[$78.084\%$]Nitrogen ($N_2$)
		\item[$20.946\%$]Oxygen ($O_2$)
		\item[$0.934\%$]Argon ($Ar$)
		\item[$0.033\%$]Carbon Monoxide ($CO$)
		\item[$0.003\%$]Other gases
		\item[]
		\item[$18.18ppm$]Neon ($Ne$)
		\item[$5.24ppm$]Helium ($He$)
		\item[$1.14ppm$]Krypton ($Kr$)
		\item[$0.09ppm$]Xenon ($Xe$)
	\end{description}
\end{frame}

\begin{frame}{Breathing gases}
	\begin{description}[lalign=parleft]
		\item[Nitrogen ($N_2$)]Narcotic
		\item[Oxygen ($O_2$)]Toxic at high $ppO_2$
		\item[Helium ($He$)]Distortion of voice 
		\item[Neon ($Ne$)] Dense, long decompression, expensive 
		\item[Hydrogen ($H$)] Explosive with more than $4\%$ of Oxygen
		\item[Argon ($Ar$)]Narcotic, dense, hard to breathe
		\item[Xenon ($Xe$)]Narcotic, anesthetic, expensive
	\end{description}
\end{frame}

% Mix
\begin{frame}{Gas mixing}
	\begin{description}
		\item[Nitrox]Nitrogen and Oxygen
		\item[Heliox]Helium and Oxygen
		\item[Trimix]Oxygen, Helium and Nitrogen (ex; TMX 18/45)
		\item[Heliair]Helium and Air ("Poor man's trimix")
		\item[Hydrox]Hydrogen and Oxygen
 		\item[Hydreliox]Hydrogen, Helium and Oxygen
	\end{description}
\end{frame}

% N2
\subsubsection{Nitrogen ($N_2$)}
\begin{frame}{\insertsubsubsection}
	\begin{itemize}
		\item Diatomic ($N_2$)
		\item Physiological inert
		\item High Density
		\begin{itemize}\item High respiratory effort at depth\end{itemize}
		\item Narcotic effect
		\begin{itemize}\item At high partial pressure ($ppN_2 > 3.2$)\end{itemize}
		\item Decompression sickness
		\begin{itemize}\item Unmetabolized by the body - Bubble formation\end{itemize}
		\item Doesn't dissolve in water
		\begin{itemize}\item but easily in grease and oil (fat tissue)\end{itemize}
	\end{itemize}
\end{frame}

\begin{frame}{\insertsubsubsection}
	\begin{itemize}
		\item Decompression sickness
		\begin{enumerate}[I]
			\item Pain
			\item CNS (Central Nervous System) and pain
			\item Severe type I and II \\ Suffocation \\ Isobaric Counter Diffusion (Trimix)
			\item Long term exposure \\ "Osteonecrosis" = bones' death
		\end{enumerate}
	\end{itemize}
\end{frame}

% O2
\subsubsection{Oxygen {$O_2$}}
\begin{frame}{\insertsubsubsection}
	\begin{itemize}
		\item Diatomic ($O_2$)
		\item Support combustion
		\begin{itemize}
			\item Violently at high pressure
		\end{itemize}
		\item Essentiel for life
		\begin{itemize}
			\item Not enough = Hypoxia
			\item Too much = Hyperoxia
		\end{itemize}
		\item Odorless, colorless, tasteless
		\item NON-flammable
	\end{itemize}
\end{frame}

\begin{frame}{\insertsubsubsection}
	\begin{description}
		\item[Anoxic]No oxygen at all
		\item[Hypoxic]Lower than normal partial pressure of oxygen
		\item[Normxic]Normal partial pressure of oxygen
		\item[Hyperoxic]Higher partial pressure of oxygen
	\end{description}
\end{frame}

\begin{frame}{Oxygen Exposure}
	\mypict{../pdf/misc/Graph_O2_Exposure.pdf}
\end{frame}

% CO2
\subsubsection{Carbon Dioxide ($CO_2$)}
\begin{frame}{\insertsubsubsection}
	\begin{itemize}
		\item Waste product of the metabolism
		\item 5\% respiratory problem
		\item 10\% respiratory distress
		\item Enhanced by "Dead space"
		\item 1\% of $CO_2$ in the air double $CO_2$ level in the blood at 40m
	\end{itemize}
\end{frame}

\begin{frame}{\insertsubsubsection}
	\mypict{../pdf/misc/Graph_CO2.pdf}
\end{frame}

% CO
\subsubsection{Carbon Monoxide ($CO$)}
\begin{frame}{\insertsubsubsection}
	\begin{itemize}
		\item Extremely toxic
		\item Odorless, colorless, tasteless
		\item Product of combustion
		\begin{itemize}
			\item Poor compressor lubrication
			\item Poor compressor positioning
		\end{itemize}
		\item Hemoglobin bounds 300x easier with $CO$ than with $O_2$
	\end{itemize}
\end{frame}

% Helium
\subsubsection{Helium ($He$)}
\begin{frame}{\insertsubsubsection}
	\begin{itemize}
		\item Substitut for Nitrogen $N_2$ in deep diving
		\item Lower respiratory effort
		\item Reduce narcosis
		\item Rare gas = expensive!
		\item High thermal conductivity ($\ne$dry suit)
		\item HPNS:\\High Pressure Neurological Syndrom
		\item Special dive tables compulsary
		\item Limited in recreational diving (price)
		\item Rebreather more economical
	\end{itemize}
\end{frame}

% COMEX
\subsubsection{Human trials}
\begin{frame}{COMEX Hydra X (1992)}
	\begin{itemize}
		\item Theo Mavrostomos
		\item Depth of -701m in recompression chamber
		\item Hydreliox $0.5\% O_2$ / $71.5\% He$ / $28\% H_2$
		\item Duration
		\begin{itemize}
			\item 4 weeks of preparation pre-dive
			\item 2 days at 10m
			\item 13 days of compression to -675m
			\item 3 days between -650 and -675m with a peak to -701m
			\item 24 days of decompression
		\end{itemize}
	\end{itemize}
\end{frame}

\begin{frame}{COMEX Hydra X (1992)}
	\mypict{../img/ComexHydraX.png}
\end{frame}

% Temperature 
\begin{frame}{Temperatures}
	\begin{description}[lalign=parleft]
		\item[$-183^{\circ}C$]Oxygen's boiling point
		\item[$-196^{\circ}C$]Nitrogen's boiling point	
		\item[$-220^{\circ}C$]Oxygen and Nitrogen are solid
		\item[$-273^{\circ}C$]Absolut Zero (No particule is moving)
	\end{description}
\end{frame}

\begin{frame}{Temperatures}
		\mypict{../pdf/misc/Graph_Temperatures.pdf}
\end{frame}

% Lois
\subsection{Gas laws}
\begin{frame}{\insertsubsection}
	Boyle-Mariotte's law
	\[	P \propto \frac{1}{V} \qquad (T=constant) \]
	Charles' law
	\[	V \propto T \qquad (P=constant)\]
	Gay-Lussac's law:
	\[ P \propto T \qquad (V=constant)\]
	Ideal gas law
	\[	\frac{P_1 x V_1}{T_1} =\frac{P_2 x V_2}{T_2} \]
\end{frame}

\begin{frame}{\insertsubsection}
	Dalton's law
	\[ P = PP1 + PP2 + PP3 +... \]
	Joule-Thomson effect
	\begin{itemize}
		\item Phenomenon in which the temperature increases when a gas undergoes adiabatic compression
	\end{itemize}
	Henry's law
	\begin{itemize}
		\item At constant temperature and at saturation, the amount of gas dissolve in a liquid is proportional to the partial pressure exerted by the gas on the liquid.
	\end{itemize}
\end{frame}

%---------------------------------------------------------------------
%										REGULATORS
%---------------------------------------------------------------------

\section{Regulators}

%----- Design requirements -----
\subsection{Design requirements}

% Variables
\begin{frame}{Variables}
	\begin{itemize}
		\item Change in ambiant pressure (depth)
		\item Change in respiratory volume
		\item Increase in breathing rate
		\item Increase in air density
		\item Friction losses
		\item Tendency to freeze (Joules-Thompson effect)
	\end{itemize}
\end{frame}

% EN250
\begin{frame}{Standard EN250}
	\begin{tabular}{l l}
		Depth: 				& 0-65m\\
		Respiratory rate:			& 0-25 inspirations/min\\
		Respiratory volume:		& 2.5 liters\\
		Respiratory cycle:		& Sinusoidale\\
		Inhalation resistance:	& 25 mbar (10 inch of water)\\
		Exhalation resistance:	& 25 mbar (10 inch of water)\\
		Work of breathing:		& 3.0 joules per litre\\
		Storage temperature:		& $-40^{\circ}C$ to $80^{\circ}C$ \\
		Working temperature:		&\\
		1. Cold water 				& $0^{\circ}C$ to $50^{\circ}C$\\
		2. Warm water				& $10^{\circ}C$ to $50^{\circ}C$\\
		Pressure:					& $35-200 bar$ (A-Clamp)\\
										& $35-300 bar$ (DIN)
	\end{tabular}
\end{frame}

% WOB
\begin{frame}{Work of breathing}
	\mypict{../pdf/misc/Breathing_Resistance_EN}
\end{frame}

% Compensation
\begin{frame}{Types of compensation}
	\begin{itemize}
		\item Compensation of ambiant pressure
		\begin{itemize}
			\item Linked to the depth
			\item Increase the intermediate pressure
			\item All regulators are made that way
		\end{itemize}
		\item Compensation of pressure loss
		\begin{itemize}
			\item Linked to the pressure in the tank
			\item Reduced pressure in the cylinder causes a decrease in the intermediate pressure 
			\item Compensation to provide a stable pressure
			\item Only some regulators (Balanced regulators)
		\end{itemize}		
	\end{itemize}
\end{frame}

%----- PRINCIPLE -----
\subsection{Principle}

\begin{frame}{\insertsubsection}
	\begin{columns}[onlytextwidth]
		\column{0.6\linewidth}
			The force exerted on an object depends on the surface and the pressure applied.\par
			To create a stable system, the forces at each end of the piston must be identical.
		\column{0.4\linewidth}
			\mypict{../pdf/1st/Piston_1} 
	\end{columns}
\end{frame}

\begin{frame}{\insertsubsection}
	\begin{columns}[onlytextwidth]
		\column{0.6\linewidth}
			When the intermediate pressure is high enough, the piston will close the orifice and the system will be stable.
		\column{0.4\linewidth}
			\mypict{../pdf/1st/Piston_1} 
	\end{columns}
\end{frame}

\begin{frame}{\insertsubsection}
	\begin{columns}[onlytextwidth]
		\column{0.6\linewidth}
			Definitions:
			\begin{align*}
				D &= Diameter [cm] \\
				S &= Surface [cm^2] \\
				  &= \tfrac{\pi*D^2}{4} \\
				\\
				P &= Pressure [bar] \\
				F &= Force [\sfrac{kg}{cm^2}]\\
				  &= P*S \\		
			\end{align*}
		\column{0.4\linewidth}
			\mypict{../pdf/1st/Piston_1} 
	\end{columns}
\end{frame}

\begin{frame}{\insertsubsection}
	\begin{columns}[onlytextwidth]
		\column{0.6\linewidth}
			Orifice:
			\begin{align*}
				D_{orifice} &= 0.447[cm] \\
				S_{orifice} &= \tfrac{\pi*0.447^2}{4}\\
				  &= 0.157[cm^2] \\
				F_{orifice} &= 200*0.157\\
				  &= 31.4 [\sfrac{kg}{cm^2}]
			\end{align*}
		\column{0.4\linewidth}
			\mypict{../pdf/1st/Piston_1} 
	\end{columns}
\end{frame}

\begin{frame}{\insertsubsection}
	\begin{columns}[onlytextwidth]
		\column{0.6\linewidth}
			Piston:
			\begin{align*}
				D_{piston} &= 2[cm] \\
				S_{piston} &= \tfrac{20*{D_{p}}^2}{4}\\
				  &= 3.14[cm^2] \\
				F_{piston} &= F_{orifice}\\
				P_{piston} &= \tfrac{F_p}{S_p}\\
				  &= \tfrac{31.4}{3.14}\\
				  &= 10 [bar]
			\end{align*}
		\column{0.4\linewidth}
			\mypict{../pdf/1st/Piston_1} 
	\end{columns}
\end{frame}

\begin{frame}{\insertsubsection}
	\begin{columns}[onlytextwidth]
		\column{0.6\linewidth}
			Problem:
			\begin{itemize}
				\item When the pressure in the tank decrease, the intermediate pressure will drop.
				\item This system in a pressure divider (ratio 1:20) and not a regulator.
			\end{itemize}
		\column{0.4\linewidth}
			\mypict{../pdf/1st/Piston_1} 
	\end{columns}
\end{frame}

\begin{frame}{\insertsubsection}
	\begin{columns}[onlytextwidth]
		\column{0.6\linewidth}
			To compensate the pressure drop in the tank, a spring is add to the system\\
			$$F_{piston}=F_{orifice}+F_{spring}$$
		\column{0.4\linewidth}
			\mypict{../pdf/1st/Piston_2} 
	\end{columns}
\end{frame}

\begin{frame}{\insertsubsection}
	\begin{columns}[onlytextwidth]
		\column{0.6\linewidth}
			The intermediate pressure can now be adjusted by changing the caracteristics of the spring. 
		\column{0.4\linewidth}
			\mypict{../pdf/1st/Piston_2} 
	\end{columns}
\end{frame}

\begin{frame}{\insertsubsection}
	\begin{columns}[onlytextwidth]
		\column{0.6\linewidth}
			Finally, the piston is drilled to allow the air to flow through to the other side.
		\column{0.4\linewidth}
			\mypict{../pdf/1st/Piston_3} 
	\end{columns}
\end{frame}

\begin{frame}{\insertsubsection}
			\mypict{../pdf/1st/Bal_vs_Unbal}
\end{frame}

%---------------------------------------------------------------------
%										1st STAGES
%---------------------------------------------------------------------
\subsection{First stage}

\begin{frame}{Design types}
	\begin{itemize}
		\item Unbalanced piston
		\item Balanced Piston
		\item Unbalanced diaphragm
		\item Balanced diaphragm
	\end{itemize}
\end{frame}

% Unbalanced Piston
\subsubsection{Unbalanced piston}
\begin{frame}{\insertsubsubsection}
	\mypict{../pdf/1st/1st_piston_unbal}
\end{frame}

\begin{frame}{\insertsubsubsection}
	\mypict{../img/Scubapro_MK2}
\end{frame}

\begin{frame}{\insertsubsubsection}
	\begin{itemize}
		\item Pros
		\begin{itemize}
			\item Cheap
			\item Easy to service
			\item Strong and durable
			\item Perfect for dive school
		\end{itemize}
		\item Cons
		\begin{itemize}
			\item Limited performances
		\end{itemize}
	\end{itemize}
\end{frame}

\begin{frame}{\insertsubsubsection}
	\begin{itemize}
		\item Models
		\begin{itemize}
			\item Aqualung Calypso
			\item Scubapro MK2
			\item US Divers Conshelf
		\end{itemize}
	\end{itemize}
\end{frame}

% Balanced Piston
\subsubsection{Balanced piston}
\begin{frame}{\insertsubsection}
	\mypict{../pdf/1st/1st_piston_bal}
\end{frame}

\begin{frame}{\insertsubsubsection}
	\mypict{../img/Scubapro_MK25}
\end{frame}

\begin{frame}{\insertsubsubsection}
	\mypict{../img/Scubapro_MK25_exploded}
\end{frame}

\begin{frame}{\insertsubsubsection}
	\begin{itemize}
		\item Pros
		\begin{itemize}
			\item Safe and reliable
			\item Excellent performance
			\item Strong
		\end{itemize}
		\item Cons
		\begin{itemize}
			\item Expensive
			\item Prone to freeze ("flow-through" piston)
			\item Complicated to service
		\end{itemize}
	\end{itemize}
\end{frame}

\begin{frame}{\insertsubsubsection}
	\begin{itemize}
		\item Models
		\begin{itemize}
			\item Halcyon H-75P
			\item Scubapro MK20, MK25
		\end{itemize}
	\end{itemize}
\end{frame}

% Unbalanced Diaphragm
\subsubsection{Unbalanced diaphragm}
\begin{frame}{\insertsubsubsection}
	\mypict{../pdf/1st/1st_dia_unbal}
\end{frame}

\begin{frame}{\insertsubsubsection}
	\begin{itemize}
		\item Not common
		\item Pros
		\begin{itemize}
			\item Increase in intermediate pressure when cylinder pressure drops
		\end{itemize}
		\item Cons
		\begin{itemize}
			\item Needs a second stage which can handle a higher intermediate pressure
			\item Harder to breathe at the begining of the dive
		\end{itemize}
	\end{itemize}
\end{frame}

\begin{frame}{\insertsubsubsection}
	\begin{itemize}
		\item Models
		\begin{itemize}
			\item None availaible on the market
		\end{itemize}
	\end{itemize}
\end{frame}

% Balanced Diaphragm
\subsubsection{Balanced diaphragm}
\begin{frame}{\insertsubsubsection}
	\mypict{../pdf/1st/1st_dia_bal}
\end{frame}

\begin{frame}{\insertsubsubsection}
	\mypict{../img/Scubapro_MK17}
\end{frame}

\begin{frame}{\insertsubsubsection}
	\mypict{../img/Scubapro_MK17_exploded}
\end{frame}

\begin{frame}{\insertsubsubsection}
	\begin{itemize}
		\item Most common regulator
		\item Pros
		\begin{itemize}
			\item Simple but effective
			\item Easily set up for cold water
		\end{itemize}
		\item Cons
		\begin{itemize}
			\item Doesn't like being flooded
		\end{itemize}
	\end{itemize}
\end{frame}

\begin{frame}{\insertsubsubsection}
	\begin{itemize}
		\item Models
		\begin{itemize}
			\item Apeks (All)
			\item Aqualung Legend, Glacia,...
			\item Halcyon H-50D
			\item Mares Abyss, Prestige,...
			\item Scubapro MK11, MK17
			\item Poseidon Jetstream
		\end{itemize}
	\end{itemize}
\end{frame}

\begin{frame}{\insertsubsubsection}
	\begin{columns}[onlytextwidth]
		\column{0.6\linewidth}
			\begin{itemize}
				\item Poseidon Xstream
				\begin{itemize}
					\item The seat is replaced by a ball closing the orifice
					\item The needle act on the ball the same way as on the seat
					\item A spring hold the ball in place
				\end{itemize}
			\end{itemize}
		\column{0.4\linewidth}
			\mypict{../img/Poseidon_Xstream}
	\end{columns}
\end{frame}

% Options
\subsubsection{First stage options}
\begin{frame}{\insertsubsubsection}
	\begin{itemize}
		\item Anti-freeze system
		\begin{itemize}
			\item Ambiant pressure filled with grease
			\item Cap filled with grease
			\item Diaphragm filled with oil
			\item Diaphragm with piston or "load transmitter" \\ (Most common)
			\item Dry ambiant pressure by "Dry bleed" (Controled leak of the first stage)
			\item "Condom and vodka"!
		\end{itemize}
	\end{itemize}
\end{frame}

%---------------------------------------------------------------------
%										2nd STAGES
%---------------------------------------------------------------------
\subsection{Second Stages}

% Differents types
\begin{frame}{Design types}
	\begin{itemize}
		\item Upstream valve
		\item Unbalanced downstream
		\item Balanced downstream
		\item Servo or pilot valve
	\end{itemize}
\end{frame}

% Upstream
\subsubsection{Upstream valve}
\begin{frame}{\insertsubsubsection}
	\mypict{../pdf/2nd/2nd_upstream}
\end{frame}

\begin{frame}{\insertsubsubsection}
	\begin{itemize}
		\item Pros
		\begin{itemize}
			\item Simple
		\end{itemize}
		\item Cons
		\begin{itemize}
			\item Upstream design\\ Need a pressure relief valve (PRV)
			\item Limited performances
		\end{itemize}
	\end{itemize}
\end{frame}

\begin{frame}{\insertsubsubsection}
	\begin{itemize}
		\item Models
		\begin{itemize}
			\item None availaible on the market
		\end{itemize}
	\end{itemize}
\end{frame}


% Downstream unbalanced
\subsubsection{Unbalanced downstream}
\begin{frame}{\insertsubsubsection}
	\mypict{../pdf/2nd/2nd_DS_unbal}
\end{frame}

\begin{frame}{\insertsubsubsection}
	\begin{itemize}
		\item Pros
		\begin{itemize}
			\item Cheap
			\item Easy to service
			\item Perfect for dive school
			\item Downstream design \\ (Act as a pressure relief valve)
		\end{itemize}
		\item Cons
		\begin{itemize}
			\item Limited performances
		\end{itemize}
	\end{itemize}
\end{frame}

\begin{frame}{\insertsubsubsection}
	\begin{itemize}
		\item Models
		\begin{itemize}
			\item Aqualung Calypso
			\item Mares Prestige
			\item Scubapro R095, R195,...
			\item US Divers Conshelf
		\end{itemize}
	\end{itemize}
\end{frame}


% Downstream balanced
\subsubsection{Balanced downstream}
\begin{frame}{\insertsubsubsection}
	\mypict{../pdf/2nd/2nd_DS_bal}
\end{frame}

\begin{frame}{\insertsubsubsection}
	\mypict{../img/Scubapro_S600_exploded}
\end{frame}

\begin{frame}{\insertsubsubsection}
	\begin{itemize}
		\item Pros
		\begin{itemize}
			\item Safe and efficient
			\item Good performances
			\item Easily adjustable
		\end{itemize}
		\item Cons
		\begin{itemize}
			\item Expensive
			\item Complicated to service
		\end{itemize}
	\end{itemize}
\end{frame}

\begin{frame}{\insertsubsubsection}
	\begin{itemize}
		\item Models
		\begin{itemize}
			\item Apeks (all)
			\item Aqualung Legend, Glacia
			\item Scubapro G250, S600, A700,...
		\end{itemize}
	\end{itemize}
\end{frame}

% Servo
\subsubsection{Servo or pilot valve}
\begin{frame}{\insertsubsubsection}
	\mypict{../img/Poseidon}
\end{frame}

\begin{frame}{\insertsubsubsection}
	\begin{itemize}
		\item Pros
		\begin{itemize}
			\item Efficient
			\item Good performances
			\item Don't freeze\\ (no metal part are moving)
		\end{itemize}
		\item Cons
		\begin{itemize}
			\item Complicated
			\item Upstream design \\ Need a pressure relief valve
			\item high inhalation resistance
		\end{itemize}
	\end{itemize}
\end{frame}

\begin{frame}{\insertsubsubsection}
	\begin{itemize}
		\item Models
		\begin{itemize}
			\item Poseidon Jetstream and Xstream
		\end{itemize}
	\end{itemize}
\end{frame}

% Options
\subsubsection{Second stage options}
\begin{frame}{\insertsubsubsection}
	\begin{itemize}
		\item Venturi effect
		\begin{itemize}
			\item Reduce respiratory effort
			\item Tendency to induce a freeflow
			\item Deflector inside the case modifiying the air flow
		\end{itemize}
	\end{itemize}
\end{frame}

%---------------------------------------------------------------------
%										INFLATORS
%---------------------------------------------------------------------
\section{Inflators}

% Hose
\subsection{Shrader valve}
\begin{frame}{\insertsubsection}
	\begin{columns}[onlytextwidth]
		\column{0.3\linewidth}
			\mypict{../img/Schrader_valve}
		\column{0.7\linewidth}
			The inflator hose are using a Schrader valve.\\
			A small upstream valve that open when it is pushed.\\
			When the hose is disconnected, the pressure close the valve and avoid any leak.\\
			Same valve used for the bike tires!
	\end{columns}
\end{frame}

% LDrysuit
\subsection{Dry suit}
\begin{frame}{\insertsubsection}
	\mypict{../img/Drysuit-valve}
\end{frame}

\begin{frame}{\insertsubsection}
	\mypict{../pdf/misc/DrySuit-Valve}
\end{frame}

%---------------------------------------------------------------------
%										VALVES
%---------------------------------------------------------------------
\section{Valves}

%----- CONCEPTION -----
\subsection{Design}

% K-Valve
\begin{frame}{K-Valve with burst disc}
	\mypict{../img/K-Valve}
\end{frame}

\begin{frame}{K-Valve with burst disc}
	\mypict{../pdf/misc/Valve}
\end{frame}

\begin{frame}{K-Valve with burst disc}
	\mypict{../img/Valve_exploded}
\end{frame}

% Isolateur
\begin{frame}{Manifold}
	\mypict{../img/Manifold}
\end{frame}

\begin{frame}{Manifold}
	\mypict{../pdf/misc/Valve_Manifold}
\end{frame}

% J-Valve
\begin{frame}{J-Valve}
	\mypict{../img/J-Valve}
\end{frame}

\begin{frame}{J-Valve}
	\mypict{../pdf/misc/J-Valve}
\end{frame}

% H-Valve
\begin{frame}{H-Valve}
	\mypict{../img/H-Valve}
\end{frame}

% Y-Valve
\begin{frame}{Y-Valve}
	\mypict{../img/Y-Valve}
\end{frame}

% J-Valve
\begin{frame}{J-Valve with manifold}
	\mypict{../img/Manifold-old}
\end{frame}

%----- CONNEXION -----
\subsection{Output}

\begin{frame}{A-clamp}
	\begin{columns}[onlytextwidth]
		\column{0.4\linewidth}
			\mypict{../img/k_outlet}
		\column{0.6\linewidth}
			\begin{itemize}
				\item Yoke, Int, A-Clamp
				\item O-ring pinched between the valve and the reg
				\item Maximum pressure of 232bar
			\end{itemize}
	\end{columns}
\end{frame}

\begin{frame}{DIN}
	\begin{columns}[onlytextwidth]
		\column{0.4\linewidth}
			\mypict{../img/din_outlet}
		\column{0.6\linewidth}
			\begin{itemize}
				\item DIN $G\sfrac{5}{8}"$ ($22.9mm$)
				\item O-ring compressed inside the valve
				\item Screwed connection, so extremely strong
				\item 5 threads for 232 bar
				\item 7 threads for 300 bar
			\end{itemize}	 
	\end{columns}
\end{frame}

\begin{frame}{Insert}
	\begin{columns}[onlytextwidth]
		\column{0.4\linewidth}
			\mypict{../img/pro_outlet}
		\column{0.6\linewidth}
			\begin{itemize}
				\item Convert a DIN to Yoke
				\item Insert screwed inside a DIN
				\item Numerous types of valves and insert\\Compatibility not guarranted
			\end{itemize}
	\end{columns}
\end{frame}

\begin{frame}{DIN M26}
	\begin{columns}[onlytextwidth]
		\column{0.4\linewidth}
			\mypict{../img/din_m26}
		\column{0.6\linewidth}
			\begin{itemize}
				\item DIN valve with a bigger diameter ($26mm$)
				\item Used for oxygen or nitrox tanks
			\end{itemize} 
	\end{columns}
\end{frame}

\begin{frame}{CGA540}
	\begin{itemize}
		\item Valve used for pure oxygen tanks
		\item Connectic metal to metal (brass)
		\item No o-ring
		\item Wrench necessary to tighten correctly and avoid any leak
	\end{itemize}
	\begin{columns}
		\column{0.4\linewidth}
			\mypict{../img/CGA540_fem}
		\column{0.4\linewidth}
			\mypict{../img/CGA540_mal}
	\end{columns}
\end{frame}

%----- Threads -----
\subsection{Tank threads}

\begin{frame}{Tank threads}
	Europe:
	\begin{itemize}
		\item $M25*2mm$ DIN (Deutsche Industrial Norm)\\
			{\o}$=25mm$, $2mm$ between each threads\\
			\hfill
		\item $M18*1.5mm$ DIN (Deutsche Industrial Norm)\\
			{\o}$=18mm$, $1.5mm$ between each threads\\
			\hfill
		\item $G \sfrac{3}{4}*14$ BSP (British Standard Pipe)\\
			{\o}$=26.4mm$, 14 threads per inch\\
			\hfill
	\end{itemize}
	USA (and worldwide):
	\begin{itemize}
		\item $\sfrac{3}{4}"*14$ NPSM\\
			(National Pipe Straight Mecanical)\\
			{\o}$=26.4mm$, 14 threads per inch\\
			\hfill
	\end{itemize}
\end{frame}

\begin{frame}{Tank threads}
	\begin{block}{ATTENTION}
		\begin{itemize}
			\item British $G \sfrac{3}{4}"$ and American $\sfrac{3}{4}"$ NPSM are really close but not 	compatible
			\item	The shape of the threads are not the same
		\end{itemize}
	\end{block}
\end{frame}

%---------------------------------------------------------------------
%										CYLINDERS
%---------------------------------------------------------------------
\section{Cylinders}

%----- MATERIAL -----
\subsection{Material}

% ALU
\begin{frame}{Aluminium alloy}
	\begin{itemize}
		\item Pros:
			\begin{itemize}
				\item Don't rust!
				\item Don't need to be painted
			\end{itemize}
		\item Cons
			\begin{itemize}
				\item Positively buoyant when empty
				\item Galvanic corrosion with brass (valve) and steel (cam bands)
				\item Maximum temperature: $150^\circ C$
			\end{itemize}
	\end{itemize}
\end{frame}

% acier
\begin{frame}{Steel}
	\begin{itemize}
		\item Pros:
			\begin{itemize}
				\item Strong
				\item Negatively buoyant when empty\\(less weight needed)
				\item Higher working pressure($232bar$)
				\item Lighter
				\item Maximum temperature: $300^\circ C$
			\end{itemize}
		\item Cons:
			\begin{itemize}
				\item Rust!!!
				\item Need to be painted
			\end{itemize}
	\end{itemize}
\end{frame}

% Carbone
\begin{frame}{Carbon fiber}
	\begin{itemize}
		\item Pros:
			\begin{itemize}
				\item High pressure ($300bar$)
				\item Light
			\end{itemize}
		\item Cons:
			\begin{itemize}
				\item Fragile!
				\item Distinctive visual and hydrostatic inspections 
			\end{itemize}
	\end{itemize}
\end{frame}

%----- COMPARISON -----
\subsection{Comparison}
\begin{frame}{\insertsubsection}
	\footnotesize
	\begin{multicols}{2}
		\textbf{Aluminium}\\
		Worthington AL80\\
		\begin{tabularx}{\linewidth}{X r}
			Pressure:		& 	$207 bar$	\\
			Volume:			&	$11.1L$		\\
			Empty weight:	&	$14.5kg$	\\
			Buoyancy empty:	&	$+1.45kg$	\\
			Buoyancy at $200bar$:	&	$-0.81kg$
		\end{tabularx}

		\textbf{Steel}\\
		Worthington 80\\
		\begin{tabularx}{\linewidth}{X r}
			Pressure:		& 	$237 bar$	\\
			Volume:			&	$11.1L$		\\
			Empty weight:	&	$12.7kg$	\\
			Buoyancy empty:	&	$-1.36kg$	\\
			Buoyancy at $200bar$:	&	$-4.08kg$
		\end{tabularx}
	\end{multicols}

	\vspace{1em}

	\begin{multicols}{2}
		\textbf{Aluminium}\\
		Catalina S80\\
		\begin{tabularx}{\linewidth}{X r}
			Pressure:		& 	$207 bar$	\\
			Volume:			&	$11.1L$		\\
			Empty weight:	&	$14.2kg$	\\
			Buoyancy empty:	&	$+1.85kg$	\\
			Buoyancy at $200bar$:	&	$-0.72kg$
		\end{tabularx}

		\textbf{Aluminium}\\
		Luxfer AL80\\
		\begin{tabularx}{\linewidth}{X r}
			Pressure:		& 	$207 bar$	\\
			Volume:			&	$11.1L$		\\
			Empty weight:	&	$14.2kg$	\\
			Buoyancy empty:	&	$+1.54kg$	\\
			Buoyancy at $200bar$:	&	$-0.63kg$
		\end{tabularx}
	\end{multicols}
\end{frame}

\end{document}