\documentclass[english,1pt,a4paper]{article}

\usepackage{../../shared/tex/course_tdi}

\title{TDI Intro to cave}
\subtitle{Révision de connaissances}

\author{TECHNICAL DIVING INTERNATIONAL}
\website{www.tdisdi.com}


%---------------------------------------------------------------------
%										FORMAT
%---------------------------------------------------------------------
% Change le format des listes
\renewenvironment{benumerate}%
{\normalfont\begin{list}{}{\samepage}}%
{\end{list}}

% Ajoute le nom à la fin d'une section
\newcommand{\sectionpage}[0]{%
	\vfill
	\rule{\textwidth}{1pt}\\%
	J'accepte blblbalbla
	\vspace{1cm}\\%
	\textbf{	Nom de l'élève: \hrulefill \vspace{1cm} \\ Date: \rule{2cm}{0.4pt} Signature: \hrulefill}%
	\pagebreak%
	}
%---------------------------------------------------------------------
%										DOCUMENT
%---------------------------------------------------------------------
\begin{document}

	\begin{titlepage}
	\begin{center}
		\includegraphics[width=13cm]{\tdilogo}\\
		\vspace{1cm}
		{\fontsize{40}{48}\selectfont \textbf{\thetitle}}\\
		\vspace{1cm}
		{\fontsize{30}{36}\selectfont \textbf{\thesubtitle}}\\
		\vspace{4cm}
		{\fontsize{18}{22}\selectfont \textbf{\theauthor}}\\
		\vspace{0.2cm}
		{\fontsize{18}{22}\selectfont \textbf{\thewebsite}}\\
	\end{center}
\end{titlepage}

%---------------------------------------------------------------------
%									Intro
%---------------------------------------------------------------------
	\section{Introduction à la plongée sous terraine}

	\begin{outline}
		\1 Quelle est la différence entre une caverne et une "cave"?	\vspace{2cm}
		\1 Donnez l’exemple d’une mauvaise motivation pour la plongée sous-terraine.	\vspace{2cm}
		\1 Listez 4 raisons pour lesquelles les plongeurs choisissent la plongée sous-terraine.	\vspace{2cm}
		\1 Qui furent parmi les premiers à faire de la plongée sous-terraine?	\vspace{2cm}
		\1 Listez 2 améliorations d’ordre technologique permettant les plongeurs d’explorer plus profondément les grottes sous-terraines.	\vspace{2cm}
	\end{outline}
	\sectionpage	

%---------------------------------------------------------------------
%									Types
%---------------------------------------------------------------------
	\section{Types de grottes}

	\begin{outline}
		\1 Comment se forment les "solution cave"?	\vspace{2cm}
		\1 Qu'elle est la forme de grotte sous marine la plus étendue?	\vspace{2cm}
		\1 Listez 2 exemples de formation géologique.	\vspace{2cm}
		\1 Qu’est-ce qu’une halocline?	\vspace{2cm}
		\1 Si vous trouvez une chambre sèche lors d’une exploration d’une grotte immergée, il est possible de respirer sans danger l’air de cette chambre.	\vspace{2cm}
		\1 Qu’est-ce qu’un résurgence?	\vspace{2cm}
		\1 Listez 2 dangers lors de plongée en siphon.	\vspace{2cm}
		\1 Définissez ce que veut dire "upstream" et "downstream" lors de plongée en sinkhole.	\vspace{2cm}
		\1 Quelle est la différence entre "sump diving" et "cave diving"?	\vspace{2cm}
		\1 Listez 2 dangers pouvant exister lors de plongée dans des mines submergées.	\vspace{2cm}
		\1 Pourquoi la plongée sous glace est-elle dangereuse?	\vspace{2cm}
		\1 Nommez un risque commun associé aux "blue holes"	\vspace{2cm}
		\1 Qu’est qu’une "sea cave"?	\vspace{2cm}
		\1 Quel sont le type de grottes se formant le plus vite?	\vspace{2cm}
		\1 Comment un "swim-through" diffère d’une "coral cave"?	\vspace{2cm}
	\end{outline}
	\pagebreak	

%---------------------------------------------------------------------
%									Accès
%---------------------------------------------------------------------
	\section{Accès aux grottes}

	\begin{outline}
		\1 Listez 3 façons permettant de préserver l’accès aux sites public.	\vspace{2cm}
		\1 Quels types de sites privés permettent de créer un fort lien entre les plongeurs et les proriétaires du terrain.	\vspace{2cm}
		\1 Qu’est qu’un site de type "express permission"?	\vspace{2cm}
		\1 Pourquoi certains propriétaires de terrain ferment-ils les yeux sur les activités de plongées sur leur site sans avoir donné de "express permission"?	\vspace{2cm}
		\1 Pourquoi est-il important de protéger l’accès à nos sites de plongée sous-terrain?	\vspace{2cm}
	\end{outline}
	\pagebreak	

%---------------------------------------------------------------------
%									Conservation
%---------------------------------------------------------------------
	\section{Conservation}

	\begin{outline}
		\1 Listez 2 types de formations géologiques présentes dans les grottes?	\vspace{2cm}
		\1 Pourquoi la vie animale en vase clos est elle particulièrement sensible?	\vspace{2cm}
		\1 Quels types de souvenirs peut être ramenés lors des plongées sous-terraine?	\vspace{2cm}
		\1 Comment la planification aide-t-elle la conservation des grottes?	\vspace{2cm}
		\1 Quel est le meilleur moyen de protéger les grottes pour les générations futures?	\vspace{2cm}
	\end{outline}
	\pagebreak	

%---------------------------------------------------------------------
%									Accident
%---------------------------------------------------------------------
	\section{Analyse d'accident}

	\begin{outline}
		\1 Pourquoi l’analyse d’accident est-elle importante?	\vspace{2cm}
		\1 Listez 5 règles d’analyse d’accident :	\vspace{2cm}
		\1 Statistiquement parlant, quelle est la raison principale causant la mort à des plongeurs spéléo entrainés.	\vspace{2cm}
		\1 Quelle est le règle d’or en plongée sous-terraine?	\vspace{2cm}
		\1 Est-ce que le fait de suivre ces règles garanti une plongée sans risque?	\vspace{2cm}
	\end{outline}
	\pagebreak	

%---------------------------------------------------------------------
%									Techniques
%---------------------------------------------------------------------
	\section{Techniques de plongée spéléo}

	\begin{outline}
		\1 Comment maitrise-t-on l’art de la flottabilité?	\vspace{2cm}
		\1 Quel est le premier facteur influençant votre flottabilité?	\vspace{2cm}
		\1 Lors de l’utilisation d’une combinaison étanche, quelle quantité d’air doit on ajouter dedans?	\vspace{2cm}
		\1 Pourquoi la flottabilité est-elle important pour les plongeurs spéléo?	\vspace{2cm}
		\1 Définissez "l’assiette" (trim) :	\vspace{2cm}
		\1 Est-ce qu’une "assiette" en position horizontale doit être maintenu à chaque instant lors de la plongée sous-terraine?	\vspace{2cm}
		\1 Qu’est-ce que votre "ligne d’assiette" (trim line)?	\vspace{2cm}
		\1 Listez 6 types de propulsion utilisé en plongée sous terraine :	\vspace{2cm}
		\1 Quelle est la différence entre un "frog kick" modifié et un "frog kick" normal?	\vspace{2cm}
		\1 Quelle est la partie offrant le moins de résistance au courant lorsqu’un passage d’une grotte tourne brusquement sur la droite?	\vspace{2cm}
		\1 Quel est "l’angle mort" d’un plongeur?	\vspace{2cm}
		\1 A quel moment un plongeur doit-il maintenir contrôle lors d’une plongée?	\vspace{2cm}
		\1 Si le contrôle est perdu lors de la plongée, quelles actions doivent être entreprises?	\vspace{2cm}
		\1 Quelle est la seule vraie urgence en plongée sous terraine?	\vspace{2cm}
		\1 Quel est le meilleur moyen d’éviter une situation de stress?	\vspace{2cm}
	\end{outline}
	\pagebreak	

%---------------------------------------------------------------------
%									Equipe
%---------------------------------------------------------------------
	\section{Conscience et dynamique d'équipe}

	\begin{outline}
		\1 Qu’est-ce que la conscience?	\vspace{2cm}
		\1 Listez 3 types de conscience :	\vspace{2cm}
		\1 Quelle est la différence entre ces 3 types?	\vspace{2cm}
		\1 Quelles sont les qualités d’un plongeur qui font de lui un membre important d’une équipe de plongée?	\vspace{2cm}
		\1 Pourquoi est-il important de choisir vos équipiers avec soin?	\vspace{2cm}
	\end{outline}
	\pagebreak	

%---------------------------------------------------------------------
%									Equipement
%---------------------------------------------------------------------
	\section{Équipement}

	\begin{outline}
		\1 Pourquoi est-il important de conserver un esprit ouvert lors de la configuration de votre équipement?	\vspace{2cm}
		\1 Listez 4 types de configuration	\vspace{2cm}
		\1 Quel est un avantage des bouteilles en acier?	\vspace{2cm}
		\1 Pourquoi les plongeurs sous-terrain préfèrent-ils les robinetteries de type DIN?	\vspace{2cm}
		\1 Est-ce qu’un détendeur 300 B peut-il être utilisé sur une robinetterie de 200 b?	\vspace{2cm}
		\1 Qu’est-ce qu’un "manifold"?	\vspace{2cm}
		\1 Listez 3 facteurs jouant un rôle important lors de la sélection d’un détendeur :	\vspace{2cm}
		\1 Est-ce qu’un plongeur doit utiliser 2 manomètres lorsqu’il plonge avec des bouteilles montées avec un manifold?	\vspace{2cm}
		\1 Quels sont 2 facteurs qui seront influencés par la forme de la BCD/wing?	\vspace{2cm}
		\1 Pourquoi une protection thermique adéquate est-elle importante pour le plongeur?	\vspace{2cm}
		\1 Quelle distraction amènera un masque défectueux?	\vspace{2cm}
		\1 Est-ce que tous les types de palmes permettent une propulsion efficace en plongée-sous terraine?	\vspace{2cm}
		\1 Quel est le nombre minimum d’outil tranchant qu’un plongeur spéléo devrait avoir?	\vspace{2cm}
		\1 Combien de lampes torches un plongeur doit il emporter?	\vspace{2cm}
		\1 En quoi les ardoises et/ou cahiers immergeables sont utiles?	\vspace{2cm}
		\1 Combien de profondimètre/chronomètre sont requis pour une plongée en sécurité?	\vspace{2cm}
		\1 Listez 3 types d’équipement de type "hardware" que les plongeurs peuvent utiliser?	\vspace{2cm}
		\1 Pourquoi certains plongeurs portent-ils des casques?	\vspace{2cm}
		\1 Listez 2 façons de rendre votre équipement hydrodynamique :	\vspace{2cm}
		\1 Est-ce que le fait d’avoir le bon équipement est suffisant pour une plongée en sécurité?	\vspace{2cm}
	\end{outline}
	\pagebreak	

%---------------------------------------------------------------------
%									Communication
%---------------------------------------------------------------------
	\section{Communication}

	\begin{outline}
		\1 Listez 3 façons utilisées pour communiquer en plongée.	\vspace{2cm}
		\1 Quelle est la façon la plus efficace de communiquer dans le noir?	\vspace{2cm}
		\1 Décrivez 3 types de signal avec une lampe torche.	\vspace{2cm}
		\1 Définissez la "conscience" de la lumière :	\vspace{2cm}
		\1 Quels sont les 3 types de signal, fait avec la main, nécessitant une réponse?	\vspace{2cm}
		\1 A quel moment le "touch contact" est-il nécessaire?	\vspace{2cm}
		\1 Définissez la différence entre le "open OK" et le "closed OK".	\vspace{2cm}
		\1 Qui est a l’initiative du mouvement "en avant" lors de la position "touch contact"?	\vspace{2cm}
		\1 Qui est responsable des décisions de direction durant le "touch contact"?	\vspace{2cm}
		\1 Comment la situation de panne de gaz doit être signalée lors du "touch contact"?	\vspace{2cm}
	\end{outline}
	\pagebreak	

%---------------------------------------------------------------------
%									Lignes
%---------------------------------------------------------------------
	\section{Ligne ou fil d'ariane}

	\begin{outline}
		\1 Pourquoi est-ce que les lignes permanentes sont positionnées au delà de la zone de lumière naturelle?	\vspace{2cm}
		\1 Listez 3 types de lignes permanentes :	\vspace{2cm}
		\1 Quelle est la fonction d’un dévidoir principal?	\vspace{2cm}
		\1 Combien de dévidoir de secours un plongeur doit-il transporter?	\vspace{2cm}
		\1 Comment les dévidoirs "jump" ou "gap" sont ils utilisés?	\vspace{2cm}
		\1 Quel est le type de dévidoir le plus large?	\vspace{2cm}
		\1 Décrivez la différence entre un dévidoir standard et dévidoir compact.	\vspace{2cm}
		\1 Pourquoi les dévidoirs de type "finger spool" sont ils utilises par certains plongeurs?	\vspace{2cm}
		\1 Quels facteurs doivent être considérés lors du choix du diamètre de la ligne temporaire?	\vspace{2cm}
		\1 Listez 3 fonctions d’une flèche permanente :	\vspace{2cm}
		\1 Si un marquer individuel appartenant a une autre équipe est en place, peut-il être utilise comme élément marquant pour la navigation?	\vspace{2cm}
		\1 Qu’est-ce qu’un cookie?	\vspace{2cm}
		\1 Listez 4 règles s’appliquant au déploiement des lignes.	\vspace{2cm}
		\1 Listez 4 règles s’appliquant à la récupération des lignes.	\vspace{2cm}
		\1 Expliquer ce qu’est un "line trap"?	\vspace{2cm}
		\1 Où doit se trouver le nœud primaire?	\vspace{2cm}
		\1 Quelle est la fonction du nœud secondaire?	\vspace{2cm}
		\1 Expliquer la différence entre un nœud secondaire et un nœud standard?	\vspace{2cm}
		\1 A quel moment un "placement" doit-il être fait?	\vspace{2cm}
		\1 Expliquer la procédure pour sécuriser une ligne temporaire à une ligne permanente.	\vspace{2cm}
		\1 À quel moment est-il nécessaire de sécuriser un ligne temporaire sur flèche posée sur une ligne permanente?	\vspace{2cm}
		\1 Pourquoi est-ce qu’un "midpoint" requière la validation de l’équipe?	\vspace{2cm}
		\1 Définissez "conscience" de la ligne.	\vspace{2cm}
		\1 Listez 3 des règles générales de bonne conduite en plongée sous terraine.	\vspace{2cm}
		\1 En quoi l’utilisation de ligne temporaire et permanente, combiné avec une bonne "conscience" de ligne, et pénétration progressive, aboutit-elle?	\vspace{2cm}
	\end{outline}
	\pagebreak	

%---------------------------------------------------------------------
%									Navigation
%---------------------------------------------------------------------
	\section{Navigation compexe (Full Cave)}

	\begin{outline}
		\1 Quels types de marqueurs sont fréquemment utilisés pour marquer un "jump"?	\vspace{2cm}
		\1 Où trouve-t-on les "gap" et quel est leur rôle?	\vspace{2cm}
		\1 Qu’est-ce qu’un "T"?	\vspace{2cm}
		\1 Expliquer la procédure à suivre pour marquer un "T".	\vspace{2cm}
		\1 Expliquer la différence entre une traversée simple et complexe.	\vspace{2cm}
		\1 A quoi le terme "complétion pressure" fait référence à?	\vspace{2cm}
		\1 Est-ce qu’une traversée complexe peut-elle être faite si l’un des équipiers atteint sa pression de demi tour avant d’avoir atteint le premier point de retour?	\vspace{2cm}
		\1 Qu’est-ce qu’un circuit?	\vspace{2cm}
		\1 Durant la portion de sortie de la première plongée lors d’un circuit complexe, 70 bar sont consommés par le plongeur consommant le plus. Assumant que tous les équipiers utilisent les mêmes bouteilles, quelle est la "completion pressure"  pour la seconde plongée?	\vspace{2cm}
		\1 A quel moment une équipe peut elle faire des plongées nécessitant des techniques de navigation complexe?	\vspace{2cm}
	\end{outline}
	\pagebreak	

%---------------------------------------------------------------------
%									Problèmes
%---------------------------------------------------------------------
	\section{Procédures de résolutions de problèmes}

	\begin{outline}
		\1 Comment un plongeur peut il éliminer le besoin d’avoir recourt a des procédure de résolutions de problèmes?	\vspace{2cm}
		\1 Quels sont les 2 types de stress?	\vspace{2cm}
		\1 Listez 5 signes et symptômes de stress chez un plongeur :	\vspace{2cm}
		\1 Quel est le meilleur moyen de gérer le stress?	\vspace{2cm}
		\1 Listez 5 étapes permettant d’éviter une pane de lampe torche.	\vspace{2cm}
		\1 Quelle action doit être entreprise lors d’une panne de lampe torche principale?	\vspace{2cm}
		\1 Si une perte de visibilité survient du a des particules en suspension dans l’eau, quelle procédure doit être suivit?	\vspace{2cm}
		\1 Pourquoi le port d’un masque de secours est recommandé durant toute plongée technique?	\vspace{2cm}
		\1 Comment l’emmêlement de ligne peut-il être évité?	\vspace{2cm}
		\1 Comment un plongeur spéléo peut-il éviter de perdre la ligne?	\vspace{2cm}
		\1 Quelle est la première étape à suivre dans le cas d’une perte ou rupture de ligne?	\vspace{2cm}
		\1 Qu’est-ce qui amène la séparation et/ou perte des équipiers?	\vspace{2cm}
		\1 Est-ce que la recherche d’un plongeur disparu doit être entreprise dans le cas d’insuffisance de réserve?	\vspace{2cm}
		\1 Quel facteur dictera la procédure à utiliser pour isoler les gaz lors d’une fuite importante?	\vspace{2cm}
		\1 Quel type de comportement peut aboutir à la perte complète de gaz?	\vspace{2cm}
		\1 Dans une situation de partage de gaz suite à une panne de gaz dans une configuration de 2 plongeurs, quel est celui qui est en premier (dans le sens de la sortie)?	\vspace{2cm}
		\1 Décrivez la procédure pour porter assistance a un plongeur inconscient.	\vspace{2cm}
		\1 Pourquoi les plongeurs spéléo devrait incorporer un système de flottabilité redondant comme faisant partie de leur système de "survie"?	\vspace{2cm}
		\1 Quelles options sont possibles dans le cas de sorties bloquées par un rocher ou débris?	\vspace{2cm}
		\1 Qu’est-ce que DCI?	\vspace{2cm}
		\1 Quel est le meilleur moyen d’éviter une DCS?	\vspace{2cm}
		\1 Listez 5 signes et symptômes de DCS.	\vspace{2cm}
		\1 Qu’est-ce qui peut provoquer une EGA?	\vspace{2cm}
		\1 Listez 5 signes et symptômes d’une EGA.	\vspace{2cm}
		\1 Décrirez le traitement a donner a des plongeurs atteint d’une DCI	\vspace{2cm}
	\end{outline}
	\pagebreak	

%---------------------------------------------------------------------
%									Plannification
%---------------------------------------------------------------------
	\section{Plannification de plongée}

	\begin{outline}
		\1 Quel est le premier pas d’un processus de planification d’une plongée sous-terraine?	\vspace{2cm}
		\1 Est-ce que les objectifs requièrent l’accomplissement?	\vspace{2cm}
		\1 Qu’est-ce qui va dicter l’ordre dans lequel chaque phase de la planification d’ordre logistique sera exécutée?	\vspace{2cm}
		\1 Nommez 2 outils utilisables pour planifier les profils de plongée.	\vspace{2cm}
		\1 Est-ce que le brevet TDI "cave diver" requière l’utilisation de gaz autre que celui de l’air?	\vspace{2cm}
		\1 Un plongeur utilisant un set de 2 bouteilles en acier de 15 L consomme 20 Bar en 10 minutes à une profondeur de 20 m. Quel sera son "SAC rate"?	\vspace{2cm}
		\1 Une équipe planifie d’explorer une grotte. Les premiers 80 m sont a une profondeur max de 10 m. Au delà des premiers 80 m, la grotte descend jusqu'à un profondeur max de 20 m.  Les explorations passées montrent une progression moyenne de 20 m par minute. La plus grosse consommation est de 20 L par minute. L’équipe voudrait explorer les premiers 200 de la grotte. Combien de gaz sera consommé durant cette plongée? Prenez en compte 5 minutes pour la pose et récupération de la ligne temporaire et 3 minutes pour le palier de sécurité.	\vspace{2cm}
		\1 Une équipe planifie de plonger dans une résurgence présentant un courant conséquent. La planification de gaz consomme est de 2800 L. Quel est la réserve minimum que l’équipe doit emporter?	\vspace{2cm}
		\1 Pourquoi les plongeurs doivent ils prendre en compte les différences de volume des bouteilles?	\vspace{2cm}
		\1 Mark et Tom veulent plonger dans une résurgence présentant un fort courant. Mark utilise 2 x 11 L en aluminium gonflés à 207 bar. Tom utilise 2 x 15 L en acier gonflées à 200 bar. Quelle est la pression de demi tour?	\vspace{2cm}
		\1 Quelles décisions doivent être prises en ce quo concerne l’équipement lors de la phase de planification?	\vspace{2cm}
		\1 Est-ce qu’un groupe de 6 plongeur est un taille idéale pour explorer des petites grottes? Pourquoi et pourquoi pas?	\vspace{2cm}
		\1 Quels facteurs doivent être considérés lors de la sélection des membres d’une équipe?	\vspace{2cm}
		\1 Durant la plongée, qu’est-ce qui aidera à la progression de l’équipe, en terme de visualisation mentale de la topographie de la grotte?	\vspace{2cm}
		\1 Lors de la planification de plongée dans des lieux reculés, comment doit être faite cette planification?	\vspace{2cm}
	\end{outline}
	\pagebreak	

%---------------------------------------------------------------------
%									Préparation
%---------------------------------------------------------------------
	\section{Se préparer à plongée}

	\begin{outline}
		\1 Comment les plongeurs peuvent-ils faire pour réduire le risque important de laisser de l’équipement derrière soi?	\vspace{2cm}
		\1 Quel est le but d’une analyse du site?	\vspace{2cm}
		\1 Qu’est-ce qui doit être revu durant le briefing pré plonger?	\vspace{2cm}
		\1 Expliquer la procédure que vous devriez suivre lorsque vous vous équipez.	\vspace{2cm}
		\1 Qu’est-ce que le START représente?	\vspace{2cm}
	\end{outline}
	\pagebreak	

%---------------------------------------------------------------------
%									Debriefing
%---------------------------------------------------------------------
	\section{Debriefing}

	\begin{outline}
		\1 Quelle est l’importance d’un debriefing à la fin d’une plongée?	\vspace{2cm}
		\1 Pourquoi est-il important de faire le debriefing avant de sortir de l’eau?	\vspace{2cm}
		\1 Listez 5 facteurs qui doivent être discutés durant le debriefing?	\vspace{2cm}
	\end{outline}
\end{document} 

