\documentclass[english,10pt,twoside]{article}

\usepackage{../../shared/tex/course_tdi}

\title{TDI Procédures de décompression}
\subtitle{Examen Final: version A}

\author{TECHNICAL DIVING INTERNATIONAL}
\website{www.tdisdi.com}

%---------------------------------------------------------------------
%										DOCUMENT
%---------------------------------------------------------------------
\begin{document}
\sloppy
\begin{titlepage}
	\begin{center}
		\includegraphics[width=13cm]{\tdilogo}\\
		\vspace{1cm}
		{\fontsize{40}{48}\selectfont \textbf{\thetitle}}\\
		\vspace{1cm}
		{\fontsize{30}{36}\selectfont \textbf{\thesubtitle}}\\
		\vspace{4cm}
		{\fontsize{18}{22}\selectfont \textbf{\theauthor}}\\
		\vspace{0.2cm}
		{\fontsize{18}{22}\selectfont \textbf{\thewebsite}}\\
	\end{center}
\end{titlepage}

%---------------------------------------------------------------------
%										Examen
%---------------------------------------------------------------------
	\begin{outline}
		\1 Ne pas effectuer des paliers de décompression nécessaires pourrait engendrer:
			\2 Accident de décompression.
			\2 Narcose
			\2 Foramen Ovale Perméable
			\2 Aucune des réponse ci-dessus

		\1 L'attention \st d'un plongeur se situe au niveau de la sécurité et les autres tâches sont secondaires.
			\2 Secondaire
			\2 Non essentiel
			\2 Primaire
			\2 Suivant

		\1 La loi de \st décrit la relation entre le volume et la pression.
			\2 Boyle
			\2 Dalton
			\2 Buhlmann
			\2 Charles

		\1 La terme "Maladie des caissons" était utilisé pour décrire:
			\2 Les accidents de décompression
			\2 Les "bends"
			\2 Une embolie gazeuse
			\2 Les réponses A et B sont correctes
%5
		\1 Le terme M-Value décrit la niveau de tolérance de \st pour un compartiment tissulaire donné:
			\2 Sous-pression
			\2 Sur-pression
			\2 Sous-égalisation
			\2 Sur-égalisation

		\1 Pour des raisons pratiques, 1 atmosphère équivaut à 1 bar dans le système métrique, et 1 atmosphère équivaut à \st psi dans le système impérial.
			\2 29.4
			\2 44.1
			\2 14.7
			\2 58.8

		\1 À 20m/66ft, l'autonomie d'une bouteille de plongée sera \st de celle à la surface.
			\2 le quart
			\2 le tiers
			\2 la moité
			\2 le double

		\1 Quelle est la pression partielle d'oxygène en respirant de l'air à 35m/115ft?
			\2 0.8 ATA
			\2 0.9 ATA
			\2 1.0 ATA
			\2 1.1 ATA

		\1 En plus de la diffusion, la perfusion aussi contrôle la quantité de gaz qu'un tissu va \st dans un certain laps de temps.
			\2 Prendre
			\2 Utiliser
			\2 Faire disparaître
			\2 Absorber
%10
		\1 Durant la remontée, il est probable que les tissus rapides soient sursaturés et \st alors que les tissus lents seront sous-saturés et \st.
			\2 Diffusent, Perfusent
			\2 Perfusent, Diffusent
			\2 Dégazent, se chargent
			\2 Se chargent ,Dégazent

		\1 En faisant des paliers de décompression, le plongeur essaie de contrôler le niveau de sursaturation, ainsi:
			\2 Reduire le niveau d'azote dans le corps et les risques associés à la narcose.
			\2 Reduire le niveau d'oxygène dans les corps et les risques associés à la toxicité de l'oxygène.
			\2 Reduire le niveau de monoxide de carbone dans le corps et les risques associés à un empoisonnement au monoxide de carbone.
			\2 Minimiser la formation de bulles et les risques associés aux accidents de décompression.

		\1 Un certain nombre de bulles silencieuses est commun après une plongée. Un excès peu entraîner une accumulation dans les capillaires entourants \st et peuvent reduire les échanges gazeux
			\2 Les alvéoles
			\2 Le cerveau
			\2 Le coeur
			\2 Les nerfs occulaires

		\1 Confusion, troubles du comportement, perte de mémoire, tremblements incontrôlables et inflammations sont des signes et symptômes d'un accident de décompression.\vf

		\1 Un ADD de \st ne comprend que des douleurs et ne touche pas le système nerveux central, le système cardiovasculaire ou le pulmonaire.
			\2 Type VI
			\2 Type II
			\2 Type I
			\2 Type III
%15
		\1 Une hydratation correcte doit commencer au minimum	\st heures avant la plongée.
			\2 2
			\2 12
			\2 18
			\2 24

		\1 \st augmente les risques d'obstructions pulmonaires chroniques, de maladie coronarienne, d'accidents vasculaires cérébraux, d'anévrisme de l'aorte, de leucémie aiguë myéloïde, de cataracte, de pneumonie, de parodontite, de cancers de la vessie, de l'œsophage, du larynx, du poumon, de la trachée, de la gorge, du col utérin, du rein, de l'estomac et du pancréas, d'accident de décompression et d'embolie gazeuse artérielle.
			\2 Boire
			\2 Fumer
			\2 Le delatplane
			\2 Conduire

		\1 Environ \st de la population est atteint d'un Foramen Ovale Perméable (FOP).
			\2 10\%
			\2 15\%
			\2 20\%
			\2 25\%

		\1 Divers Alert Network (DAN) recommande un intervalle de surface minimum, avant un vol en avion, de \st heures pour une simple plongée dans la limite de non-décompression.
			\2 Plus de 18
			\2 18
			\2 12
			\2 24

		\1 Le meilleur moyen d'éviter la narcose est d'éviter les plongées profondes, ou de choisir un mélange gazeux adapté à la profondeur planifiée\vf	
%20
		\1 Une augmentation de la pression partielle de CO2 est considérée comme un facteur contribuant à l'apparition de:
			\2 Narcose, toxicité à l'oxygène, accident de décompression
			\2 Vertiges, étourdissements, nausée
			\2 FOP, maladie des montagnes, mal de mer
			\2 Aucune des réponses ci-dessus

		\1 Les tables de plongées basées sur le modèle de Haldane encouragent le plongeur a remonter aussi proche de la surface que possible, sans excéder la limite critique de sursaturation, et d'y rester jusqu'à ce que suffisamment de gaz inerte se soit échappé pour permettre au plongeur de remonter jusqu'au prochain palier de décompression.\vf


		\1 Les tables de Workman, les tables de l'US Navy, et l'algorithme ZHL-16 de Bühlmann, ont suivi les hypothèses générales de Haldane et sont classés comme:
			\2 "Haldanian modifiés"
			\2 "Neo-Haldanian."
			\2 Les réponse A et B sont correctes
			\2 Aucune des réponses ci-dessus

		\1 Les modèles "à double phase" ou "à bulles" tentent de prédire le comportement des gaz en solution et à l'état libre (bulles).\vf

		\1 Les modèles "à double phase" ou "à bulles" tentent de contrôler \st des bulles dans les tissus.
			\2 Le nombre
			\2 La quantité
			\2 Le type
			\2 La taille
%25
		\1 VPM est un modèle "à double phase" qui suggère que des bulles de différents(es) \st ont une perméabilité variable.
			\2 Gaz
			\2 Formes
			\2 Tailles
			\2 Textures

		\1 En plongée technique, RGBM est un acronyme de:
			\2 Reduced Gradient Bubble Model (Modèle à Bulles à Gradient Reduit)
			\2 Reduced Gas Bubble Marker (Marqueur de Bulles à Gaz Reduit)
			\2 Radiant Gaz Buoyancy Model (Modèle de Flottabilité à Gaz Rayonnant)
			\2 Real Gaz Buoyancy Model (Modèle de Flottabilité à Gaz Réel)

		\1 L'utilisation d'un recycleur impose au plongeur de transporter et conserver une réserve de gaz en \st suffisante qui peut être utilisée en cas de défaillance d'un recycleur.
			\2 Circuit fermé
			\2 Circuit semi-fermé
			\2 Circuit ouvert
			\2 Aucune des réponses ci-dessus

		\1 Les bouteilles de plongée sont disponibles dans une variété de tailles, de \st et de pressions d'usage.
			\2 Matériaux
			\2 Formes (rondes, carrées, ...)
			\2 Types (amont, avales,)
			\2 Toutes les réponses ci-dessus


		\1 Les robinetteries \st ont deux sorties, permettant aux plongeurs de connecter un détendeur redondant à une mono-bouteille.
			\2 A et Z
			\2 C et X
			\2 H et Y
			\2 K et Q
%30	
		\1 Les ponts isolateurs sont des vannes spéciales reliant deux bouteilles, permettant au plongeur d'avoir accès à la totalité du gaz en ne respirant que sur un seul détendeur.\vf

		\1 Toutes les bouteilles de plongée transportées par un plongeur doivent être accompagnées \st sauf si les bouteilles sont connectées entre elles par un pont isolateur
			\2 De cerclages
			\2 De caches
			\2 D'un manomètre
			\2 D'un système de flottabilité

		\1 Les combinaisons humides et étanches aideront à prévenir l'\st; cepedant, des précautions doivent être prises pour éviter la surchauffe
			\2 Hyperthermie
			\2 Hydrothermal
			\2 Hypochondrie
			\2 Hypothermie

		\1 Les masques avec une jupe \st sont préférés car ils réduisent les éblouissements à l'intérieur des lentilles.
			\2 Claire
			\2 Jaune
			\2 Opaque
			\2 Noire

		\1 \st et \st offrent aux plongeurs la possiblité de collecter des informations et de communiquer avec les membres de l'équipe.
			\2 Ardoises
			\2 Calepins	étanches
			\2 Les réponse A et B sont correctes
			\2 Aucune des réponses ci-dessus
%35
		\1 Lors d'une plongée à décompression, un minimum de trois dévidoirs doit être transporté par chaque membre de l'équipe.\vf

		\1 Chaque plongeur, peu importe \st, devrait comprendre comment utiliser et avoir accès à un analyseur de gaz.
			\2 Le niveau de certification
			\2 Le type d'équipement possédé
			\2 L'endroit dans le monde
			\2 Le genre

		\1 Quand un équipement est gréé, le plongeur doit placer les accessoires dans son \st
			\2 Triangle de vie
			\2 Sillage
			\2 Champ de vision
			\2 Aucune des réponses ci-dessus

		\1 Les protections de \st n'enlèvent pas la nécessité de contrôler une bouteille avant un changement de gaz.
			\2 Détendeurs
			\2 Bouteilles
			\2 Binôme
			\2 Aucune des réponses ci-dessus

		\1 Une position correcte impose au plongeur de modifier sa posture en fonction de:
			\2 La température de l'eau
			\2 Visibilité
			\2 Direction et vitesse de mouvement
			\2 Les réponse A et B sont correctes
%40
		\1 Différents styles de \st ont des applications uniques et tous doivent être maitrisés afin d'atteindre une efficacité optimale sans perturber les sédiments.
			\2 Corps
			\2 Descentes
			\2 Remontées
			\2 Palmage

		\1 Il est toléré de dévier de la respiration idéale quand c'est par choix. \vf

		\1 \st est composée de trois caractéristiques: la conscience personnelle, la conscience globale et la conscience situationnelle.
			\2 L'attention aigue
			\2 L'attention globale
			\2 La véritable attention
			\2 L'auto-attention

		\1 Dès que possible, les plongeurs doivent \st de se positionner dans les angles morts des autres membres
			\2 Essayer
			\2 S'efforcer
			\2 Tenter
			\2 Éviter

		\1 Le  \st considérera que tout va bien si les lampes de ses binômes sont visibles et stables.
			\2 Plongeur du milieu
			\2 Dernier plongeur
			\2 Chef de palanquée
			\2 Deuxième plongeur
%45
		\1 Considérez utiliser des phrases \st sur une ardoise qui peuvent être montrées durant la plongée.
			\2 Standardisées
			\2 Préplannifiées
			\2 Courtes
			\2 Simples

		\1 Les tombants, les épaves et les récifs sont des exemples d'objets pouvant être utilisés comme références \st.
			\2 Tactiles
			\2 De taille
			\2 Tangibles
			\2 Visuelles

		\1 Les plongeurs doivent respecter scrupuleusement les profondeurs maximales d'utilisation (MOD) pour:
			\2 Seulement les mélanges fonds
			\2 Seulement les mélanges de voyage
			\2 Tous les mélanges transportés
			\2 Les mélanges de décompression

		\1 \st d'une plongée va/vont dicter l'ordre dans lequel les phases logistiques de la plannifications seront éxectuées
			\2 Les objectifs
			\2 Le profil
			\2 La durée
			\2 Les facteurs limitants

		\1 La formule de profondeur maximale d'utilisation dérivée de la formule de Dalton est:
			\2 $MOD = Mix \div Dose$
			\2 $MOD = Dose \div Mix$
			\2 $MOD = Mix \times Dose$
			\2 $MOD = Dose \times Mix$
%50			
		\1 Les logiciels de décompression sont relativement \st et sont facilement disponibles pour les ordinateurs personnels, PDA et téléphones portables.
			\2 Coûteux
			\2 Bon marchés
			\2 Chers
			\2 Onéreux

		\1 La pression atmosphérique est dépendante de \st; Par conséquent, l'élévation du site de plongée par rapport au niveau de la mer doit être prise en compte dans le plan de plongée.
			\2 L'attitude
			\2 L'altitude
			\2 La longitude
			\2 La latitude

		\1 La pression minimale à laquelle un plongeur doit amorcer sa remontée peut être déterminée durant la plongée et n'est pas une étape nécessaire dans la planification pré-plongée.
			\2 Vrai
			\2 Faux

		\1 En plongeant sous plafond, des procédures différentes sont utlisées, notamment l'utilisation de la règle:
			\2 Des deux
			\2 Des tiers
			\2 Des quarts
			\2 Du parlement

		\1 Une équipe de plongeurs à décompression est idéalement composée de deux ou trois membres; cependant, une équipe plus grande peut être plus sûre pour les plongées \st en raison de la quantité de gaz et de l'équipement supplémentaire.
			\2 Longues et peu profondes
			\2 Courtes et profondes
			\2 Courtes et peu profondes
			\2 Longues et profondes
%55

		\1 Lors de la planification de plongées dans des endroits réculés, planifiez les plongées:
			\2 Intelligemment et rapidement
			\2 Rapidement et prudemment
			\2 Prudemment et de manière conservative
			\2 Rapidement et imprudemment

		\1 La méthode la plus inefficace pour faire face au stress est la relaxation et le contrôle de la respiration. \vf

		\1 Si une panne de lampe devait avoir lieu en cours de plongée:
			\2 Rester avec les membres de l'équipe et déployer une lampe de secours
			\2 Signaler et indiquer le problème aux autres membres
			\2 Terminer la plongée
			\2 Toutes les réponses ci-dessus

		\1 Si un emmêlement se produit:
			\2 S'arrêter, signaler à l'équipe, essayer de se libérer, attendre l'aide d'un binôme
			\2 Lutter et se tourner afin de se libérer
			\2 Gonfler son système de flottabilité, retenir sa réspiration et nager vers la surface
			\2 Toutes les réponses ci-dessus

		\1 La plupart des \st peuvent être arrétés(es) en fermant la robinetterie
			\2 Hernies
			\2 Hemostatins
			\2 Hémorragies
			\2 Haies
%60
		\1 Si une plongée a été correctement planifiée, chaque plongeur aura suffisament \st pour que deux plongeurs rejoignent la surface en complétant leurs paliers de décompression.
			\2 D'équipement
			\2 De plombs
			\2 De scooter
			\2 De gaz

		\1 Si \st crée(nt) un dépassement de profondeur, ou de temps de plongée, l'équipe doit utiliser une planification de secours et effectuer les paliers de décompression selon.
			\2 Le jour de la semaine
			\2 Des circonstances prévues
			\2 Des circonstances imprévues
			\2 Le moment de la journée

		\1 L'habilité a déployer efficacement un \st est une compétence nécessaire qui prend du temps et de la pratique pour se développer.
			\2 BSMB
			\2 SSMB
			\2 DSMB
			\2 MBAS

		\1 Le Driver Ambulance Network (DAN) recommande d'effectuer un examen neurologique dès qu'un plongeur est suspecté d'accident de décompression. \vf

		\1 La recompression dans l'eau est une procédure extrêmement \st.
			\2 Sûre
			\2 Sans risque
			\2 Risquée
			\2 Infaillible
%65
		\1 La création d'une liste d'objets \st réduit les risques d'oublier du matériel important.
			\2 Jolis à avoir
			\2 Inutiles
			\2 Nécessaires
			\2 De luxe

		\1 Chaque membre de l'équipe n'a pas besoin de comprendre complètement tous les aspects de la plongée prévue avant de se préparer, aussi longtemps que le chef d'équipe les comprend. \vf

		\1 \st dans l'eau, tester chaque pièce d'équipement et s'assurer de la disponibilité durant la plongée
			\2 Avant d'entrer
			\2 Après être entré
			\2 En entrant
			\2 Aucune des réponses ci-dessus

		\1 L'équipe doit effectuer un(e) \st approfondi(e) après chaque plongée de décompression.
			\2 Briefing
			\2 Débriefing
			\2 Séance de planification
			\2 Aucune des réponses ci-dessus 

		\1 Les disscussions post-plongée effectuées \st la plongée rassembleront des informations qui sont encore fraiches et permettent de fournir une période de repos avant la fatigue d'après-plongée.
			\2 Directement après
			\2 Avant
			\2 Une à deux heures après
			\2 Dans les douze heures après
%70
		\1 Un compte rendu complet comprendra une discussion portant sur:
			\2 Identification des problèmes rencontrés et de l'analyse de la réaction de l'équipe.
			\2 Précision du plan de plongée.
			\2 Caractéristiques notables du site de plongée
			\2 Toutes les réponses ci-dessus

	\end{outline}
\end{document} 
