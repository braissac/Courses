%!TEX root = decoproc_fr.tex

\section{Physiologie}

\begin{frame}{Système respiratoire}
	\mypict{../img/sys_resp.png}
\end{frame}

\begin{frame}{Système sanguin}
	\mypict{../img/circulation_sanguine}
\end{frame}

%---------------------------------------------------------------------
%										Echanges gazeux
%---------------------------------------------------------------------
\subsection{Echanges gazeux}

\begin{frame}{Echanges gazeux}  
	Diffusion:
	\begin{itemize}
		\item Echange gazeux entre:
		\begin{itemize}
		 	\item Les alvéoles pulmonaires et les capillaires (poumons)
		 	\item Les capillaires et les tissus (corps)
		 \end{itemize}
		\item Mouvement de molécules
		\item Haute concentration $\rightarrow$ Basse concentration
		\item Loi d'Henry\\ (Quantité de gaz dissous proportionnelle à la pression partielle)
	\end{itemize}
\end{frame}

\begin{frame}{Echanges gazeux}  
	Perfusion tissulaire:
	\begin{itemize}
		\item Quantité de flux sanguin qu'un tissu reçoit
		\item Perfusion élevée = Compartiment rapide\\(muscles, foie, reins, poumons et système nerveux)
		\item Perfusion lente = Compartiment lent\\(tendons, ligaments, os et cartilages)
	\end{itemize}
\end{frame}

\begin{frame}{Echanges gazeux}  
	Surface (saturation):
	\begin{itemize}
		\item Azote: Gaz inerte, peu ou pas de diffusion
		\item Oxygène: Se diffuse dans le coprs
		\begin{itemize}
			\item  Alvéoles pulmonaires$\rightarrow$  système sanguin
		\end{itemize}
		\item Dioxide de carbone: Déchet produit par le métabolisme
		\begin{itemize}
			\item Système sanguin $\rightarrow$  alvéoles pulmonaires
		\end{itemize}
	\end{itemize}
\end{frame}

\begin{frame}{Echanges gazeux}  
	Descente (sous-saturation):
	\begin{itemize}
		\item Pression tissulaire < Pression partielle respirée\\
				$\hookrightarrow$ L'azote se diffuse dans le système sanguin
	\end{itemize}
	\vfill
	Au fond (saturation):
	\begin{itemize}
		\item Pression tissulaire = Pression partielle respirée
		\item Etat de saturation (plus d'échange de gaz inertes)
		\item Temps avant saturation différents pour chaques compartiments
	\end{itemize}
	\vfill
	Remontée (super-saturation):
	\begin{itemize}
		\item Pression tissulaire > Pression partielle respirée\\
				$\hookrightarrow$ L'azote se diffuse vers le système respiratoire
	\end{itemize}
\end{frame}

\begin{frame}{Echanges gazeux}  
	Niveau de saturation dépend de:
	\begin{itemize}
		\item Diffusion (profondeur et gaz)
		\item Perfusion (type de tissu)
		\item Temps d'exposition
	\end{itemize}
\end{frame}

%---------------------------------------------------------------------
%										Formation de bulles
%---------------------------------------------------------------------
\subsection{Formation de bulles}

\begin{frame}{Formation de bulles}  
	Gaz dissous dans le sang\\
	Liaisons moléculaires garde le gaz en solution\\
	Diminution de pression ambiante:
	\begin{itemize}
		\item Diffusion du gaz vers les poumons
		\item Tension trop élevée = formation de bulles
	\end{itemize}
\end{frame}

\begin{frame}{Formation de bulles}
	Micronoyaux gazeux:
	\begin{itemize}
		\item Fragilisation des liaisons moléculaires facilitant la formation de bulles
	\end{itemize}
	\vfill
	Cause:
	\begin{itemize}
		\item Modification de la circulation sanguine
		\item Mouvement des tissus et des articulations
		\item Supersaturation localisée
	\end{itemize}
\end{frame}

%---------------------------------------------------------------------
%										M-Values
%---------------------------------------------------------------------
\subsection{M-Values}

\begin{frame}{M-Values}  
	M-Values:
	\begin{itemize}
		\item Niveau maximum de super-saturation toléré afin d'éviter un niveau excessif de bulles
	\end{itemize}
	\vfill
	Propriétés:
	\begin{itemize}
		\item Perfusion sanguine détermine le type de compartiment tissulaire
		\item M-Values différentes pour chaques compartiments
		\item M-Values varient en fonction de la pression ambiante (profondeur)
	\end{itemize}
\end{frame}

\begin{frame}{M-Values}
	Limite de non-décompression:
	\begin{itemize}
		\item Possibilité de remonter à la surface sans atteindre la limite de super-saturation (M-Values)
	\end{itemize}
	\vfill
	Dépassement des limites = paliers de décompression obligatoires
	\vfill
	Le principe des M-Values est un modèle est n'est donc pas parfait!
\end{frame}

%---------------------------------------------------------------------
%										Bulles
%---------------------------------------------------------------------
\subsection{Bulles}
\begin{frame}{Bulles}
	Réaction du corps au bulles gazeuses
	\begin{itemize}
		\item Changement moléculaire des protéines du sang
		\item Réaction du système immunitaire
		\item Leucocytes (Globules blancs) entourent la bulle de fibrine
		\item Plus de diffusion possible!
		\item Sécrétion d'histamine (inflamation)
		\item Altération du flux sanguin et coagulation
		\item Anoxie et necrose des tissus
	\end{itemize}
\end{frame}

\begin{frame}{Bulles silencieuses}  
	Bulles silencieuses:
	\begin{itemize}
		\item Bulles suffisament petites pour ne pas altérer la circulation sanguine
		\item Véhiculées jusqu'au poumons
		\item Diffusion à travers les alvéoles
		\item Détectée à l'aide d'un Doppler
		\item Peut s'accumuler dans les capillaires autour des alvéoles et limiter la diffusion
		\item Fatigue et limite les plongées répétitives
	\end{itemize}
\end{frame}

%---------------------------------------------------------------------
%										Accident de décompression
%---------------------------------------------------------------------
\subsection{Accident de décompression}

\begin{frame}{Accident de décompression}  
	Symptômes:
	\begin{multicols}{2}
		\begin{itemize}
			\item Démangeaisons
			\item Difficulté à uriner
			\item Douleur articulaire
			\item Douleur musculure
			\item Eruptions cutannées
			\item Essoufflements
			\item Evanouissement
			\item Faiblesses
			\item Fatigue
			\item Hémoptysie (crachats sanguinolents)
			\item Inflammations
			\item Maux de tête
			\item Nausée
			\item Paralysie
			\item Perte de conscience
			\item Perte de mémoire
			\item Perte de sensibilité
			\item Trouble de l'ouïe
			\item Trouble du comportement
			\item Troubles visuels
			\item Vertiges
			\item Vomissement
		\end{itemize}
	\end{multicols}
\end{frame}

\begin{frame}{Accident de décompression}
	Traitement:
	\begin{itemize}
		\item Oxygénothérapie: accélère la diffusion et cicatrise les tissus endommagés
		\item Aspirine: anti-coagulant
		\item Hydratation: compense la perte de plasma durant la plongée
		\item Recompression: réduit la taille des bulles
	\end{itemize}
\end{frame}

%---------------------------------------------------------------------
%										Surpression pulmonaire
%---------------------------------------------------------------------
\subsection{Surpression pulmonaire}

\begin{frame}{Surpression pulmonaire}
	Cause:
	\begin{itemize}
		\item Augmentation du volume pulmonaire à la remontée
		\item Blocage ou expiration insuffisante
		\item Surpression pulmonaire et déchirure alvéolaire
	\end{itemize}
	\vfill
	Conséquences:
	\begin{itemize}
		\item Pneumothorax:\\Rupture du feuillet interne de la plèvre
		\item Emphysème (du médiastin ou sous-cutané):\\Rupture des deux feuillets de la plèvre
		\item Embolie gazeuse:\\Passage de l'air à travers les capillaires pulmonaires et dans le système sanguin
	\end{itemize}
\end{frame}
	
\begin{frame}{Surpression pulmonaire}
	\mypict{../img/surpression}
\end{frame}

\begin{frame}{Surpression pulmonaire}  
	Symptômes:
	\begin{multicols}{2}
		\begin{itemize}
			\item Arrêt cardiaque
			\item Convulsions
			\item Crachats sanguinolents
			\item Désorientation
			\item Douleurs thoraciques
			\item Etourdissement
			\item Evanouissement
			\item Faiblesses
			\item Paralysie
			\item Perte de sensibilté
			\item Sang dans la bouche ou dans le nez
			\item Troubles visuels
		\end{itemize}
	\end{multicols}
\end{frame}

\begin{frame}{Augmentation des risques}
	Facteurs augmentant les risques
	\begin{itemize}
		\item Age
		\item Altitude
		\item Blessures
		\item Cigarettes, fumée
		\item Deshydratation
		\item Exercice
		\item FOP (Foramen Oval Perforé)
		\item Obésité
		\item Température
	\end{itemize}
\end{frame}

\begin{frame}{Augmentation des risques}  
	Altitude:
	\begin{itemize}
			\item Diminution de la pression ambiante
			\item Gradient de pression en fin de plongée 
		\end{itemize}	
\end{frame}

\begin{frame}{Augmentation des risques}
	Cigarette:
	\begin{itemize}
		\item Obstructions pulmonaires chroniques
		\item Fragilise les alvéoles pulmonaires
	\end{itemize}
	Obésité:
	\begin{itemize}
		\item Solubilité élevée de l'azote
		\item Faible perfusion
		\item Risque lors de profondes/longues plongées
	\end{itemize}
	Exercice
	\begin{itemize}
		\item Augmente la quantité de micronoyaux, donc favorise la formation de bulles
		\item Augmente le rythme circulatoire (perfusion), donc la quantité d'azote absorbée
	\end{itemize}
\end{frame}
