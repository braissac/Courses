%!TEX root = decoproc_fr.tex

\section{Techniques et procédures}

\begin{frame}{Techniques et procédures}  
	\textbf{Trois techniques de base:}
	\begin{itemize}
		\item FLOTTABILITÉ
		\item TRIM (position)
		\item PROPULSION (palmage)
	\end{itemize}
\end{frame}

%---------------------------------------------------------------------
%										Flottabilité
%---------------------------------------------------------------------
\subsection{Flottabilité}

\begin{frame}{Flottabilité}  
	\begin{center}
		\resizebox{!}{18pt}{\strut\textbf{\insertsubsection}\strut}\\
		Habilité à maintenir une position constante dans la colonne d'eau
	\end{center}
	\vfill
	\begin{multicols}{2}
		Techniques:
		\begin{itemize}
			\item BCD principalement
			\item Poumons ballast
			\item Combinaison étanche
			\item Pression des bouteilles
		\end{itemize}

		\columnbreak

		Bénéfices:
		\begin{itemize}
			\item Économie d'énergie
			\item Éfficacité des paliers
			\item Protection des fonds marins,
			\item ...
		\end{itemize}
	\end{multicols}
\end{frame}

%---------------------------------------------------------------------
%										Trim
%---------------------------------------------------------------------
\subsection{Trim}

\begin{frame}{Trim}
	\begin{center}
		\resizebox{!}{18pt}{\strut\textbf{\insertsubsection}\strut}\\
		Orientation et position du corps dans la colonne d'eau
		\vfill
		Corps plat, genoux plié, bras en avant, tête relevée
	\end{center}
	\vfill
	\begin{multicols}{2}
		Changements:
		\begin{itemize}
			\item Direction de voyage
			\item Lestage
			\item Air dans la wing
			\item Combinaison étanche
			\item Pression des bouteilles
		\end{itemize}

		\columnbreak

		Bénéfices:
		\begin{itemize}
			\item Économie d'énergie
			\item Éfficacité des paliers
			\item Protection des fonds marins,
			\item ...
		\end{itemize}
	\end{multicols}
\end{frame}

\begin{frame}{Trim}  
	\mypict{../img/drysuit_trim}
\end{frame}

%---------------------------------------------------------------------
%										Propulsion
%---------------------------------------------------------------------
\subsection{Propulsion}

\begin{frame}{Propulsion}
	\begin{center}
		\resizebox{!}{18pt}{\strut\textbf{\insertsubsection}\strut}\\
		Habilité à se déplacer dans la colonne d'eau
	\end{center}
	\vfill
	\begin{multicols}{2}
		Techniques:
		\begin{itemize}
			\item Frog kick
			\item Frog kick modifé
			\item Back kick
			\item Helicopter turn
			\item Flutter kick
			\item Pull and glide
		\end{itemize}

		\columnbreak

		Bénéfices:
		\begin{itemize}
			\item Économie d'énergie
			\item Facilité de déplacement
			\item Maitrise du positionnement
			\item Protection des fonds marins,
			\item ...
		\end{itemize}
	\end{multicols}
\end{frame}

\begin{frame}{Propulsion}  
	\textbf{Frog kick:}\\
	Utilisé en permanence
	
	\begin{enumerate}
		\item Garder son trim horizontal
		\item Palmes à plat, parallèles au sol
		\item Angles des genoux entre $90^{\circ}$ et $120^{\circ}$
		\item Ouvrir les chevilles
		\item Pousser en étandant et fermant les jambes, sans abaisser les genoux
		\item Remettre les palmes à plat
		\item Plier les genoux
		\item ... et recommencer! 
	\end{enumerate}
\end{frame}

\begin{frame}{Propulsion}  
	\textbf{Frog kick modifié:}\\
	Utilise uniquement les chevilles

	\begin{enumerate}	
		\item Garder son trim horizontal
		\item Palmes à plat, parallèles au sol
		\item Angles des genoux entre $90^{\circ}$ et $120^{\circ}$
		\item Ouvrir les chevilles en gardant les talons ensembles
		\item Fermer les chevilles (applaudir avec les pieds)
		\item Remettre les palmes à plat
		\item ... et recommencer!
	\end{enumerate}
\end{frame}

\begin{frame}{Propulsion}  
	\textbf{Back kick}\\
	Le plus important!\\
	Permet de se stabiliser et garder sa position

	\begin{enumerate}
		\item Garder son trim horizontal
		\item Angles des genoux entre $90^{\circ}$ et$120^{\circ}$
		\item Etendre les genoux EN GARDANT LES PALMES A PLAT! \\
				(des palmes qui remontent entraine un mouvement en avant)
		\item Ouvrir les chevilles
		\item Replier le bas des jambes en garder les genoux au même niveau
		\item Ramener les genoux et pieds ensembles
		\item ... et recommencer!
	\end{enumerate}
\end{frame}

\begin{frame}{Propulsion}  
	\textbf{Helicopter turn}
	\begin{itemize}
		\item Permet de pivoter sur soi-même
		\item Frog kick d'une seule jambe
		\item Back kick modifié d'une seule jambe\\(utilise le dessus de la palme)
	\end{itemize}
\end{frame}

%---------------------------------------------------------------------
%										Conscience et perception
%---------------------------------------------------------------------
\subsection{Conscience et perception}

\begin{frame}{Conscience et perception}  
	La \textbf{véritable attention} est composée de trois caractéristiques:
	\begin{itemize}
		\item La conscience personnelle
		\item La conscience globale
		\item La conscience situationnelle
	\end{itemize}
\end{frame}

\begin{frame}{Conscience et perception} 
	\textbf{Conscience personnelle:}\\
	\begin{itemize}
		\item Respiration
		\item Flottabilité
		\item Trim
		\item Propulsion
		\item Equipement hydrodynamique
		\item Reserve d'air
		\item Profondeur, temps
		\item Planification
		\item ...
	\end{itemize}
\end{frame}

\begin{frame}{Conscience et perception}  
	\textbf{Conscience globale:}
	\begin{itemize}
		\item Position par rapport aux autres membres de l'équipe
		\item Contrôle des autres membres (équipement, consommation, état mental)
		\item Emplacement sur le cite de plongée
		\item Techniques adaptées au milieu (palmage)
		\item Dangers environnants (ligne, fillets, ...)
		\item ...
	\end{itemize}
\end{frame}

\begin{frame}{Conscience et perception}  
	\textbf{Conscience situationnelle:} Anticipation
	\begin{itemize}
		\item Consommation adaptée me permétant de continuer la plongée
		\item Possibilité de retourner aux bouteilles de décompression
		\item Courant affectant mon trajet de retour
		\item Anticipation de problemes
		\item Limite de non-décompression
		\item ...
	\end{itemize}
\end{frame}

%---------------------------------------------------------------------
%										Communication
%---------------------------------------------------------------------
\subsection{Communication}

\begin{frame}{Communication}  
	\textbf{Plusieurs types de communication}
	\begin{itemize}
		\item Positionnement
		\item Lampe
		\item Mains
		\item Écrite
		\item Verbale
	\end{itemize}
\end{frame}

\begin{frame}{Communication}  
	\textbf{Positionnement}
	\begin{itemize}
		\item Positionnement par rapport aux membres de l'équipe
		\item Côte-à-côte ou en V (mouvement)
		\item Face-à-face (statique)
	\end{itemize}
\end{frame}

\begin{frame}{Communication}  
	\textbf{Lampe:}
	\vfill
	\begin{columns}[onlytextwidth]
		\column{0.3\linewidth}
			Ok, tout va bien:
			\mypict{../img/light_ok}
		\column{0.3\linewidth}
			Besoin d'attention:
			\mypict{../img/light_look}
		\column{0.3\linewidth}
			Urgence:
			\mypict{../img/light_ooa}
	\end{columns}
\end{frame}

\begin{frame}{Communication}  
	\textbf{Signes:}
	\vfill
	\mypict{../img/hand/numbers}
\end{frame}

\begin{frame}{Communication}
	\mypict{../img/hand/ok}\\
	\mypict{../img/hand/problem}
\end{frame}

\begin{frame}{Communication}
	\mypict{../img/hand/level}\\
	\mypict{../img/hand/deco}
\end{frame}

\begin{frame}{Communication}
	\mypict{../img/hand/travel}
\end{frame}

\begin{frame}{Communication}
	\mypict{../img/hand/line}\\
	\mypict{../img/hand/entangle}
\end{frame}

%---------------------------------------------------------------------
%										Changement de gaz
%---------------------------------------------------------------------
\subsection{Changement de gaz}

\begin{frame}{Changement de gaz}
	\textbf{TDI:} MODS
	\begin{description}
		\item[M]Mix
		\item[O]Open
		\item[D]Depth
		\item[S]Switch
	\end{description}
	\textbf{PADI:} NOTOX
	\begin{description}
		\item[N]Noter (information sur la bouteille)
		\item[O]Observer (profondeur actuelle)
		\item[T]Tourner (ouvrir la robinetterie)
		\item[O]Orienter (le deuxième étage)
		\item[X]Echanger et contrôler
	\end{description}
\end{frame}











