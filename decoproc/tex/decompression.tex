%!TEX root = decoproc_fr.tex

\section{Théorie de la décompression}
%---------------------------------------------------------------------
%								Modèle d'Haldane
%---------------------------------------------------------------------
\subsection{Modèle d'Haldane}

\begin{frame}{Modèle d'Haldane}
	\textbf{John Scott Haldane}
	\begin{itemize}
		\item First dive tables based on scientific modèle
		\item Corps absorbe et relâche les gaz inertes
		\item Cinq compartiments
		\item Introduit le concept de sursaturation
		\item Niveau maximum: "limite critique"
		\item Remonter aussi proche de la limite critique que possible
	\end{itemize}
\end{frame}

%---------------------------------------------------------------------
%								Modèles Neo-Haldanian
%---------------------------------------------------------------------
\subsection{Modèles Neo-Haldanian}

\begin{frame}{Modèles Neo-Haldanian}  
	\textbf{Modèles Neo-Haldanian}
	\begin{itemize}
		\item Formation de bulles silencieuses (Doppler)
		\item Modèles à "Gas en dissolution"
		\item Workmann, US Navy, Bühlmann,...
		\item Modèles Haldanian modifiés
	\end{itemize}
\end{frame}

%---------------------------------------------------------------------
%								Paliers profonds
%---------------------------------------------------------------------
\subsection{Paliers profonds}

\begin{frame}{Paliers profonds}
	\textbf{Richard L. Pyle}
	\begin{itemize}
		\item Observations pratiques et non scientifiques
		\item Plongées dans la zone des 55m-65m
		\item Sensations de fatigues ou malaise
		\item Symptômes pas constants
		\begin{itemize}
			\item Pas de symptômes du tout
			\item Fatigué est impossible de conduire
		\end{itemize}
	\end{itemize}
\end{frame}

\begin{frame}{Paliers profonds}
	\textbf{Richard L. Pyle}
	\begin{itemize}
	 	\item Recherche de corrélation
	 	\begin{itemize}
	 		\item Exposition, paliers, courant, visiblité, température, nombre d'heures de sommeil, hydratation,...
	 		\item Nombre de poissons péchés durant la plongée!
	 	\end{itemize}
	\end{itemize} 	
\end{frame}

\begin{frame}{Paliers profonds}
	\textbf{Richard L. Pyle}
	\begin{itemize}
	 	\item Vessie natatoire
	 	\item Doit être vidée à la remontée
	 	\item Arrêt à mi-chemin entre le fond et le premier palier
	 	\item Arrêt de 2-3 minutes
	\end{itemize} 	
\end{frame}

\begin{frame}{Paliers profonds}
	\textbf{Paliers profonds ou Pyle's Stop:}
	\begin{enumerate}
		\item Calculer sa planification
		\item Ajouter un palier de 2-3 minutes à mi-chemin entre le fond et le premier palier de décompression
		\item Recalculer la planification avec le palier profond
		\item Si la distance entre le palier profond et le premier palier de décompression est supérieur à 9m, ajouter un palier profond
		\item Recommencer tant que nécessaire
	\end{enumerate}	
\end{frame}

%---------------------------------------------------------------------
%								Facteurs de Gradient
%---------------------------------------------------------------------
\subsection{Facteurs de Gradient}

\begin{frame}{Facteurs de Gradients}  
	\textbf{Erik  C. Baker}
	\begin{itemize}
		\item Plongeur trimix, spéléo, et ingénieur en électronique
		\item Incohérences dans le conservatisme des paliers profonds
		\item Idée de créer les facteurs de gradient
		\item Basé sur les tables de Bühlmann
	\end{itemize}
\end{frame}

\begin{frame}{Facteurs de Gradients}  
	\textbf{Facteurs de Gradients}
	\begin{itemize}
		\item Fourni un niveau de conservatisme cohérent
		\item Paliers profonds dans la zone de décompression
		\item Contrôle la sursaturation
		\item Marge de sécurité de la M-Value
		\item Exprimé en pourcentage (100\% = M-Value)
		\item Facteurs de gradient bas (fond) et haut (surface)
	\end{itemize}
\end{frame}

\begin{frame}{Facteurs de Gradients}  
	\mypict{../img/gradient_factors}
\end{frame}

%---------------------------------------------------------------------
%								VGM
%---------------------------------------------------------------------
\subsection{VGM}


%---------------------------------------------------------------------
%								Modèles à double phase
%---------------------------------------------------------------------
\subsection{Modèles à double phase}


%---------------------------------------------------------------------
%								Comportement des bulles
%---------------------------------------------------------------------
\subsection{Comportement des bulles}


%---------------------------------------------------------------------
%								Quantité de bulles
%---------------------------------------------------------------------
\subsection{Quantité de bulles}


%---------------------------------------------------------------------
%								VPM
%---------------------------------------------------------------------
\subsection{VPM}


%---------------------------------------------------------------------
%								RGBM
%---------------------------------------------------------------------
\subsection{RGBM}