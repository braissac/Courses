%!TEX root = decoproc_fr.tex

\section{Théorie de la décompression}
%---------------------------------------------------------------------
%								Modèle d'Haldane
%---------------------------------------------------------------------
\subsection{Modèle d'Haldane}

\begin{frame}{Modèle d'Haldane}
	\textbf{John Scott Haldane}
	\begin{itemize}
		\item Premières tables de plongée basées sur un modèle scientifique
		\item Corps absorbe et relâche les gaz inertes
		\item Cinq compartiments
		\item Introduit le concept de sursaturation
		\item Niveau maximum: "limite critique"
		\item Remonter aussi proche de la limite critique que possible
	\end{itemize}
\end{frame}

%---------------------------------------------------------------------
%								Modèles Neo-Haldanian
%---------------------------------------------------------------------
\subsection{Modèles Neo-Haldanian}

\begin{frame}{Modèles Neo-Haldanian}  
	\textbf{Modèles Neo-Haldanian}
	\begin{itemize}
		\item Formation de bulles silencieuses (Doppler)
		\item Modèles à "Gas en dissolution"
		\item Workmann, US Navy, Bühlmann,...
		\item Modèles Haldanian modifiés
	\end{itemize}
\end{frame}

%---------------------------------------------------------------------
%								Paliers profonds
%---------------------------------------------------------------------
\subsection{Paliers profonds}

\begin{frame}{Paliers profonds}
	\textbf{Richard L. Pyle}
	\begin{itemize}
		\item Observations pratiques et non scientifiques
		\item Plongées dans la zone des 55m-65m
		\item Sensations de fatigues ou malaise
		\item Symptômes pas constants
		\begin{itemize}
			\item Pas de symptômes du tout
			\item Fatigué est impossible de conduire
		\end{itemize}
	\end{itemize}
\end{frame}

\begin{frame}{Paliers profonds}
	\textbf{Richard L. Pyle}
	\begin{itemize}
	 	\item Recherche de corrélation
	 	\begin{itemize}
	 		\item Exposition, paliers, courant, visiblité, température, nombre d'heures de sommeil, hydratation,...
	 		\item Nombre de poissons péchés durant la plongée!
	 	\end{itemize}
	\end{itemize} 	
\end{frame}

\begin{frame}{Paliers profonds}
	\textbf{Richard L. Pyle}
	\begin{itemize}
	 	\item Vessie natatoire
	 	\item Doit être vidée à la remontée
	 	\item Arrêt à mi-chemin entre le fond et le premier palier
	 	\item Arrêt de 2-3 minutes
	\end{itemize} 	
\end{frame}

\begin{frame}{Paliers profonds}
	\textbf{Paliers profonds ou Pyle's Stop:}
	\begin{enumerate}
		\item Calculer sa planification
		\item Ajouter un palier de 2-3 minutes à mi-chemin entre le fond et le premier palier de décompression
		\item Recalculer la planification avec le palier profond
		\item Si la distance entre le palier profond et le premier palier de décompression est supérieur à 9m, ajouter un palier profond
		\item Recommencer tant que nécessaire
	\end{enumerate}	
\end{frame}

%---------------------------------------------------------------------
%								Facteurs de Gradient
%---------------------------------------------------------------------
\subsection{Facteurs de Gradient}

\begin{frame}{Facteurs de Gradients}  
	\textbf{Erik  C. Baker}
	\begin{itemize}
		\item Plongeur trimix, spéléo, et ingénieur en électronique
		\item Incohérences dans le conservatisme des paliers profonds
		\item Idée de créer les facteurs de gradient
		\item Basé sur les tables de Bühlmann
	\end{itemize}
\end{frame}

\begin{frame}{Facteurs de Gradients}  
	\textbf{Facteurs de Gradients}
	\begin{itemize}
		\item Fourni un niveau de conservatisme cohérent
		\item Paliers profonds dans la zone de décompression
		\item Contrôle la sursaturation
		\item Marge de sécurité de la M-Value
		\item Exprimé en pourcentage (100\% = M-Value)
		\item Facteurs de gradient par paire:
		\begin{itemize}
			\item GF(Low) contrôle la profondeur du premier palier\\(Niveau de sursaturation avant le premier palier)
			\item GF(High) contrôle la durée du dernier palier\\(Niveau de sursaturation à la fin de la plongée)
		\end{itemize}
	\end{itemize}
\end{frame}

\begin{frame}{Facteurs de Gradients}  
	\mypict{../img/gradient_factors}
\end{frame}

%---------------------------------------------------------------------
%								Modèles à double phase
%---------------------------------------------------------------------
\subsection{Modèles à bulles}

\begin{frame}{Modèles à bulles}
	\textbf{Albert Richard Behnke Jr (1903 - 1992)}  
	\begin{itemize}
		\item Médecin pour l'institut de recheche médicale de l'U.S. Navy
		\item Différencies l'embolie gazeuse (AGE) des accidents de décompression
		\item Ce n'est pas l'éxistance de bulles silencieuses, maisla quantité de gaz libre qui crée un accident de décompression
	\end{itemize}
\end{frame}

\begin{frame}{Modèles à bulles}  
	\textbf{Brian Hills \& Dr. Hugh LeMessurier ($\sim$ 1963)}
	\begin{itemize}
		\item Etudient les plongeur de perles en Australie
		\item Ne plongent pas selon les tables Buhlamnn
		\item Système empirique, paliers plus profonds et remontée direct depuis 7m.
		\item Temps de décompression réduit de 2/3 par rapport aux tables de l'US Navy
		\item Nouvelle approche: étudier les gaz dissous et les gaz à l'état libre (bulles)
		\item Peu de soutien, trop différent, passé aux oubliettes!
	\end{itemize}
\end{frame}

%---------------------------------------------------------------------
%								Comportement des bulles
%---------------------------------------------------------------------

\begin{frame}{Comportement des bulles}  
	\textbf{Comportement}
	\begin{itemize}
	\item  Pour qu'une bulle grandisse, il faut que la pression interne soit plus grande que la pression externe.
	\end{itemize}
\end{frame}

\begin{frame}{Comportement des bulles}
    \begin{columns} % the "c" option specifies center vertical alignment

    \column{.5\textwidth}
		\textbf{Pression interne}
		\begin{itemize}
			\item Presion d'azote dans la bulle
		\end{itemize}

		\vspace{1em}
		
		\textbf{Pression externe}
		\begin{itemize}
			\item Pression ambiante
			\item Tension de surface (peau de la bulle)\\Appelée pression de Laplace
		\end{itemize}
    \column{.4\textwidth}
     	\mypict{../img/bubble}
    \end{columns}
\end{frame}

\begin{frame}{Comportement des bulles}
	\textbf{Changement de taille des bulles}
		\begin{itemize}
			\item Pression ambiante
			\item Concentration de gaz dans la bulle
		\end{itemize}
\end{frame}

\begin{frame}{Comportement des bulles}  
	\textbf{Comportement à saturation}
	\begin{itemize}
		\item Equilibre des pressions:$$P_{interne} = P_{ambiante}+P_{Laplace}$$ \pause
		\item Pression tissulaire environnante:$$P_{tissus}= PN{_2}_{ambiante}$$ \pause
		\item Diminution de taille (diffusion externe) si: $$P_{interne} > P_{tissus}$$
	\end{itemize}
\end{frame}

\begin{frame}{Comportement des bulles}  
	\textbf{Exemple:}
	\begin{itemize}
		\item Plongeur respirant de l'air à 40m:
				\begin{align*}
					P_{ambiante} &= 5 bar\\
					P_{Laplace} &= .5 bar\\
					P_{intene} &= P_{ambiante} + P_{Laplace} = 5.5 bar
				\end{align*} \pause

		\item Pression dans les tissus (saturés!):
				$$P_{tissus} = FN_2 \times P_{ambiante} = .79 \times 5 =3.95 bar $$ \pause

		\item La taille de la bulle va diminuer!
	\end{itemize}
\end{frame}

\begin{frame}{Comportement des bulles}  
	\textbf{Remontée}
	\begin{itemize}
		\item Changement rapide de pression ambiante
		\item Pressions tissulaires dépendantes des périodes de chaques compartiments
		\pause \vfill
		\item Pression interne de la bulle dimiue
		\pause \vfill
		\item Lorsque la pression interne est inférieure à la pression tissulaire, le gaz se diffuse dans la bulle, et celle-ci augmente!
		\pause \vfill
		\item Les paliers de décompression limite la pression ambiante pour éviter ce phénomène
	\end{itemize}
\end{frame}

\begin{frame}{Comportement des bulles}  
	\textbf{Exemple:}
	\begin{itemize}
		\item Le même plongeur remonte rapidement à 20m:
				\begin{align*}
					P_{ambiante} &= 3 bar\\
					P_{Laplace} &= .5 bar\\
					P_{intene} &= 3.5 bar
				\end{align*} \pause
		\item Pression tissulaire n'a pas eu le temps de s'adapter:
				$$P_{tissus} =3.95 bar $$ \pause

		\item La taille de la bulle va augmenter!
	\end{itemize}
\end{frame}

\begin{frame}{Comportement des bulles}
	\textbf{Pression de Laplace}
	\begin{itemize}
		\item Directement liée au rayon de la bulle
		\item Rayon important, pression de Laplace faible\\(la "peau" de la bulle devient plus fine)
		\item Lorsque la bulle grandi, la pression externe diminue, et la bulle grandi encore plus rapidement!
		\item Des bulles de différentes taille ne réagiront pas pareillement
		\vfill
		\item Les grosses bulles augmentent, les petites bulles diminuent
	\end{itemize}
\end{frame}

\begin{frame}{Comportement des bulles}
	\textbf{Rayon critique}
	\begin{itemize}
		\item Rayon à partir duquel une bulle commence à grossir
		\item Modèles essaient de contrôler la taille des bulles
		\item En dessous du rayon critique, les bulles vont diminuer lorsque le plongeur remonte
	\end{itemize}
\end{frame}

%---------------------------------------------------------------------
%								Quantité de bulles
%---------------------------------------------------------------------
\subsection{Quantité de bulles}

\begin{frame}{Quantité de bulles}  
	\textbf{Réalité:}
	\begin{itemize}
		\item Corps humain complexe
		\item Plusieurs compartiments (période et perfusion variées)
		\item Bulles de toutes les tailles
	\end{itemize}
\end{frame}

\begin{frame}{Quantité de bulles}
	\textbf{Tissus rapides}
	\begin{itemize}
		\item Se saturent durant la descente et au fond
		\item Haut niveau de sursaturation à la remontée
		\item Creation de bulles
		\item Paliers profonds nécessaires
	\end{itemize}
	\vfill
	\textbf{Tissus lents}
	\begin{itemize}
		\item Se saturent durant les paliers profonds
		\item Augmente la durée totale des paliers
	\end{itemize}
\end{frame}

\begin{frame}{Quantité de bulles}  
	\textbf{Bulles}
	\begin{itemize}
		\item Bulles silencieuses ne cause pas d'ADD
		\item Beaucoup de petites bulles, peu de grandes
		\item Modèles contrôle le rayon critique
		\item Laisse grandir que les bulles d'une certaine taille
		\item Permet de contrôler le volume de gaz inerte
	\end{itemize}
\end{frame}

%---------------------------------------------------------------------
%								VPM: Modèle à perméabilité variable
%---------------------------------------------------------------------
\subsection{VPM: Modèle à perméabilité variable}

\begin{frame}{VPM: Modèle à perméabilité variable}
	\textbf{Principe:}
	\begin{itemize}
		\item Basé sur le modèle de Buhlmann
		\item 16 compartiments
		\item Ajoute une seule bulle dans chaque compartiment
		\item Modèle à gaz dissous et à bulles combinés
	\end{itemize}
\end{frame}

\begin{frame}{VPM: Modèle à perméabilité variable}
	\textbf{Perméablité variable:}
	\begin{itemize}
		\item "Peau" des bulles composée de micro-billes (molécules)
		\item La diffusion se fait entre les molécules
		\item Lorsque la presion ambiante augmente, la perméabilité diminue
		\item D'où le nom: Perméabilité variable
	\end{itemize}
\end{frame}

\begin{frame}{VPM: Modèle à perméabilité variable}
	\textbf{Perméablité variable:}\\
	\mypict{../img/permeability}
\end{frame}

\begin{frame}{VPM: Modèle à perméabilité variable}  
	\textbf{Fonctionnement de l'alogrithme:}
	\begin{itemize}
		\item Diffusion des gaz selon Buhlmann
		\item VPM surveille le rayon des bulles
		\item VPM calcule la pression interne de la bulle
		\item Comparaison entre Buhlmann et VPM pour savoir si une bulle grandi
		\item Palier avant qu'une bulle grandisse
	\end{itemize}
\end{frame}

\begin{frame}{VPM: Modèle à perméabilité variable}  
	\textbf{Variante:}
	\begin{itemize}
		\item Original: aucune bulle ne grandi - trop conservateur
		\item Corps peu gérer un certain volume de gaz libre
		\item Seules les bulles d'une certaine taille peuvent grandir
	\end{itemize}
\end{frame}

\begin{frame}{VPM: Modèle à perméabilité variable}
	\textbf{VPM-B:}
	\begin{itemize}
		\item Incorpore la loi de Boyle
		\item Corrige la pression de Laplace avec le nouveau volume (plus grand)
		\item Ajout de conservatisme
	\end{itemize}
	\vfill
	\textbf{VPM-B/E:}
	\begin{itemize}
		\item Algorithme non divulgué
		\item Ajout de conservatisme pour les plongées extrêmes
	\end{itemize}
\end{frame}
%---------------------------------------------------------------------
%								RGBM
%---------------------------------------------------------------------
\subsection{RGBM: Reduce Gradient Bubble Model}

\begin{frame}{RGBM: Reduce Gradient Bubble Model}  
	\textbf{RGBM vs VPM:}
	\begin{itemize}
		\item Basé sur VPM
		\item Initiallement prévu pour les plongées répetitives
	\end{itemize}
\end{frame}

\begin{frame}{RGBM: Reduce Gradient Bubble Model}  
	\textbf{RGBM vs VPM:}
	\begin{itemize}
		\item VPM: entièrement basé sur le volume des bulles
		\item RGBM: volume des bulles pour les paliers profonds, gaz dissous pour la surface
	\end{itemize}
\end{frame}

\begin{frame}{RGBM: Reduce Gradient Bubble Model}  
	\textbf{RGBM vs VPM:}
	\begin{itemize}
		\item Reduit les gradients autorisés par 3 facteurs:
		\begin{itemize}
			\item Bulles conservées lors de plongées répetitives
			\item Profils inversés
			\item Présence de micro-noyaux
		\end{itemize}
	\end{itemize}
\end{frame}

\begin{frame}{RGBM: Reduce Gradient Bubble Model}  
	\textbf{RGBM vs VPM:}
	\begin{itemize}
		\item Prend en compte la "fenêtre oxygène"
		\item Différence de pression partielle d'oxygène entre le sang artériel et les tissus
	\end{itemize}
\end{frame}