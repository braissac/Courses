%!TEX root = decoproc_fr.tex

\section{Introduction}

%---------------------------------------------------------------------
%										Historique
%---------------------------------------------------------------------
\subsection{Plongée à décompression}

\begin{frame}{Plongée à décompression}
	Sous l'eau:
	\begin{itemize}
		\item Augmentation de la pression ambiante
		\item Le corps absorbe les gaz inertes (azote principalement)\\
				Passage du système respiratoire au système sanguin
		\item Quantité absorbée proportionnelle à la durée d'exposition
	\end{itemize}
	\vfill
	À la remontée
	\begin{itemize}
		\item Diminution de la pression ambiante
		\item Les gaz intertes s'échappent\\
				Passage du système sanguin au système respiratoire
	\end{itemize}
\end{frame}

\begin{frame}{Plongée à décompression}  
	Accident de décompression
	\begin{itemize}
		\item Quantitée de gaz intertes accumulée importante
		\item Remontée trop rapide
		\item Gaz inertes passent de la forme solluble à la forme gazeuse (formation de bulles)
	\end{itemize}
\end{frame}

\begin{frame}{Plongée à décompression}  
	Limite de non-décompression
	\begin{itemize}
		\item Faible quantité de gaz intertes absorbée
		\item Possibilité de remonter jusqu'à la surface sans palier
		\item Palier de sécurité pour diminuer le risque
	\end{itemize}
	\vfill
	Plongée à décompression
	\begin{itemize}
		\item Importante quantité de gaz intertes absorbée
		\item Impossibilité de remonter jusqu'à la surface sans palier
		\item Paliers de décompression nécessaires afin de permettre aux gaz inertes de repasser dans le système sanguin
		\item Création d'un "plafond virtuel"
	\end{itemize}
\end{frame}

\begin{frame}{Plongée à décompression}  
	\begin{itemize}
		\item Plongées plus longues en profondeurs
		\item Exploration (épaves, récifs, grottes, ...)
		\item Recherche scientifiques
		\item Recherche et récupération
		\item Travaux commerciaux
		\item ...
	\end{itemize}
\end{frame}

%---------------------------------------------------------------------
%										Historique
%---------------------------------------------------------------------
\subsection{Historique}

\begin{frame}{Historique}
	\textbf{Robert Boyle (1627 - 1691)}
	\begin{itemize}
		\item Physicien et chimiste Irlandais
		\item Série de tests avec un caisson basse pression
		\item Observation sur un serpent
		\item Découverte des maladies de décompression
	\end{itemize}
\end{frame}

\begin{frame}{Historique}  
	\textbf{Maladie des caissons (1841 - )}
	\begin{itemize}
		\item Chantiers pour la construction de ponts et de tunnels
		\item Travailleurs dans un caisson à une profondeur de 20m pendant plusieurs heures
		\item Douleurs, vertiges et essoufflement de retour à la surface
		\item Symptômes disparaissent lors d'un retour en profondeurs
	\end{itemize}
\end{frame}

\begin{frame}{Historique}  
	\textbf{Pont de Brooklyn (1870 - 1883)}
	\begin{itemize}
		\item Utilisation de caissons pour la contructions des piliers
		\item Plusieurs travailleurs dans l'incapacité de se tenir droit
		\item Naissance du terme "bends" (plier)
	\end{itemize}
\end{frame}

\begin{frame}{Historique}  
	\textbf{Paul Bert (1833 - 1886)}
	\begin{itemize}
		\item Etudie et enseigne la physiologie en France
		\item Auteur de \textit{La Pression Barometrique} (1878)
		\item Découvre:
		\begin{itemize}
			\item La toxicité de l'oxygène (Effet Paul Bert)
			\item La réduction de pression forme des bulles dans les tissus
			\item Les bulles sont composées d'azote
		\end{itemize}
		\item Introduit les paliers profonds (stop à mi-chemin entre la profondeur maximum et la surface)
		\item Observe les avantages de la recompression et de l'administration d'oxygène pure
	\end{itemize}
\end{frame}

\begin{frame}{Historique}  
	\textbf{John Scott Haldane (1860 - 1936)}
	\begin{itemize}
		\item Physiologiste Ecossais
		\item Travaille avec la marine Royale
		\item Effort pour réduire les accidents de décompression
		\item Chaques tissus absorbent et relâchent les gaz à un rythme différent
		\item Introduction des compartiments tissulaires, des périodes et des pressions maximales de sursaturation
		\item Première tables de plongée publiées dans le "Journal de l'Hygiène" en 1908
	\end{itemize}
\end{frame}

\begin{frame}{Historique}
	\textbf{US Navy (1912 - )}
	\begin{itemize}
		\item Programme pour tester les tables de Haldane à plus de 18m
		\item Publication des premières tables en 1915
		\item Révision en 1937 et en 1956 pour les plongées longues et profondes
		\item Publication du premier manuel de plongée de l'US NAVY en 1924
		\item Réduction de plus de 60\% des accidents de décompression entre 1937 et 1956
	\end{itemize}
\end{frame}

\begin{frame}{Historique}  
	\textbf{Robert Workmann (1921 - )}
	\begin{itemize}
		\item Table de l'US NAVY inappropriées pour des plongées longues et profondes
		\item Modification du modèle de base
		\item Niveau de sursaturation maximale varie avec la profondeur
		\item Introduction du terme "M-Value"
		\item Résultats présents sous la forme d'une équation permettant de l'utiliser à toutes profondeurs
	\end{itemize}
\end{frame}

\begin{frame}{Historique}  
	\textbf{Albert Bühlmann (1923 - 1994)}
	\begin{itemize}
		\item Docteur en Médecine à l'Université de Zurich, Suisse
		\item Travaille avec Hannes Keller pour développer des tables multi-gaz
		\item Algorithme de décompression comportant 16 compartiments
		\item Plongée à 120m en 1959, et 300m en 1962!
		\item Modification du modèle pour la plongée en altitude
		\item Publication de l'algorithme ZHL-16 (Zurich Limits) toujours utilisé aujourd'hui
	\end{itemize}
\end{frame}