\documentclass[aspectratio=1610,english,14pt]{beamer}

\usepackage{../../shared/tex/banstyle}

%---------------------------------------------------------------------
%										DOCUMENT
%---------------------------------------------------------------------
\begin{document}


\begin{frame}{Scuba Diving Equipment}
	\mypict{../img/scaph}
\end{frame}

%---------------------------------------------------------------------
%										TANKS
%---------------------------------------------------------------------
\section{Tanks}	

\begin{frame}{Tanks}
	\mypict{../img/tanks}
\end{frame}

\begin{frame}{Tanks}
	\begin{outline}
		\1 Steel
			\2 Thinner walls, less buoyant, more corrosion
		\1 Aluminium
			\2 Thicker walls, more buoyant, less corrosion
		\1 Size
			\2 Ban's (S80) = 11.1L aluminum tank
	\end{outline}

	\refbloc{PADI Encyclopedia 3-49}
\end{frame}
%---------------------------------------------------------------------
\begin{frame}{Tanks}
	\begin{outline}
		\1 Markings
			\2 Serial number
			\2 Steel or aluminium alloy
			\2 Working and test pressures
			\2 Manufacturer
			\2 Hydrostatic inspection date
				\3 "+" allows overfilled by 10\%
	\end{outline}

	\refbloc{PADI Encyclopedia 3-52}
\end{frame}

%---------------------------------------------------------------------
%										Cylinders Inspection
%---------------------------------------------------------------------
\section{Cylinders Inspection}

\begin{frame}{Cylinders Inspection}
	\mypict{../img/inspection}
\end{frame}

\begin{frame}{Visual Inspection}
	\begin{outline}
		\1 Why?
			\2	Industry Standard
			\2 Check tanks between hydrostatic inspections
			\2 Avoid excessive corrosion around the valve
			\2 Usually once a year (but national standards may vary)
	\end{outline}
\end{frame}
%---------------------------------------------------------------------
\begin{frame}{Visual Inspection}
	\begin{outline}
		\1 How?
			\2 Remove valve
			\2 Check inside with bright light
			\2 Inspect outside for unusual impacts or marks
	\end{outline}
\end{frame}
%---------------------------------------------------------------------
\begin{frame}{Hydrostatic Inspection}
	\begin{outline}
		\1 Why?
			\2 Every few years (follow National standards)
			\2 Exposed to high temperature (>82$^\circ$C)
			\2 Damaged due to impact
			\2 After tumbling due to internal corrosion
			\2 Empty for 2 years or more
	\end{outline}
\end{frame}
%---------------------------------------------------------------------
\begin{frame}{Hydrostatic Inspection}
	\begin{outline}
		\1 How?
			\2 Fill tank with water
			\2 Immerged in water chamber
			\2 Pressurize above working pressure ($\approx 5/3$)
			\2 Mesure volume displacement under pressure
			\3 (metal fatigue)
			\2 Check volume displacement after the test
			\3 (metal elesticity)
	\end{outline}
\end{frame}

%---------------------------------------------------------------------
%										VALVES
%---------------------------------------------------------------------
\section{Valves}

\begin{frame}{Valves}
	\mypict{../img/valve}
\end{frame}

\begin{frame}{Valves}
	\begin{columns}[onlytextwidth]
		\column{0.25\linewidth}
			\mypict{../img/k-valve}
		\column{0.75\linewidth}
			\begin{outline}
				\1 K-Valve
					\2 Most common valve
					\2 Simple ON/OFF valve
					\2 Burst disk between 125\% and 166\% of the working pressure
			\end{outline}
	\end{columns}

	\vfill
	
	\begin{columns}[onlytextwidth]
		\column{0.25\linewidth}
			\mypict{../img/j-valve}
		\column{0.75\linewidth}
			\begin{outline}
				\1 J-Valve
					\2 Lever used as a reserve
					\2 Spring close the valve at around 20-40 bar
					\2 Lever must be open when filling (lower position)
			\end{outline}
	\end{columns}
\end{frame}
%---------------------------------------------------------------------
\begin{frame}{Valves}
	\begin{columns}[onlytextwidth]
		\column{0.25\linewidth}
			\mypict{../img/din}
		\column{0.75\linewidth}
			\begin{outline}
				\1 DIN
					\2 Regulator screws inside the valve
					\2 Stronger, used for overhead diving
					\2 Can be used up to 300 bar
			\end{outline}
	\end{columns}

	\vfill

	\begin{columns}[onlytextwidth]
		\column{0.25\linewidth}
			\mypict{../img/yoke}
		\column{0.75\linewidth}
			\begin{outline}
				\1 Yoke, A-Clamp, Int
					\2 Screw holding the regulator against the valve
					\2 o-ring on the valve
					\2 Can be used up to 232 bar
			\end{outline}
	\end{columns}
\end{frame}

%---------------------------------------------------------------------
%										REGULATORS
%---------------------------------------------------------------------
\section{Regulators}

\begin{frame}{Regulators}
	\mypict{../img/regs}
\end{frame}
%---------------------------------------------------------------------
\begin{frame}{Definitions}
	\begin{outline}
		\1 Upstream
			\2 Open against the air flow
		\1 Downstream
			\2 Open with the air flow
		\1 Demand valve
			\2 Air is given only upon inhalation
		\1 Fail safe design
			\2 Downstream design
			\2 Will freeflow if the regulator freeze
	\end{outline}
\end{frame}
%---------------------------------------------------------------------
\begin{frame}{First Stages}
	\begin{outline}
		\1 Types
			\2 Unbalanced Piston
			\2 Balanced Piston
			\2 Balanced Membrane
	\end{outline}

	\refbloc{PADI Encyclopedia 3-60 and 3-62}
\end{frame}
%---------------------------------------------------------------------
\begin{frame}{First Stages}
	\begin{outline}
		\1 Aims
			\2 Reduce High Pressure to Intermediate Pressure
			\2 Fail safe design
			\2 Balanced design:
				\3 Same air flow and IP throughout the dive
				\3 IP doesn't change with tank pressure
				\3 Air flow stable with 2 divers
			\2 Environmental seal
				\3 Prevent regulator from freezing
				\3 Avoid freeflow in cold water
	\end{outline}
\end{frame}
%---------------------------------------------------------------------
\begin{frame}{Second Stages}
	\begin{outline}
		\1 Types
			\2 Unbalanced Upstream (obsolete)
			\2 Unbalanced Downstream
			\2 Balanced Downstream
			\2 Servo or pilot valve
	\end{outline}

	\refbloc{PADI Encyclopedia 3-61}
\end{frame}
%---------------------------------------------------------------------
\begin{frame}{Second Stages}
	\begin{outline}
		\1 Aims
			\2 Reduce Intermediate Pressure to Ambiant Pressure
			\2 Not always a fail safe design!
			\2 Balanced design:
				\3 Same inspiration effort throughout the dive
				\3 Effort doesn't change with IP or depth
		\1 Principle:
			\2 Classic: On inhalation, a diaphragm flexes and open a valve
			\2 Servo: the diaphragm opens a pilot valve which opens the main valve
	\end{outline}
\end{frame}

%---------------------------------------------------------------------
%										GAUGES
%---------------------------------------------------------------------
\section{Gauges}

\begin{frame}{Gauges}
	\mypict{../img/gauges}
\end{frame}
%---------------------------------------------------------------------
\begin{frame}{Gauges}
	\begin{outline}
		\1 Submersible Pressure Gauges (SPG)
			\2 Open Bourdon tube
			\2 Spiral which expands under pressure
		\1 Depth gauge
			\2 Capillary gauge (water moving in a transparant tube)
			\2 Open Bourdon tube
			\2 Oil-filled Bourdon tube
			\2 Diaphragm
		\1 Computer
			\2 Transducer converting the pressure in electrical current
	\end{outline}
\end{frame}

%---------------------------------------------------------------------
%										NITROX
%---------------------------------------------------------------------
\section{Nitrox}

\begin{frame}{Nitrox}
	\mypict{../img/nitrox}
\end{frame}
%---------------------------------------------------------------------
\begin{frame}{Nitrox}
	\begin{outline}
		\1 Equipment compatibility
			\2 Follow manufacturer guidelines
			\2 O2 clean equipment with >40\% oxygen
			\2 Follow national regulations
		\1 Procedures
			\2 Mix analysed by the diver
			\2 Content sticker on the tank
	\end{outline}
\end{frame}
%---------------------------------------------------------------------
\end{document}