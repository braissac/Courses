%!TEX root = intrototech_ppt_fr.tex

\section{La plongée technique?}

%------------------------------------------------------------------
\begin{frame}
	\begin{outline}
		\1 Chapitres abordés:
			\2 Terminologie
			\2 Plongée sportive vs technique
			\2 Histoire
			\2 En commun
			\2 Considérations personnelles
			\2 Révisions des questions
	\end{outline}
\end{frame}


%---------------------------------------------------------------------
%										Terminologie
%---------------------------------------------------------------------
\subsection{Terminologie}

%------------------------------------------------------------------
\begin{frame}{\insertsubsection}
	\begin{outline}
		\1 Il y trois catégories de plongée:
			\2 Plongée Recréative
			\2 Plongée Commerciale
			\2 Plongée Scientifique
	\end{outline}
\end{frame}

%------------------------------------------------------------------
\begin{frame}{\insertsubsection}
	\begin{outline}
		\1 Différences entre les catégories:
			\2 La plongée recréative est un hobby
			\2 Les autres sont des activités professionnelles
	\end{outline}
\end{frame}

%------------------------------------------------------------------
\begin{frame}{\insertsubsection}
	\begin{outline}
		\1 Deux types de plongée recréative:
			\2 Plongée sportive
			\2 Plongée technique
	\end{outline}
\end{frame}


%---------------------------------------------------------------------
%								Sportive vs Technique
%---------------------------------------------------------------------
\subsection{Sportive vs Technique}

%------------------------------------------------------------------
\begin{frame}{\insertsubsection}
	\begin{outline}
		\1 Plongée Sportive
			\2 Air ou Nitrox (max 40\% d'oxygène)
		\1 Plongée Technique
			\2 Elargit les options 
	\end{outline}
\end{frame}

%------------------------------------------------------------------
\begin{frame}{\insertsubsection}
	\begin{outline}
		\1 Plongée Sportive
			\2 Profondeurs limitées
		\1 Plongée Technique
			\2 Peut augmenter considérablement les profondeurs
	\end{outline}
\end{frame}

%------------------------------------------------------------------
\begin{frame}{\insertsubsection}
	\begin{outline}
		\1 Plongée Sportive
			\2 Pas d'envionnement sous plafond
			\2 Plongées dans la limite de non-décompression
		\1 Plongée Technique
			\2 Peut inclure des pénetrations en épave ou en grottes
			\2 Peut inclure des plongées à décompression
	\end{outline}
\end{frame}

%------------------------------------------------------------------
\begin{frame}{\insertsubsection}
	\begin{outline}
		\1 Plongée Sportive
			\2 Approche désinvolte, laxiste, peut être tolérée
		\1 Plongée Technique
			\2 Plannification et exécution précises requises
	\end{outline}
\end{frame}

%------------------------------------------------------------------
\begin{frame}{\insertsubsection}
	\begin{center}
		\Large
	\end{center}
\end{frame}

%------------------------------------------------------------------
\begin{frame}{\insertsubsection}
	\begin{outline}
		\1 Composants critiques d'un processus planification
			\2 Itinéraire et objectifs clairement définis
			\2 Profondeur, temps et paliers de décompression
			\2 Sélection des mélanges gazeux et gestion des gaz
			\2 Planifications d'urgence
	\end{outline}
\end{frame}

%------------------------------------------------------------------
\begin{frame}{\insertsubsection}
	\begin{outline}
		\1  Un plongeur technique doit être préparé à gérer efficacement n'importe quelle urgence raisonnablement prévisible
			\2 Equipement approprié
			\2 Compétances personnelles bien rodées
			\2 Etat d'esprit adapté 
	\end{outline}
\end{frame}


%---------------------------------------------------------------------
%									Histoire
%---------------------------------------------------------------------
\subsection{Histoire}

%------------------------------------------------------------------
\begin{frame}{\insertsubsection}
	\begin{outline}
		\1 Origine de la plongée en scaphandre autonome
			\2 Inventée en 1943 par Cousteau \& Gagnan
			\2 Déployée par les forces Alliées pendant la Seconde Guerre Mondiale
			\2 Raportée à la maison par les vétérans
	\end{outline}
\end{frame}

%------------------------------------------------------------------
\begin{frame}{\insertsubsection}
	\begin{outline}
		\1 Premiers jours de la plongée civile
			\2 Les instructeurs étaient principalement des anciens militaires
			\2 Entrainement suivant les régimes militaires
				\3 Long et détaillé
				\3 Conditions physiques exigentes
	\end{outline}
\end{frame}

%------------------------------------------------------------------
\begin{frame}{\insertsubsection}
	\begin{outline}
		\1 Naissances des agences de plongée recréative
			\2 Los Angeles fût le premier état en 1954
			\2 Plusieurs suivirent des les années 60
			\2 Objectif principal: formation standardisée, reconnue par une organisation
	\end{outline}
\end{frame}

%------------------------------------------------------------------
\begin{frame}{\insertsubsection}
	\begin{outline}
		\1 L'évolution de la plongée sportive
			\2 Objectifs
				\3 Encourager une plus grande participation 
				\3 Rendre la plongée plus simple
				\3 Garder la plongée aussi sûre que possible
			\2 Methodologie
				\3 Organiser les premiers niveaux de formation
				\3 Limiter les domaines d'activité
	\end{outline}
\end{frame}

%------------------------------------------------------------------
\begin{frame}{\insertsubsection}
	\begin{outline}
		\1 L'émergence de la plongée technique
			\2 Certains se sont sentis inutilement limités par les limites de la plongée sportive
			\2 Recherches ailleurs pour des outils et des techniques permettant d'élargir leurs activités personnelles
				\3 Retour aux protocols militaires 
				\3 Regarde les développemnts plus récents de la plongée commerciale ou scientifique
	\end{outline}
\end{frame}

%------------------------------------------------------------------
\begin{frame}{\insertsubsection}
	\begin{outline}
		\1 Naissance des agences de plongée technique
			\2 Premier centre d'attention sur le nitrox, en 1985
			\2 Puis continu de s'étendre dans d'autre domaine
	\end{outline}
\end{frame}

%------------------------------------------------------------------
\begin{frame}{\insertsubsection}
	\begin{outline}
		\1 TDI: Technical Diving International
			\2 La plus grande et plus innovante agence internationnale
			\2 Racine dans les pionniers des la plongée au nitrox
			\2 Programme actuel comprend:
				\3 Nitrox et Nitrox avancé
				\3 Procédure de décompression et Plongée étendue
				\3 Trimix et Trimix avancé
				\3 Recycleur à circuit fermé et semi-fermé
				\3 Plongée sur épave avancé
				\3 Plongée en caverne et grotte
	\end{outline}
\end{frame}


%---------------------------------------------------------------------
%									Histoire
%---------------------------------------------------------------------
\subsection{Points communs en Plongée Recréative}

%------------------------------------------------------------------
\begin{frame}{\insertsubsection}
	\begin{outline}
		\1 La plongée est la plongée, sportive ou technique
			\2 Même principes de physique et physiologie
			\2 Majeure partie de l'équipement a la même utilistion
			\2 Tous les plongeurs commencent avec l'Open Water
	\end{outline}
\end{frame}

%------------------------------------------------------------------
\begin{frame}{\insertsubsection}
	\begin{outline}
		\1 Relation entre la plongée sportive et technique
			\2 Certains programmes sportifs bâtissent les fondations pour des formations techniques avancées
				\3 Exemple: La spécialité Epave est un pré-requis naturel pour le cours d'Epave Avancé
			\2 D'autre cours sportifs fournissent des compétances directement transférables à la plongée technique
				\3 Exemples: Combinaison étanche, Navigation sous-marine,...
	\end{outline}
\end{frame}

%------------------------------------------------------------------
\begin{frame}{\insertsubsection}
	\begin{outline}
		\1 Pour la plongée sportive et technique
			\2 Programme progressif et formation continue
			\2 Variété de programme pour accomoder les désirs de chacun
	\end{outline}
\end{frame}


%---------------------------------------------------------------------
%									Histoire
%---------------------------------------------------------------------
\subsection{Considérations Personnelles}

%------------------------------------------------------------------
\begin{frame}{\insertsubsection}
	\begin{outline}
		\1 Est-ce que la plongée technique est pour vous?
			\2 Conditions préalables s'appliquent
			\2 Instructeur pour offrir des conseils
			\2 Demande un investissement en formation et en équipement
			\2 Développe vos connaissances
			\2 Rends un plongeur plus confortable, capable et confiant
			\2 Vous pouvez procéder étape par étape
			\2 Vous n'êtes pas obligés d'abandonner la plongée sportive!
	\end{outline}
\end{frame}


%---------------------------------------------------------------------
%									Histoire
%---------------------------------------------------------------------
\subsection{Révisions des questions}

%------------------------------------------------------------------
\begin{frame}{\insertsubsection}
	\begin{outline}
		\1 La plongée technique peut inclure les activités suivantes:
			\2 Plongée à décompression
			\2 Changement de mélange gazeux
			\2 Pénetration dans les recoins sombres d'une épave ou d'une grotte
	\end{outline}
\end{frame}
%------------------------------------------------------------------
\begin{frame}{\insertsubsection}
	\begin{outline}
		\1 La plongée technique ne nécessite généralement pas d'équipement supplémentaire par rapport à la plongée récréative.
			\2 Faux
	\end{outline}
\end{frame}
%------------------------------------------------------------------
\begin{frame}{\insertsubsection}
	\begin{outline}
		\1 La plongée technique ne nécessite généralement pas de formation spécialisée au-delà de celle prévue en plongée récréative.
			\2 Faux
	\end{outline}
\end{frame}
%------------------------------------------------------------------
\begin{frame}{\insertsubsection}
	\begin{outline}
		\1 La planification d'une plongée technique doit inclure les éléments suivants:
			\2 Itinéraire et objectifs clairement définis
			\2 Profondeur, temps et paliers de décompression
			\2 Sélection des mélanges gazeux et gestion des gaz
			\2 Planifications d'urgence
	\end{outline}
\end{frame}
%------------------------------------------------------------------
\begin{frame}{\insertsubsection}
	\begin{outline}
		\1 Selon le manuel de plongée de la NOAA, "Il n'a pas de plongée \st prise à la légère"
			\2 A décompression
	\end{outline}
\end{frame}
%------------------------------------------------------------------
\begin{frame}{\insertsubsection}
	\begin{outline}
		\1 La plongée technique peut inclure les activités suivantes:
			\2 Plongée à décompression
			\2 Changement de mélange gazeux
			\2 Pénetration dans les recoins sombres d'une épave ou d'une grotte
	\end{outline}
\end{frame}