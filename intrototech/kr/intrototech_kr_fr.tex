\documentclass[english,10pt,a4paper]{article}

\usepackage{../../shared/tex/course_tdi}

\title{TDI Intro To Tech}
\subtitle{Révision de connaissances}

\author{TECHNICAL DIVING INTERNATIONAL}
\website{www.tdisdi.com}

%---------------------------------------------------------------------
%										DOCUMENT
%---------------------------------------------------------------------
\begin{document}
\begin{titlepage}
	\begin{center}
		\includegraphics[width=13cm]{\tdilogo}\\
		\vspace{1cm}
		{\fontsize{40}{48}\selectfont \textbf{\thetitle}}\\
		\vspace{1cm}
		{\fontsize{30}{36}\selectfont \textbf{\thesubtitle}}\\
		\vspace{4cm}
		{\fontsize{18}{22}\selectfont \textbf{\theauthor}}\\
		\vspace{0.2cm}
		{\fontsize{18}{22}\selectfont \textbf{\thewebsite}}\\
	\end{center}
\end{titlepage}

%---------------------------------------------------------------------
%									Fundamentals
%---------------------------------------------------------------------
	\section{Principe de Pression}

	\begin{outline}
		\1 La plongée technique peut inclure les activités suivantes:
			\2 Plongée à décompression
			\2 Changement de mélange gazeux
			\2 Pénetration dans les recoins sombres d'une épave ou d'une grotte
	
		\1 La plongée technique ne nécessite généralement pas d'équipement supplémentaire par rapport à la plongée récréative. \vf

		\1 La plongée technique ne nécessite généralement pas de formation spécialisée au-delà de celle prévue en plongée récréative. \vf

		\1 La planification d'une plongée technique doit inclure les éléments suivants:
			\2 Itinéraire et objectifs clairement définis
			\2 Profondeur, temps et paliers de décompression
			\2 Sélection des mélanges gazeux et gestion des gaz
			\2 Planifications d'urgence

		\1 Selon le manuel de plongée de la NOAA, "Il n'a pas de plongée \st prise à la légère"
		%	"There is no such thing as casual decompression dive"
	\end{outline}
	\pagebreak

%---------------------------------------------------------------------
%									Equipment
%---------------------------------------------------------------------
	\section{Equipement de Plongée Technique}

	\begin{outline}
		\1 Quel(s) principe(s) s'applique(nt) lors de la sélection et de la configuration de l'équipement pour les plongeurs techniques
			\2 Rédondance des élements critiques
			\2 Hydrodynamisme
			\2 Eliminations des éléments superflus
	
		\1 Le moyen les plus efficace d'utiliser deux bouteilles est des les joindre ensembles à l'aide d'un \st
		\vspace{1cm}

		\1 Afin de conserver la redondance des premiers étages lors de l'utilisation d'un mono-bouteille, la bouteille doit être équipée d'une robinetterie \st
		\vspace{1cm}

		\1 Lors de l'utilisation d'une bouteille additionelle, celle-ci est généralement attachée à l'avant du BCD. \vf

		\1 Le type de robinetterie préconisée pour la plongée technique est de type:
			\2 DIN
			\2 Etrier / Yoke

		\1 Il n'y a pas de bénéfice à utiliser un détendeur muni d'un long flexible hormis en plongée en grotte. \vf

		\1 Parmis les élements suivants, lesquels peuvent être utlisés en tant que système de contrôle de la flottabilité redondant?
			\2 BCD avec une double vessie
			\2 Cominaison étanche
			\2 Parachute de relevage

		\1 Une plongeur technique devrait être équipé d'au minimum \st profondimètre(s) et \st dispositif(s) de chronométrage.
		\vspace{1cm}

		\1 Le froid peut être un élement influançant la sensiblité d'un plongeur à un accident de décompression. \vf

		\1 Un lestage correct doit être basé sur les caractéristiques de flottabilité d'un plongeur \st
			\2 au début de la plongée
			\2 à la fin de la plongée

		\1 Une plongeur technique devrait être équipé d'au minimum \st couteau(x).
		\vspace{1cm}

		\1 Un parachute de relevage et un dévidoire font partie de l'équipement standard d'un plongeur technique. \vf

		\1 Chaque plongeur devrait être équipé d'un équipement de signalisation de surface \st et d'un \st. % visible & audible
		\vspace{1cm}

		\1 Un masque de secours n'offre aucun bénéfice pratique à un plongeur technique. \vf

		\1 Quels types de lampes peuvent être utiles à un plongeur techniques?
			\2 Lampe à batterie déportée (canister)
			\2 Gros phare sous-marin
			\2 Petite lampe de poche
			\2 Flash sous-marin (strobe)
	\end{outline}
	\pagebreak

%---------------------------------------------------------------------
%									Physique et Lois des gas
%---------------------------------------------------------------------
	\section{Gaz et gestion d'air}
	\begin{outline}
		\1 La pression ambiante à la surface est de \st bar / atm.
		\vspace{1cm}

		\1 La pression ambiante augmente de 1 bar / atm tous les \st mètres d'eau de mer.
		\vspace{1cm}

		\1 Quand la pression ambiante augmente, le volume de gaz dans un récipient flexible:
			\2 Augmente
			\2 Diminue

		\1 Durant une plongée, la pression totale dans les poumons d'un plongeur est égale à la pression ambiante. \vf

		\1 L'air contient \st d'oxygène est \st d'azote.
		\vspace{1cm}

		\1 Lors de l'utilisation d'air en plongée récréative (maximum 40m), la principale préoccupation d'un plongeur est son exposition élevée à quel gaz?
			\2 Oxygène
			\2 Azote

		\1 Le Nitrox contient un pourcentage plus élevé d'\st que celui trouvé dans l'air, et un pourcentage plus faible d'\st.
		\vspace{1cm}

		\1 Le Nitrox fourni au plongeur un moyen efficace de prolonger sa limite de non-décompression à une profondeur donnée. \vf

		\1 Le Nitrox introduit une nouvelle préoccupation pour le plongeur, qui est le risque de toxicité à \st du système nerveux central (CNS).
		\vspace{1cm}

		\1 Il y a un risque inhérent de \st ou d'\st en utilisant de l'oxygène pure à 100\%, ce qui requière des techniques et du matériel adaptés.
		\vspace{1cm}

		\1 Les détendeurs ne nécessitent généralement pas de préparation ou de nettoyage particuliers avant d'étre utilisés avec quel mélange de nitrox?
			\2 EAN30
			\2 EAN40
			\2 EAN50
			\2 EAN60

		\1	Pour une plongée d'une durée raisonnable, au-delà des profondeurs traditionnelles de la plongée récréative, l'utilisation de bouteilles de décompression devient une nécessité. \vf

		\1 La narcose n'est pas néfaste par elle-même mais elle peut considérablement affecter le jugement et les réactions d'un plongeur. \vf

		\1 En Trimix, quel gaz est ajouté au mélange afin de reduire la narcose?
			\2 Oxygène
			\2 Azote
			\2 Helium
			\2 Argon

		\1 Connaitre sa propre consommation d'air (SAC) est un élement primordial d'une gestion de gaz efficace. \vf 
	\end{outline}
	\pagebreak
	
%---------------------------------------------------------------------
%									Physique et Lois des gas
%---------------------------------------------------------------------
	\section{Activités Tech}

	\begin{outline}
		\1 Les objectifs des la plongée sont toujours prioritaires sur le plan de plongée. \vf

		\1 Rien n'est plus important que la sureté des plongeurs. \vf

		\1 Si quelques chose parait incorrect, il \st incorrect, et la planification d'urgence adéquate doit être appliquée. \vf

		\1 Qui peut prendre la décision de terminer une plongée?
			\2 N'importe quel plongeur
			\2 Uniquement le plongeur en tête du groupe
			\2 Uniquement le plonguer le plus expérimenté

		\1 Quand les tables de plongée indiquent un palier de décompression à 3 mètres, ce palier peut être effectué n'importe où entre 3 et 6 mètres sans influencer la décompression. \vf

		\1 Sur son ardoise, un plongeur technique écrit la planification pour la profondeur et le temps planifiés, ainsi que la planification pour un profondeur \st et un temps \st.
		\vspace{1cm}

		\1 Quel environnement est le plus enclin à changer lors d'une brève période de temps.
			\2 Les grottes
			\2 Les épaves

		\1 Dans les épaves et les grottes, le sol est généralement recouvert de limon qui peut rapidement réduire la visibilité s'il est agité par un plongeur ou par le remous des palmes. \vf

		\1 Le plus grand danger en plongée sur épave ou en grotte est le risque de \st et de se retrouver piégé à l'intérieur en raison de la désorentation.
		\vspace{1cm}

		\1 Les plongeurs sur épave et en grotte utilisent un \st afin de déployer une référence visuelle et tactile.
		\vspace{1cm}

		\1 Un recycleur est un appareil à circuit \st ou semi-\st qui permet de recylcer les gaz expirés.
		\vspace{1cm}

		\1 Le gaz expiré par une plongeur contient du \st, un sous-produit du métabolisme, qui est absorbé dans le recycleur en passant à travers un produit absorbant.
		\vspace{1cm}

		\1 Quel type de recycleur maintient une pression partielle d'oxygène constante en profondeur?
			\2 SCR (semi-fermé)
			\2 CCR (fermé)

		\1 La plongée en solitaire est toujours appropriée, dans toutes les activités techniques.\vf

		\1 Un plongeur technique doit être préparé, en terme d'équipement, de performance personnelle et d'état d'esprit, à gérer efficacement n'importe quelle \st raisonnablement prévisible qui pourrait arriver durant la plongée.
		\vspace{1cm}
	\end{outline}
	\sectionpage


\end{document} 
