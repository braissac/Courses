%!TEX program = <pdflatex> 

\documentclass[aspectratio=1610,english,12pt]{beamer}

\usepackage{../../shared/tex/mystyle}

% Image
\newcommand{\vipict}[1]{%
	\hfill
	\begin{center}
		\includegraphics[width=0.9\linewidth,height=0.7\textheight,keepaspectratio]{#1}
	\end{center}
	\hfill 
}

% Description
\defbeamertemplate{description item}{align left}{\insertdescriptionitem\hfill}
\setbeamertemplate{description item}[align left]

\author[]{Chris Braissant}
\title[]{TDI Visual Inspection Procedures}
\institute{Ban's Technical Diving }
\date{\today}	
	
%---------------------------------------------------------------------
%										DOCUMENT
%---------------------------------------------------------------------
\begin{document}

\begin{frame}[plain]
	\maketitle
\end{frame}

%---------------------------------------------------------------------
%										Introduction
%---------------------------------------------------------------------

\section{Introduction}

\subsection{Intro, Paperwork, Schedule}

\begin{frame}{Course overview}
	Promote safety in the care and maintenance of high-pressure scuba cylinders.\\~\\
	Train in proper handling, filling and inspection techniques.\\~\\
	Identification of various defective conditions.\\
\end{frame}

\begin{frame}{Paperwork}
	Liability Release and Express Assumption of Risk\\~\\
	Medical Statement
\end{frame}

%---------------------------------------------------------------------
%										Safe handling of cylinders
%---------------------------------------------------------------------
\section{Safe handling of cylinders}

\subsection{Testing Standards}

\begin{frame}{\insertsubsection}
	USA:
	\begin{itemize}
		\item DOT: Department of Transport
		\item CGA: Compressed Gas Association
	\end{itemize}

	Canada:
	\begin{itemize}
		\item TC: Transport Canada
	\end{itemize}

	Europe:
	\begin{itemize}
		\item BS: British Standards
		\item EN: European Nation
	\end{itemize}

	Thailand:
	\begin{itemize}
		\item TIS: Thai Industrial Standards
	\end{itemize}
\end{frame}

\begin{frame}{DOT: Department of Transport}
	Steel cylinders:
	\begin{itemize}
		\item CGA C-6\\Standard for Visual Inspection of Steel Compressed Gas Cylinders
	\end{itemize}
	Aluminium cylinders:
	\begin{itemize}
		\item CGA C-6.1\\Standard for Visual Inspection of High Pressure Aluminum Alloy Compressed Gas Cylinders
	\end{itemize}
	Composite cylinders:
	\begin{itemize}
		\item CGA C-6.2\\Standard for Visual Inspection and Requalification of Fiber Reinforced High Pressure Cylinders
	\end{itemize}
\end{frame}

\begin{frame}{Europe}
	Steel cylinders:
	\begin{itemize}
		\item BS EN 1968\\Transportable gas cylinders. Periodic inspection and testing of seamless steel gas cylinders
	\end{itemize}
	Aluminium cylinders:	
	\begin{itemize}
		\item BS EN 1802\\Transportable gas cylinders. Periodic inspection and testing of seamless aluminium alloy gas cylinders
	\end{itemize}
	Composite cylinders:
	\begin{itemize}
		\item BS EN ISO 11623\\Transportable gas cylinders. Periodic inspection and testing of composite gas cylinders
	\end{itemize}
\end{frame}

\begin{frame}{Europe:}
	Summary of the standards:
	\begin{itemize}
		\item Hydrostatic test: 5 years
		\item Visual inspection: 2 1/2 years
		\item Stamping: YYYY/MM
		\item Content label: BS EN ISO 7225 (Air, Nitrox, Trimix -  Compressed Gas - Oxidizing)
		\item Color coding: EN1089-3 (For "Diver at work") 
	\end{itemize}
\end{frame}

\begin{frame}{Thailand}  
	All cylinders:
	\begin{itemize}
		\item TIS 358-2551\\Use and maintenance of compressed gas cylinders
	\end{itemize}
\end{frame}

% ------- Chapter review -------
\subsection{Chapter review}

\begin{frame}{Chapter 2: review}
	\begin{enumerate}
		\item \textit{What is the recommended fill rate for scuba cylinders?}\\
				300-600 psi/min or 20-40 bar/min\\~\\
		\item \textit{If you suspect that your customer put his own EOI sticker on his cylinder, what can you do about it?}\\
				In any doubt, tell the customer and inspect the cylinder\\~\\
		\item \textit{Can you stamp a 3AL over an SP6498 that the hydro people forgot to do?}\\
				No, it can only be done by an hydro facility.
	\end{enumerate}
\end{frame}

\begin{frame}{Chapter 2: review}
	\begin{enumerate}
		\setcounter{enumi}{3}
		\item \textit{Describe the appearance of an approved burst disc as well as two exemple of an unapproved one.}\\
				Approved: Vent holes in the side of the assembly.\\
				Unapproved: single hole in the end or no hole \\~\\ 
		\item \textit{An EOI sticker applied to a cylinder within the past year is ample evidence to authorize filling?}\\
				No. The sticker is in itself not legal justification. The operator is responsible for assuring that the cylinder is safe to fill.\\~\\
		\item \textit{As an inspector, what can you tell by the marking in the photograph below?}
				Not safe to fill.\\
				SP6498: Need to be stamped 3AL over\\
				9  $\Uparrow$ 74: Hydro over due. 
	\end{enumerate}
\end{frame}

%---------------------------------------------------------------------
%										Tools
%---------------------------------------------------------------------
\section{Tools}

\subsection{Valve}
\begin{frame}{Removing the handle}  
	Tank knob screwdriver\\
		\vipict{../img/knob_driver}
\end{frame}

\begin{frame}{Removing the valve}  
	Valve removing tool (DIN):\\
	\vipict{../img/din_removal} 
\end{frame}

\begin{frame}{Removing the valve}
	Adjustable Wrench:\\
	\vipict{../img/adj_wrench}
\end{frame}

\begin{frame}{Valve inspection}
	Various set of wrench, depending of the manufacturer:\\
	\vipict{../img/wrench}	
\end{frame}

\begin{frame}{Valve inspection}  
	Set of allen key (dip tube):\\
	\vipict{../img/allen}
\end{frame}

\begin{frame}{Valve inspection}  
	Go-NoGo gauge:\\
	\vipict{../img/valve_nogo}
\end{frame}

\subsection{Tank Inspection}
\begin{frame}{Tank Inspection}  
	Bright light:\\
	\vipict{../img/light}
\end{frame}

\begin{frame}{Tank Inspection}  
	Pit gauge:\\
	\vipict{../img/pit_probe}	
\end{frame}

\begin{frame}{Neck Inspection}  
	Dental mirror:\\
	\vipict{../img/dental_mirror}
\end{frame}

\begin{frame}{Neck Inspection}  
	Opti-viewer:\\
	\vipict{../img/opti_viewer}	
\end{frame}

\begin{frame}{Neck Inspection}  
	Go-NoGo gauge:\\
	\vipict{../img/valve_go}	
\end{frame}

\begin{frame}{Cleaning}  
	Brass or nylon brushes:\\
	\vipict{../img/brass_brush}	
\end{frame}


%---------------------------------------------------------------------
%										Procedures
%---------------------------------------------------------------------
\section{Procedures}
Identification
Preparation
External visual inspection
Internal visual inspection
Neck and threads
Pressure test
Inspection of valve
Final operation
Rejection



%---------------------------------------------------------------------
%										Visual inspection
%---------------------------------------------------------------------
\section{Visual inspection}

%---------------------------------------------------------------------
%										Cylinder Inspection
%---------------------------------------------------------------------
\section{Cylinder Inspection}

% ------- Exterior -------
\subsection{Exterior}
\begin{frame}{Bow}
	Slight curve in the sidewall of a cylinder. Not dangerous and does not affect cylinder performance. Commonly know as "banana shape"
	\vipict{../img/bow}
\end{frame}

\begin{frame}{Bulges}
	Visible swelling of the cylinder. Extremely rare and dangerous condition.
	\vipict{../img/buldge}
\end{frame}

\begin{frame}{Cuts and digs}
	Defect that are indicated by removed or upset metal caused by contact with a sharp object.
	\vipict{../img/cut}
\end{frame}

\begin{frame}{Dents}
	Deformation of the cylinder caused by a blunt object so that the metail is relocated and the wall thickness is not reduced.
	\vipict{../img/dent}
\end{frame}

\begin{frame}{Fire damage}
	Excessive general or localized heating of a cylinder.
	\vipict{../img/fire}
\end{frame}

% ------- Interior -------
\subsection{Interior}
\begin{frame}{Crack}
	Split or rift in the metal.
	\vipict{../img/crack}
\end{frame}

\begin{frame}{Fold and valley}  
	Sharp visual groove (fold) or shallow and smooth depression (valley), usually found in the crown area
	\vipict{../img/fold}
\end{frame}

\begin{frame}{General corrosion and rust}  
	Uniform loss of metal reducing the wall thickness
	\vipict{../img/corrosion}
\end{frame}

\begin{frame}{Corrosion pits}  
	Corrosion forming isolated craters, without significant alignment
	\vipict{../img/pit}
\end{frame}

% ------- Chapter review -------
\subsection{Chapter review}

\begin{frame}{Chapter 4: review}
	\begin{enumerate}
		\item \textit{What is corrsion in an aluminium cylinder and how can you identify it?}\\
				Oxydation of the metal or damage caused by galvanic corrosion.\\
				Usually appears as a white crust or dust.\\~\\
		\item \textit{What is corrosion in a steel cylinder called?}\\
				Rust\\~\\
		\item \textit{What causes galvanic corrosion?}\\
				Electrochemical process in which one metal corrodes preferentially to another when both metals are in electrical contact, in the presence of an electrolyte (salt water).
	\end{enumerate}
\end{frame}

\begin{frame}{Chapter 4: review}
	\begin{enumerate}
		\setcounter{enumi}{3}
		\item \textit{What causes a tool stop mark and why does it causes us concern in the visual test procedure?}\\
				The stop mark is caused by a tool stopping in its process, and starting again (threading tap). It can easily be mistaken for a crack.\\~\\ 
		\item \textit{Explain the difference between broadspread and line corrosion}\\
				Broadspread: General corrosion over a wide area.\\
				Line corrosion: String of pits arranged in somewhat of a line.
	\end{enumerate}
\end{frame}

%---------------------------------------------------------------------
%										Rejection and condemnation
%---------------------------------------------------------------------
\section{Rejection and condemnation}

\begin{frame}{Criteras}
	Depend on:
	\begin{itemize}
		\item Type of cylinder (alu, steel, composite)
		\item Law in the country
		\item Manufacturer recommendations
	\end{itemize}
\end{frame}

\begin{frame}{Exterior}
	\begin{description}[leftmargin=3em]
		\item [Bow] Not an issue\\~\\
		\item [Bulges] Condemn all of them\\~\\
		\item [Cuts \& digs] Any cut deeper than $\sfrac{1}{32}"$ ($0.76mm$).\\15\% of the wall thickness if known\\~\\
		\item [Dents] Any dent deeper than $\sfrac{1}{16}"$ ($1.53mm$)\\Diameter of 	less than $2"$ ($50.8mm$)\\~\\
		\item [Fire] More than $350^{\circ} F$
	\end{description}
\end{frame}
%---------------------------------------------------------------------
%										Other services
%---------------------------------------------------------------------
\section{Other services}
\subsection{Valve inspection}
\subsection{Cylinder cleaning}

\end{document}
