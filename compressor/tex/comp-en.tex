%!TEX program = <pdflatex> 

\documentclass[aspectratio=1610,english,12pt]{beamer}

\usepackage{../../shared/tex/mystyle}

\author[]{Chris Braissant}
\title[]{TDI Compressor operator}
\institute{Ban's Diving Resort}
\date{\today}	
	
%---------------------------------------------------------------------
%										DOCUMENT
%---------------------------------------------------------------------
\begin{document}

\begin{frame}[plain]
	\maketitle
\end{frame}

%----- INTRODUCTION -----

\section{Introduction}

\subsection{Intro, Paperwork, Schedule}

\begin{frame}{Course overview}
	Entry-level course\\
	Operating principle\\
	Routine maintenance\\
	Proper handling\\
	Filter changing\\
	Practical part in the workshop\\
\end{frame}

\begin{frame}{Paperwork}
	Liability Release and Express Assumption of Risk\\
	Medical Statement
\end{frame}

\begin{frame}{Liability Release}
	\begin{columns}[onlytextwidth]
		\column{0.6\linewidth}
			\mypict{../pdf/misc/Liability}
		\column{0.4\linewidth}
			\begin{enumerate}\itemsep0em 
				\item Compressor Operator
				\item Your name
				\item Instr.: Chris Braissant
				\item Facility: Ban's diving
				\item Your name
				\item Instr.: Chris Braissant
				\item Facility: Ban's Diving
				\item - blank -
				\item Training agency:\\TDI/SDI
				\item Signature and date
				\item Witness signature and date
				\item Your initials
			\end{enumerate}
	\end{columns}
\end{frame}

\section{Theory}

\subsection{Heat exchange}
	\begin{frame}{Polytropic process}
		A process which occurs with an INTERCHANGE OF BOTH HEAT AND WORK between the system and its surroundings.\par
		\[ \boxed{PV^n=constant}	 \qquad (n=polytropic \; index)\]
	\end{frame}

	\begin{frame}{Isothermal process}
		A process which occurs at CONSTANT TEMPERATURE.\\
		(Boyle-Mariotte's Law) \par
		\[ \boxed{PV=constant} \qquad (n=1)\]
	\end{frame}

	\begin{frame}{Adiabatic process}
		A process in which heat does not enter or leave the system concerned (all heat is contained).\par
		\[ \boxed{PV^{\gamma}=constant} \qquad (\gamma=adiabatic \; index) \]

		\vspace{1cm}
				
		Monoatomic gas (inert gas)
		\begin{itemize}
			\item helium, argon, neon, xenon,...\\ $\gamma = \sfrac{5}{3}$   
		\end{itemize}
		Diatomic gas
		\begin{itemize}
			\item oxygen, nitrogen, hydrogen, air...\\ $\gamma = \sfrac{7}{5}$ 
		\end{itemize}
	\end{frame}

	\begin{frame}{Adiabatic compression}
		Without going to much into details, the formula to find the temperature after compression in a adiabatic process is:
		\[ \boxed{ T_2 = T_1 \left( \frac{P_2}{P_1} \right) ^{\frac{\gamma-1}{\gamma}} }\]
		The temperature after compression is independant of the volume (or size) of a compressor.\\Only the pressures and the ambiant temperature could change the final temperature.
	\end{frame}

	\begin{frame}{Adiabatic compression}
		\mypict{../pdf/Periodic_table}
	\end{frame}

	\begin{frame}{Adiabatic compression}
		\mypict{../pdf/Adiabatic_temperature}
	\end{frame}

\section{Compressor design}

\subsection{Single stage}

	\begin{frame}{Single stage's design}
		\begin{columns}[onlytextwidth]
			\column{0.3\linewidth}
				\mypict{../pdf/single_stage}
			\column{0.7\linewidth}
				A compressor's stage is composed of:
				\begin{itemize}
					\item Inlet and outlet pipe
					\item Inlet and outlet valve
					\item Cylinder
					\item Piston - compressing the air in the chamber
					\item Piston rings - making a seal between the piston and the cylinder 
					\item Rod - connecting the crankshaft to the piston
					\item Crankshaft - moving the piston up and down (not on the drawing)
				\end{itemize}
		\end{columns}
	\end{frame}

	\begin{frame}{Single stage's design}
		\mypict{../pdf/single_acting}
	\end{frame}

	\begin{frame}{Compressor phases}
		\begin{columns}[onlytextwidth]
			\column{0.3\linewidth}
				\mypict{../pdf/single_stage_1}
			\column{0.7\linewidth}
				First phase:
				\begin{itemize}
					\item Inlet valve is open
					\item Outlet valve is closed
					\item The chamber is filled with low pressure air
				\end{itemize}
		\end{columns}
	\end{frame}

	\begin{frame}{Compressor phases}
		\begin{columns}[onlytextwidth]
			\column{0.3\linewidth}
				\mypict{../pdf/single_stage_2}
			\column{0.7\linewidth}
				Second phase:
				\begin{itemize}
					\item Piston starts to go upward
					\item Pressure increase in the piston
					\item Inlet valve is closed
					\item Pressure not sufficient enough to open the outlet valve
				\end{itemize}
		\end{columns}
	\end{frame}

	\begin{frame}{Compressor phases}
		\begin{columns}[onlytextwidth]
			\column{0.3\linewidth}
				\mypict{../pdf/single_stage_3}
			\column{0.7\linewidth}
				Thrid phase:
				\begin{itemize}
					\item Piston reaches the maximum stroke
					\item Pressure open the outlet valve
					\item High pressure air leave the chamber
				\end{itemize}
		\end{columns}
	\end{frame}

	\begin{frame}{Compressor phases}
		\begin{columns}[onlytextwidth]
			\column{0.3\linewidth}
				\mypict{../pdf/single_stage_4}
			\column{0.7\linewidth}
				Fourth phase:
				\begin{itemize}
					\item Piston goes downward
					\item Pressure drops in the piston
					\item Inlet valve is open
					\item Outlet valve is closed
					\item The chamber is filled with low pressure air
				\end{itemize}
		\end{columns}
	\end{frame}

	\begin{frame}{Compressor valves}
		\begin{columns}[onlytextwidth]
			\column{0.3\linewidth}
				\mypict{../img/Valve.jpg}
			\column{0.7\linewidth}
				\begin{itemize}
					\item Same principle as a dump valve
					\item Pressure push intermediate plate
					\item Spring are compressed
					\item Air can flow through the gap
				\end{itemize}
		\end{columns}
	\end{frame}

	\begin{frame}{Compressor valves}
		\mypict{../img/Valve.jpg}
	\end{frame}

	\begin{frame}{Temperature}
		The temperature of the gas after compression can be determined using the formulas seen before:
			\[ \boxed{T_2 = T_1 \left( \frac{P_2}{P_1} \right) ^{\frac{\gamma-1}{\gamma}}} \]
		which give us a temperature of $1257 [^{\circ}C]$ !!!\par
		\vfill		
		\begin{tabularx}{\linewidth}{X l}
			Temp. before compression: & $T_1 = 27 [^{\circ}C]=300[K]$\\
			Pressure before compression: & $P_1 = 1 bar$\\
			Pressure after compression: & $P_2 = 300 bar$ \\
			Adiabatic index: & $\gamma = \sfrac{7}{5} \Rightarrow {\frac{\gamma-1}{\gamma}}=\sfrac{2}{7}$
		\end{tabularx}
	\end{frame}

	\begin{frame}{Temperature}
		The temperature of the gas after compression can be determined using the formulas seen before:
		\begin{align*}
			T_2 	= T_1 \left( \frac{P_2}{P_1} \right) ^{\frac{\gamma-1}{\gamma}}
					= 300 \left( \frac{300}{1} \right) ^{\frac{2}{7}}
					= 1530 [K]
					= 1257 [^{\circ}C]
		\end{align*}
	\end{frame}

%\subsection{Chemical reactions}
%\subsection{Compression rates and heat}
%
%\section{Mechanical operation}
%\subsection{Starting and stopping}
%\subsubsection{Electrical engine}
%\subsubsection{Fuel engine}
%\subsubsection{Diesel engine}
%
%\subsection{Bleeding and pressure releases}
%\subsubsection{Drains}
%\subsubsection{Safety valves}
%\subsubsection{PMV}
%
%\section{Tech related}
%\subsection{Oxygen clean air creation}
%\subsubsection{Oxygen less / free}
%\subsubsection{Inline filter}
%
%\subsection{DIN valve use}
%\subsection{Air Bank systems}
%\subsubsection{Air bank}
%\subsubsection{Blending panel}
%
%\section{Logging}
%\subsection{Pre Operation checks}
%\subsubsection{Check list}
%\subsubsection{Maintenance}
%
%\subsection{Monitoring filter life}
%\subsubsection{Log book}
%\subsubsection{Electronic devices}
%
%\section{Remote}
%\subsection{Using petrol powered engines}
%\subsection{Planning compressor size based on needs}
%\subsection{Estimating output based on compressors}
%
%\section{Mixing procedures}
%\subsection{Partial pressure blending}
%\subsection{Continuous blending}
%\subsection{Membrane separation system}


%%%%%%%%%%%%%%%%%%%% FILTRATION %%%%%%%%%%%%%%%%%%%%
\section{Filtration}

\subsection{Air Quality}

\begin{frame}{Air Quality}
	\begin{tabularx}{\linewidth}{|X|c|c|c|c|}
		\hline	& EN12021 & BS4275 & $O_2$ Clean Air & PADI 5 stars \\ \hline
		Oxygen 	& \multicolumn{4}{c|}{$21 \pm 1\%$}\\ \hline
		$CO_2$	& \multicolumn{4}{c|}{$<500ppm$}\\ \hline
		$CO$		& $<15ppm$	& $<5ppm$	& $<5ppm$	& $<10ppm$\\ \hline
		Oil		& $<0.5ppm$	& $<0.5ppm$	& $<0.3ppm$	& $<5ppm$\\ \hline
		Water	& \multicolumn{3}{c|}{$<25mg/m^2$} & Not specified \\ \hline
		Odor / taste			& \multicolumn{4}{c|}{None}\\ \hline	
	\end{tabularx}
\end{frame}


%%%%%%%%%%%%%%%%%%%% RUNNING %%%%%%%%%%%%%%%%%%%%
\section{Starting and running}

% COMPRESSOR
\subsection{Compressor}
\begin{frame}{Compressor pre-startup checks}
	\begin{itemize}
		\item Oil level
		\item Input filter
		\item Output filter
		\item Fan belt
		\item Hoses
		\item Filter towers
		\item Log book
	\end{itemize}
\end{frame}

\begin{frame}{Compressor startup sequence}
    \begin{itemize}
    	\item Condensate drains
    	\item Drain valves opened
    	\item All Gauges at zero
    	\item Start compressor
    	\item Clacking noise from 3rd stage
    	\item Close drain valves
    	\item Pressure maintaining valve (PMV)
    	\item Safety valves goes off
    \end{itemize}
\end{frame}

\begin{frame}{Compressor shut sown sequence}
	\begin{itemize}
		\item Open drain valves
		\item Gauges going down
		\item Turn off engine / motor
	\end{itemize}
\end{frame}

\begin{frame}{Compressor runinng tests}
    \begin{itemize}
    	\item RPM - Motor speed
    	\item Vibration
    	\item Condensate every x minutes
    	\item Filter
    \end{itemize}
\end{frame}

% PETROL
\subsection{Petrol engine}
\begin{frame}{Petrol engine pre-startup checks}
	\begin{itemize}
			\item Oil level
			\item Input filter
			\item Spark plug
			\item Petrol level
		\end{itemize}	
\end{frame}

\begin{frame}{Petrol engine startup sequence}
	\begin{itemize}
		\item Turn fuel ON
		\item Turn ignition ON
		\item Place choke on COLD START position
		\item Set throttle to 1/3 full speed
		\item Slowly pull starter cord to remove slack
		\item Sharply pull starter cord
		\item Set throttle to full speed
		\item Place choke on WARM/RUN position
	\end{itemize}
\end{frame}

\begin{frame}{Petrol engine switch off sequence}
	\begin{itemize}
		\item Slowwly set throttle to minimum speed
		\item Turn fuel OFF until motor stop
		\item Turn ignition OFF
	\end{itemize}
\end{frame}

%%%%%%%%%%%%%%%%%%%% SOURCES %%%%%%%%%%%%%%%%%%%%
\section{Sources}
\begin{frame}{Sources}
    \begin{itemize}
    	\item url
  %  	\item \hyperlink{}{http://en.wikipedia.org/wiki/Gas_compressor}
   % 	\item \hyperlink{}{http://en.wikipedia.org/wiki/Adiabatic_process}
    \end{itemize}
\end{frame}
\end{document}
