%!TEX root = advnitrox_fr.tex

\section{Equipement}

\begin{frame}{Compatibilité à l'oxygène}  
	Régle des 40\%:
	\begin{itemize}
		\item Au dessous: considéré comme de l'air
		\item Au dessus: considéré comme de l'oxygène pur
	\end{itemize}
\end{frame}

\begin{frame}{Compatibilité à l'oxygène}  
	Compatible?
	\begin{itemize}
		\item Spécifications du fabricants
		\item Matériaux compatibles (o-ring, lubrifiants, métaux, ...)
		\item Nettoyé? (absence de particules, hydrocarbone, ...)\\
				Re-nettoyer si utilisé avec du matériel contaminé (bouteille d'air)
	\end{itemize}
\end{frame}

\begin{frame}{Compatibilité à l'oxygène}  
	Nettoyage
	\begin{itemize}
		\item Démonter complétement le détendeur
		\item Nettoyer toutes les pièces à l'aide d'un produit adapté\\(cf: manuel du fabricant)
		\item Contrôler l'absence de contaminent (huile, graisse, lubrifiant, ...)
		\item Lubrifié à l'aide de produit adapté (cf: manuel du fabricant)
		\item Remonter le détendeur avec des outils propres!
		\item Tester
	\end{itemize}
\end{frame}

\begin{frame}{Compressor}
	\begin{tcolorbox}[colback=white,colframe=mblue,title=ATTENTION]
		\centering\textbf{NE JAMAIS METTRE DE L'OXYGÈNE PUR À L'ENTRÉE D'UN COMPRESSEUR!}
	\end{tcolorbox}
\end{frame}