%!TEX root = advnitrox_fr.tex

\section{Formula}

\begin{frame}{Loi de Boyle-Mariotte}
	Relation entre Pression et Volume\\
	\[	\boxed{ P \propto \frac{1}{V} \qquad (T=constant) } \]
	D'ou:
	\[ \boxed{ P_1 \times V_1 = P_2 \times V_2 } \]
\end{frame}

\begin{frame}{Loi de Boyle-Mariotte}  
	\underline{Exemple:} Compression d'un ballon à 30m
	\begin{tabularx}{\linewidth}{l X}
		Pression à la surface:	&	$P_1 = 1 bar$\\
		Volume à la surface:		&	$V_1 = 8l$\\
		Pression à 30m:			&	$P_2 = (30/10)+1 = 4 bar$\\
		Volume à 30m:				&	$V_2 = (P_1 \times V_1) / P_2 = (1 \times 8) / 4 = 2l$
	\end{tabularx}
\end{frame}

\begin{frame}{Loi de Dalton}  
	Relation entre pression partielle, pression ambiante et fraction gazeuse.\\
	\[ \boxed{ \dfrac{PP}{Pa \bigg| Fg} } \]
	
	PP = Pression Partielle [bar]\\
	Pa = Pression Ambiant [bar]\\
	Fg = Fraction gazeuse [-]	
\end{frame}

\begin{frame}{Loi de Dalton}
	Pression partielle:
	\begin{itemize}
		\item Permet de déterminer la pression partielle d'un gaz en profondeur
		\item Exposition à l'oxygène (toxicité)
		\item Exposition à l'azote (décompression)
	\end{itemize}
	\[ \boxed {PP = Pa \times Fg} \]
	\vfill
	\underline{Exemple:}	Pression partielle d'oxygène pour de l'air à 40m\\
	\[ PO_2 = 5 \times 0.21 = 1.05 bar\]
\end{frame}

\begin{frame}{Loi de Dalton}
	Profondeur maximum d'utilisation (MOD):
	\begin{itemize}
		\item Permet de déterminer la profondeur maximale à laquelle un mélange gazeux peut être utilisé
	\end{itemize}
	\[ \boxed{Pa = PP / Fg} \]
	\vfill
	\underline{Exemple:}	Profondeur maximale d'utilisation d'un Nitrox 36\\
	\[ Pa = 1.6 / 0.36 = 4.4 bar \]
	\[ P = (4.4 - 1) \times 10 = 34m \]	
\end{frame}

\begin{frame}{Loi de Dalton}
	Mélange idéal:
	\begin{itemize}
		\item Permet de déterminer le mélange idéal pour une profondeur donnée
	\end{itemize}
	 \[ \boxed{Fg = PP / Pa} \]
	\vfill
	\underline{Exemple:}	Mélange idéal pour une plongée à 28m\\
	\[	Pa = (28 / 10)+1 = 3.8 bar\]
	\[ FO_2 = 1.4 / 3.8 = 0.36 = EAN36 \]
\end{frame}

\begin{frame}{Profondeur équivalente à l'air (EAD)}
	La pression partielle d'azote détermine la limite de non-décompression\\
	Converti la profondeur actuelle au nitrox en une profondeur équivalente à l'air\\
	Permet d'utiliser une table de plongée à l'air, peu importe le mélange de nitrox utilisé\\

	\[	EAD = \left( \dfrac{ {FN_2}_{mix} }{ {FN_2}_{air} } \times (Profondeur+10) \right) -1 \]
	\vfill
	\underline{Exemple:} Profondeur équivalente pour un Nitrox 36 à 30m\\

	\[	EAD = \left( \dfrac{1-0.32}{0.79} \times (30+10) \right) -10 = \dfrac{0.68}{0.79} \times 40 - 10 = 24.4m \]
\end{frame}
