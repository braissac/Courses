\documentclass[english,12pt,a4paper]{article}

\usepackage{../../shared/tex/course_tdi}

\title{TDI Nitrox avancé}
\subtitle{Révision de connaissances}

\author{TECHNICAL DIVING INTERNATIONAL}
\website{www.tdisdi.com}


%---------------------------------------------------------------------
%										FORMAT
%---------------------------------------------------------------------
% Change le format des listes
\renewenvironment{benumerate}%
{\normalfont\begin{list}{}{\samepage}}%
{\end{list}	}

%---------------------------------------------------------------------
%										DOCUMENT
%---------------------------------------------------------------------
\begin{document}

	\begin{titlepage}
	\begin{center}
		\includegraphics[width=13cm]{\tdilogo}\\
		\vspace{1cm}
		{\fontsize{40}{48}\selectfont \textbf{\thetitle}}\\
		\vspace{1cm}
		{\fontsize{30}{36}\selectfont \textbf{\thesubtitle}}\\
		\vspace{4cm}
		{\fontsize{18}{22}\selectfont \textbf{\theauthor}}\\
		\vspace{0.2cm}
		{\fontsize{18}{22}\selectfont \textbf{\thewebsite}}\\
	\end{center}
\end{titlepage}

%---------------------------------------------------------------------
%									Fundamentals
%---------------------------------------------------------------------
	\setcounter{section}{1}
	\section{Principes fondamentaux de la plongée technique}

	\begin{outline}
		\1 Un plongeur élite aura son attention centrée sur?
			\2 \hspace{-2em}\hrulefill
			\2 \hspace{-2em}\hrulefill
	
		\1 Un plongeur élite se concentre sur quelles compétances?
			\2 \hspace{-2em}\hrulefill
			\2 \hspace{-2em}\hrulefill
			\2 \hspace{-2em}\hrulefill
	
		\1 Quelle est l'influence des poumons sur la flottabilité?
			\2 \hspace{-2em}\hrulefill
			\2 \hspace{-2em}\hrulefill
			\2 \hspace{-2em}\hrulefill
	
		\1 Pour une respiration idéale, le plongeur devrait remplir ses poumons par le \underline{\hspace{1.5cm}} et les vider par le \underline{\hspace{1.5cm}}.
	
		\1 Pour quelles raisons devrait-on s'écarter d'un rythme respiratoire idéal?
			\2 \hspace{-2em}\hrulefill
			\2 \hspace{-2em}\hrulefill
	
		\1 En plongée, quand les mains sont utilisées pour se déplacer ou se tourner
			\2 \hspace{-2em}\hrulefill
			\2 \hspace{-2em}\hrulefill
	
		\1 Quels sont les avantages à plonger en position alignée?
			\2 \hspace{-2em}\hrulefill
			\2 \hspace{-2em}\hrulefill
			\2 \hspace{-2em}\hrulefill
	
		\1 Dans quelle position doit être un plongeur dans la partie active de la plongée près 	d'un fond marin délicat?
			\2 \hspace{-2em}\hrulefill
			\2 \hspace{-2em}\hrulefill

		\1 Y a-t-il un moment en plongée où les compétances de base peuvent être ignorées?
			\2 \hspace{-2em}\hrulefill
			\2 \hspace{-2em}\hrulefill
	\end{outline}
	\pagebreak

%---------------------------------------------------------------------
%									Physique et Lois des gas
%---------------------------------------------------------------------
	\section{Physique et Lois des gas}
	\begin{outline}
		\1 L'oxygène est un gaz \underline{\hspace{1.5cm}} et \underline{\hspace{1.5cm}}.

		\1 Est-ce que l'oxygène est nécessaire pour la vie?

		\1 Citer deux calculs  que vous pouver faire grâce à la loi de Bolye.
			\2 \hspace{-2em}\hrulefill
			\2 \hspace{-2em}\hrulefill

		\1 Citer trois calculs que vous pouver faire grâce à la loi de Dalton
			\2 \hspace{-2em}\hrulefill
			\2 \hspace{-2em}\hrulefill
			\2 \hspace{-2em}\hrulefill

		\1 Pourquoi la Profondeur Équivalente à l'Air (EAD) permet au plongeur de plonger plus longtemps?
			\2 \hspace{-2em}\hrulefill
			\2 \hspace{-2em}\hrulefill
			\2 \hspace{-2em}\hrulefill
			\2 \hspace{-2em}\hrulefill
	\end{outline}
	\pagebreak

%---------------------------------------------------------------------
%									Physiologie
%---------------------------------------------------------------------
	\section{Physiologie}
	\begin{outline}
		\1 L'hypoxie est:
			\2 \hspace{-2em}\hrulefill
			\2 \hspace{-2em}\hrulefill
			
		\1 N'importe quel mélange avec moins de \underline{\hspace{1.5cm}} pourcent d'oxygène est considéré comme hypoxique.

		\1 Pourquoi le risque d'hypoxie est plus élevé avec un recylceur?
			\2 \hspace{-2em}\hrulefill
			\2 \hspace{-2em}\hrulefill
			\2 \hspace{-2em}\hrulefill

		\1 Un mélange gazeux ne contenant pas du tout d'oxygène est appelé:
			\2 \hspace{-2em}\hrulefill

		\1 Si un plongeur est exposé à plus de \underline{\hspace{1.5cm}} ATA de pression partielle d'oxygène, ce exposition doit être suivie.

		\1 Les signes et symptômes d'une toxité à l'oxygène du système nerveux central sont caractérisées par:
			\2 \hspace{-2em}\hrulefill
			\2 \hspace{-2em}\hrulefill
			\2 \hspace{-2em}\hrulefill
			\2 \hspace{-2em}\hrulefill
			\2 \hspace{-2em}\hrulefill
			\2 \hspace{-2em}\hrulefill
			\2 \hspace{-2em}\hrulefill

		\1 Quel est le pourcentage d'exposition à l'oxygène pour une plongée avec une exposition à une pression pertielle de 1.2 ATA pendant 55 minutes?
			\2 \hspace{-2em}\hrulefill
			\2 \hspace{-2em}\hrulefill

		\1 Quels sont les signes et symptômes de la toxicité pulmonaire à l'oxygène?
			\2 \hspace{-2em}\hrulefill
			\2 \hspace{-2em}\hrulefill

		\1 Combien d'unités d'oxygène (OTUs) un plongeur aura-t-il après une\\plongée de 55 minutes à une pression partielle d'oxygène de 1.2 ATA?
			\2 \hspace{-2em}\hrulefill
			\2 \hspace{-2em}\hrulefill

		\1 Quelle quantité d'OTUs est autorisée pour cinq jours de plongée?
			\2 \hspace{-2em}\hrulefill

		\1 Citer trois choses qu'un plongeur peut faire pour améliorer ses remontées en plongée sportive.
			\2 \hspace{-2em}\hrulefill
			\2 \hspace{-2em}\hrulefill
			\2 \hspace{-2em}\hrulefill

		\1 Quels sont les signes et symptômes de la narcose?
			\2 \hspace{-2em}\hrulefill
			\2 \hspace{-2em}\hrulefill
			\2 \hspace{-2em}\hrulefill

		\1 Décrire comment minimiser le dioxide de carbone pendant la plongée.
			\2 \hspace{-2em}\hrulefill
			\2 \hspace{-2em}\hrulefill
			\2 \hspace{-2em}\hrulefill
			\2 \hspace{-2em}\hrulefill

		\1 Quelle est la source principale de monoxide de carbone dans un mélange gazeux?
			\2 \hspace{-2em}\hrulefill
			\2 \hspace{-2em}\hrulefill

		\1 Décrire comment le système immunitaire peut compliquer un accident de plongée.
			\2 \hspace{-2em}\hrulefill
			\2 \hspace{-2em}\hrulefill
			\2 \hspace{-2em}\hrulefill
			\2 \hspace{-2em}\hrulefill
	\end{outline}
	\pagebreak

%---------------------------------------------------------------------
%									Formules
%---------------------------------------------------------------------
	\section{Formules}
	\begin{outline}
	\1 Quelle est la différence de pression à 30m / 99 ft comparé à la surface?
			\2 \hspace{-2em}\hrulefill
			\2 \hspace{-2em}\hrulefill
			\2 \hspace{-2em}\hrulefill

	\1 Quel est le meilleur mélange pour une plongée à 32m / 107 ft?
			\2 \hspace{-2em}\hrulefill
			\2 \hspace{-2em}\hrulefill
			\2 \hspace{-2em}\hrulefill

	\1 Quelle est la profondeur maximale d'utilisation pour un EAN27 pour la partie profonde et pour la décompression?
			\2 \hspace{-2em}\hrulefill
			\2 \hspace{-2em}\hrulefill
			\2 \hspace{-2em}\hrulefill

	\1 Quelle est la pression partielle d'oxygène à 30m / 100 ft pour un EAN30?
			\2 \hspace{-2em}\hrulefill
			\2 \hspace{-2em}\hrulefill
			\2 \hspace{-2em}\hrulefill

	\1 Quelle est la profondeur équivalente à l'air (EAD) pour un EAN36 à 25m / 85ft?
			\2 \hspace{-2em}\hrulefill
			\2 \hspace{-2em}\hrulefill
			\2 \hspace{-2em}\hrulefill
	\end{outline}
	\pagebreak

%---------------------------------------------------------------------
%									Equipement
%---------------------------------------------------------------------
	\section{Equipement}
	\begin{outline}
		\1 L'équipement utilisé avec un pourcentage d'oxygène supérieur à \underline{\hspace{1.5cm}} pourcent doit être nettoyé pour l'utilisation avec de l'oxygène.

		\1 Citer 3 étapes dans le nettoyage pour l'utilisation avec de l'oxygène.
			\2 \hspace{-2em}\hrulefill
			\2 \hspace{-2em}\hrulefill
			\2 \hspace{-2em}\hrulefill

		\1 Que faire si une pièce d'équipement est contaminée ou utilisée avec un gaz non compatible?
			\2 \hspace{-2em}\hrulefill
			\2 \hspace{-2em}\hrulefill
			\2 \hspace{-2em}\hrulefill
	\end{outline}
	\pagebreak

%---------------------------------------------------------------------
%								Utilisation de Nitrox
%---------------------------------------------------------------------
	\section{Utilisation de Nitrox}
	\begin{outline}
		\1 Citer trois outils permettant de planifier une plongée.
			\2 \hspace{-2em}\hrulefill
			\2 \hspace{-2em}\hrulefill
			\2 \hspace{-2em}\hrulefill

		\1 Citer trois limites de l'utilisation d'un ordinateur de plongée.
			\2 \hspace{-2em}\hrulefill
			\2 \hspace{-2em}\hrulefill
			\2 \hspace{-2em}\hrulefill

		\1 Citer trois avantages à utliser un ordinateur de plongée.
			\2 \hspace{-2em}\hrulefill
			\2 \hspace{-2em}\hrulefill
			\2 \hspace{-2em}\hrulefill

		\1 Quelle est la meilleure redondace à un ordinateur de plongée?
			\2 \hspace{-2em}\hrulefill
			\2 \hspace{-2em}\hrulefill
	\end{outline}
	\pagebreak

%---------------------------------------------------------------------
%								Planification
%---------------------------------------------------------------------
	\section{Planification}
	\begin{outline}
		\1 Quels sont les bénéfices à planifier manuellement une plongée?
			\2 \hspace{-2em}\hrulefill
			\2 \hspace{-2em}\hrulefill
			\2 \hspace{-2em}\hrulefill
			\2 \hspace{-2em}\hrulefill

		\1 Citer cinq aspects important d'une planification.
			\2 \hspace{-2em}\hrulefill
			\2 \hspace{-2em}\hrulefill
			\2 \hspace{-2em}\hrulefill
			\2 \hspace{-2em}\hrulefill
			\2 \hspace{-2em}\hrulefill

		\1 Pourquoi est-il important de calculer la consommation de gaz?
			\2 \hspace{-2em}\hrulefill
			\2 \hspace{-2em}\hrulefill
			\2 \hspace{-2em}\hrulefill

		\1 Un plongeur utilise 12bar de gaz à 10m en 10 minutes. Le plongeur utilise une bouteille de 11l gonflée à 200bar. Quelle est sa consommation?
			\2 \hspace{-2em}\hrulefill
			\2 \hspace{-2em}\hrulefill
			\2 \hspace{-2em}\hrulefill
			\2 \hspace{-2em}\hrulefill

		\1 Une plongeur ayant une consommation de 13l/min planifie une plongée à 25m pour 35 minutes. Quelle est la quantité de gaz nécessaire pour la partie profonde de la plongée?
			\2 \hspace{-2em}\hrulefill
			\2 \hspace{-2em}\hrulefill
			\2 \hspace{-2em}\hrulefill
			\2 \hspace{-2em}\hrulefill

		\1 Effectuer une planification pour un plongeur voulant faire une plongée multi-niveaux sur un tombant à 28m pour 30 minutes. Le plongeur à une consommation de 5l/min. La plongée sera effectuée dans un environnement tropical en eau chaude. Inclure le choix du gaz et pourquoi, l'exposition à l'oxygène, la planification de la décompression, les calculs d'approvisionnement en gaz, les choix d'équipement, et les considérations thermiques.
			\2 \hspace{-2em}\hrulefill
			\2 \hspace{-2em}\hrulefill
			\2 \hspace{-2em}\hrulefill
			\2 \hspace{-2em}\hrulefill
			\2 \hspace{-2em}\hrulefill
			\2 \hspace{-2em}\hrulefill
			\2 \hspace{-2em}\hrulefill
			\2 \hspace{-2em}\hrulefill
	\end{outline}
	\pagebreak

%---------------------------------------------------------------------
%								Protocoles
%---------------------------------------------------------------------
	\section{Protocoles}
	\begin{outline}
		\1 Quels sont la avantages d'une checklist pré-plongée?
			\2 \hspace{-2em}\hrulefill
			\2 \hspace{-2em}\hrulefill
			\2 \hspace{-2em}\hrulefill

		\1 Citer trois choses qu'un plongeur doit faire avant de respirer une mélange gazeux.
			\2 \hspace{-2em}\hrulefill
			\2 \hspace{-2em}\hrulefill
			\2 \hspace{-2em}\hrulefill

		\1 Lister les éléments de l'acronyme START et définir chaque élément.
			\2[S] \hrulefill
			\2[T] \hrulefill
			\2[A] \hrulefill
			\2[R] \hrulefill
			\2[T] \hrulefill

		\1 Décrire une descente idéale.
			\2 \hspace{-2em}\hrulefill
			\2 \hspace{-2em}\hrulefill
			\2 \hspace{-2em}\hrulefill

		\1 Décrire une remontée idéale.
			\2 \hspace{-2em}\hrulefill
			\2 \hspace{-2em}\hrulefill
			\2 \hspace{-2em}\hrulefill

		\1 Pour une plongée à 30m, lister les paliers à effectuer en ayant une approche "de paliers de sécurité multi-niveaux".
			\2 \hspace{-2em}\hrulefill
			\2 \hspace{-2em}\hrulefill
			\2 \hspace{-2em}\hrulefill

		\1 Décrire la philosophie idéale pour une planification d'urgence.
			\2 \hspace{-2em}\hrulefill
			\2 \hspace{-2em}\hrulefill
			\2 \hspace{-2em}\hrulefill
	\end{outline}
	\pagebreak

%---------------------------------------------------------------------
%								Mélanges gazeux
%---------------------------------------------------------------------
	\section{Mélanges gazeux}
	\begin{outline}
		\1 Décrire le mélange par pression partielle.
			\2 \hspace{-2em}\hrulefill
			\2 \hspace{-2em}\hrulefill
			\2 \hspace{-2em}\hrulefill
			\2 \hspace{-2em}\hrulefill
			\2 \hspace{-2em}\hrulefill
			\2 \hspace{-2em}\hrulefill

		\1 Décrire le mélange par flux continu.
			\2 \hspace{-2em}\hrulefill
			\2 \hspace{-2em}\hrulefill
			\2 \hspace{-2em}\hrulefill
			\2 \hspace{-2em}\hrulefill
			\2 \hspace{-2em}\hrulefill
			\2 \hspace{-2em}\hrulefill

		\1 Décrire le fonctionnement d'un système à membrane.
			\2 \hspace{-2em}\hrulefill
			\2 \hspace{-2em}\hrulefill
			\2 \hspace{-2em}\hrulefill
			\2 \hspace{-2em}\hrulefill
			\2 \hspace{-2em}\hrulefill
			\2 \hspace{-2em}\hrulefill

		\1 

	\end{outline}
	\pagebreak
\end{document} 
