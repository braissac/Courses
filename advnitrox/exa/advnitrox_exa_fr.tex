\documentclass[english,10pt,a4paper, twoside]{article}

\usepackage{../../shared/tex/course_tdi}

\title{TDI Nitrox avancé}
\subtitle{Examen final}

\author{TECHNICAL DIVING INTERNATIONAL}
\website{www.tdisdi.com}

%---------------------------------------------------------------------
%										DOCUMENT
%---------------------------------------------------------------------
\begin{document}
\sloppy
\begin{titlepage}
	\begin{center}
		\includegraphics[width=13cm]{\tdilogo}\\
		\vspace{1cm}
		{\fontsize{40}{48}\selectfont \textbf{\thetitle}}\\
		\vspace{1cm}
		{\fontsize{30}{36}\selectfont \textbf{\thesubtitle}}\\
		\vspace{4cm}
		{\fontsize{18}{22}\selectfont \textbf{\theauthor}}\\
		\vspace{0.2cm}
		{\fontsize{18}{22}\selectfont \textbf{\thewebsite}}\\
	\end{center}
\end{titlepage}

%---------------------------------------------------------------------
%									Fundamentals
%---------------------------------------------------------------------
	\begin{outline}
		\1 Pour développer son esprit technique, un plongeur doit baser ses limites de plongée sur:
			\2 Quantité de gaz disponnible
			\2 Equipement utilisé
			\2 Expérience
			\2 Toutes les réponses ci-dessus
	
		\1 Les compétances fondamentales sont:
			\2 Flottabilité et position
			\2 Palmage et respiration
			\2 Réponses A et B
			\2 Aucune des réponses ci-dessus
	
		\1 Optimiser \st aide le plongeur à maitriser d'autres techniques lorsqu'il progresse durant le cours.
			\2 La flottabilité
			\2 L'habillement
			\2 Le choix des couleurs pour les accessoires
			\2 Toutes les réponses ci-dessus
	
		\1 La respiration idéale est techniquement appellée "Respiration initiée par le diaphragme". \vf
% 5	
		\1 La loi de Boyle décrit le rapport \st entre la pression et le volume à température constante.
			\2 Rival
			\2 Directement proportionnel
			\2 Inversement proportionnel
			\2 Toutes les réponses ci-dessus

		\1 La loi de Dalton est utilisée pour calculer:
			\2 Profondeur maximale d'utilisation (MOD)
			\2 Mélange idéal
			\2 Pression partielle
			\2 Toutes les réponses ci-dessus
		
		\1 L'hypoxie est:
			\2 Pas assez d'oxygène
			\2 Trop d'oxygène
			\2 Pas assez d'azote
			\2 Trop d'azote

		\1 Bien que l'oxygène soit nécessaire pour la vie, une trop grande quantité peut être dangereux. \vf

		\1 Les plongeurs choisissent une pression partielle d'oxygène de \st comme maximum recommandé durant la partie profonde de la plongée.
			\2 1.2 ATA
			\2 1.4 ATA
			\2 1.6 ATA
			\2 1.8 ATA
% 10
		\1 Les plongeurs doivent surveiller l'exposition à l'oxygène à l'aide de:
			\2 Tables de plongée
			\2 Formules mathématiques qu'ils ont développé
			\2 Limite d'exposition à l'oxygène (CNS clock)
			\2 Table de marée

		\1 Si un plongeur est exposé à une pression partielle d'oxygène de 1.4 ATA pendant 30 minutes, quelle sera son exposition à l'oxygène?\\ (Utiliser les tables)
			\2 20\%
			\2 30\%
			\2 40\%
			\2 50\%

		\1 Le terme OTUs veut dire:
			\2 Over Time Usually (Habituellement au dessus du temps)
			\2 Out The Upperdeck (Sur le pont supérieur)
			\2 Oxygen Tolerance Units (Unité de tolérance à l'oxygène)
			\2 Aucune des réponses ci-dessus

		\1 Le reflex inspiratoire est déclanché par:
			\2 Dioxide de carbone
			\2 Oxygène
			\2 Monoxide de carbone
			\2 Azote			

		\1 Les globulles rouges réagissent \st avec le monoxide de carbone qu'avec l'oxygène.
			\2 Moins rapidement
			\2 A la même vitesse
			\2 Plus rapidement
			\2 Aucune des réponses ci-dessus
% 15
		\1 Un plongeur à 25m utilisera son gaz combien de fois plus rapidement qu'à la surface?
			\2 2.5x
			\2 3.5x
			\2 4.5x
			\2 5.5x

		\1 Un plongeur à 95ft utilisera son gaz combien de fois plus rapidement qu'à la surface?
			\2 1.9x
			\2 2.9x
			\2 3.9x
			\2 4.9x

		\1 Un plongeur souhaite faire une plongée à 95ft, quel est le mélange idéal pour cette plongée? ($PO_2 = 1.4bar$)
			\2 6\%
			\2 16\%
			\2 26\%
			\2 36\%

		\1 Un plongeur souhaite faire une plongée à 25m, quel est le mélange idéal pour cette plongée? ($PO_2 = 1.4bar$)
			\2 33\%
			\2 35\%
			\2 38\%
			\2 40\%

		\1 Un plongeur souhaite savoir à quelle profondeur il peut plonger avec un mélange à 31\% d'oxygène.
			\2 35m ou 116ft
			\2 30m ou 100ft
			\2 27m ou 90ft
			\2 24m ou 80ft
% 20
		\1 Un plongeur souhaite savoir à quelle quantité d'oxygène il sera exposé lors d'une plongée à 20m avec un EAN32.
			\2 .96 bar
			\2 1.16 bar
			\2 1.26 bar
			\2 1.46 bar

		\1 Un plongeur souhaite savoir à quelle quantité d'oxygène il sera exposé lors d'une plongée à 80ft avec un EAN32.
			\2 1.10 ATA
			\2 1.12 ATA
			\2 1.14 ATA
			\2 1.20 ATA

		\1 Si un plongeur plonge à 25m avec une EAN33, quelle est la profondeur équivalente à l'air (EAD) pour cette plongée?
			\2 25.7m
			\2 31.7m
			\2 29.9m
			\2 19.7m

		\1 Si un plongeur plonge à 105ft avec une EAN33, quelle est la profondeur équivalente à l'air (EAD) pour cette plongée?
			\2 84ft
			\2 94ft
			\2 104ft
			\2 114ft

		\1 TDI recommande que toutes les bouteilles utilisées avec du nitrox soit:
			\2 Sablées
			\2 Inspectées chaque mois
			\2 Correctement nettoyées pour l'oxygène
			\2 Aucune des réponses ci-dessus
% 25
		\1 Beaucoup d'ordinateurs personnels peuvent être programmés pour plusieurs mélanges gazeux. \vf
		
		\1 Le volume respiratoire à la minute (RMV) est la quantité de gaz:
			\2 Qu'un plongeur véhicule a travers ses poumons en une minute
			\2 Qu'un plongeur utilise durant une plongée
			\2 Qu'un plongeur respire durant sa carrière
			\2 Aucune des réponses ci-dessus

		\1 Avant de plongée avec une bouteille pouvant contenir un mélange gazeux, un plongeur doit:
			\2 Analyser le contenu
			\2 Vérifier que le contenu correspondent au marquage
			\2 Correctement annoter la bouteille
			\2 Toutes les réponses ci-dessus

		\1 START est un acronyme pour S-Drill, Team, Air, Route, Tables. \vf

		\1 Le but d'une procédure "d'abortement" (bailout) est:
			\2 De permettre à un plongeur qui a assez d'air de retourner à la surface en effectuant un palier de sécurité
			\2 De ressortir de l'eau en vitesse en cas d'urgence
			\2 De vider le bateau du surplus d'eau pour éviter qu'il coule
			\2 Aucune des réponses ci-dessus

		\1 Les différentes façon d'effectuer un mélange gazeux comprennent:
			\2 Pression partielle
			\2 Flux continu
			\2 Membrane
			\2 Toutes les réponses ci-dessus

	\end{outline}
\end{document} 
