\documentclass[aspectratio=1610,english,14pt]{beamer}
% Size:  8pt, 9pt, 10pt, 11pt, 12pt, 14pt, 17pt, 20pt.
\usepackage{../../shared/tex/mystyle_tdi}

\author[]{Chris Braissant}
\title[]{VPM - Variable Permeability Model}
\institute{Ban's Diving Resort}

%---------------------------------------------------------------------
%										DOCUMENT
%---------------------------------------------------------------------
\begin{document}

\begin{frame}[plain]
	\maketitle
\end{frame}

%---------------------------------------------------------------------
%										THEORY
%---------------------------------------------------------------------
\section{Introduction}
\subsection{Multideco}
\begin{frame}{}
	\mypict{../img/multideco}
\end{frame}

\begin{frame}
	\mypict{../img/vplanner}
\end{frame}

\begin{frame}
	\mypict{../img/baltic}
\end{frame}

\subsection{Importance to understand VPM}
\begin{frame}{\insertsubsection}
	What is that VPM-B and why is it important to know it?
	\begin{itemize}
		\item Popular decompression model
		\item Alternative to Buhlmann ZHL16
		\item Way to double check your schedule
		\item Understand your dive profile
	\end{itemize}
\end{frame}

\subsection{Objective}
\begin{frame}{\insertsubsection}
	Understand:
	\begin{itemize}
		\item Buhlmann dissolved model
		\item Bubbles behaviour
		\begin{itemize}
			\item Diffusion with tissue compartment
			\item Laplace pressure
			\item Grow or shrink
			\item Critical radius
		\end{itemize}
		\item Principle of VPM model
		\begin{itemize}
			\item How the model is made
			\item Variable permeability
			\item Bubble crushing
			\item Algorithm
			\item Variants
		\end{itemize}
	\end{itemize}
\end{frame}

\subsection{Outline}
\begin{frame}{\insertsubsection}
	Topics:
	\begin{itemize}
		\item Dissolved Gas Model (Buhlmann ZHL16)
		\item Bubble behaviour
		\item VPM model
	\end{itemize}
\end{frame}

\section{Presentation}

%---------------------------------------------------------------------
%										DISSOLVED GAS
%---------------------------------------------------------------------
\subsection{Dissolved Gas Model}
	\begin{frame}
		\resizebox{!}{18pt}{\strut\textbf{\insertsubsection}\strut}\\
		\vfill
   \end{frame}

\begin{frame}{\insertsubsection}
	\textbf{Buhlmann ZHL16}
	\begin{itemize}
	 	\item Gas dissolved in tissue
		\item 16 tissue compartments
		\item Compartments on-gas and off-gas a different rates (Half-Time)
		\item Track supersaturation in the tissue compartments
		\item Decompression stop when a compartment reaches the M-Value
		\item Pressure in a tissue compartment ("tissue tension") vary for each compartment
	 \end{itemize} 
\end{frame}

%---------------------------------------------------------------------
%										BUBBLE BEHAVIOUR
%---------------------------------------------------------------------
\subsection{Bubble Behaviour}
	\begin{frame}
		\resizebox{!}{18pt}{\strut\textbf{\insertsubsection}\strut}\\
		\vfill
   \end{frame}

\begin{frame}{\insertsubsection}
	A bubble will grow if the pressure pushing out is greater than the pressure pushing in
	\pause\vfill
	Pressure pushing in:
	\begin{itemize}
		\item Ambiant Pressure
		\item Surface tension (Laplace pressure)
	\end{itemize}
	\pause\vfill
	Pressure pushing out:
	\begin{itemize}
		\item Bubble pressure
	\end{itemize}
\end{frame}

\begin{frame}{\insertsubsection}  
	\centering\includegraphics[height=0.8\textheight]{../img/bubble.png}
\end{frame}

\begin{frame}{\insertsubsection}
	\textbf{Comportment at saturation:}\\
	\vfill
	Equilibrium of pressure: (Stable bubble)
	$$P_{bubble}=P_{ambiant}+P_{Laplace}$$
	\pause \vfill
	Tissue tension around the bubble:
	$$P_{tissue}=Partial\ pressure\ of\ inert\ gas$$
	\pause \vfill
	If the pressure in the bubble is greater than the tissue tension, gas will diffuse out.\\
	The bubble would then shrink!
	$$P_{bubble}>P_{tissue}\rightarrow Bubble\ shrinks$$
\end{frame}

\begin{frame}{\insertsubsection}
	\textbf{Exemple:}\\
	A diver breating air at 40m:
	\begin{align*}
		P_{ambiant}&=5bar\\
		P_{Laplace}&=0.5bar\\
		P_{bubble}&=P_{ambiant}+P_{Laplace}=5.5bar
	\end{align*}
	\pause
	Pressure in the tissue compartment:
	$$P_{tissue}=FN_2*P_{ambiant}=0.79*5=3.95bar$$
	\pause
	The bubble pressure in greater than the tissue tension, the bubble would then diffuse out and shrink
\end{frame}

\begin{frame}{\insertsubsection}
	\textbf{Ascent:}\\ 
	The ambiant pressure will change quickly and the tissue tension will be delayed (depending on the compartment half-time).\\
	\pause\vfill
	The pressure in the bubble would drop below the tissue tension.\\
	\pause\vfill
	The gas will diffuse in and the bubble will grow.\\
	\pause\vfill
	Hence the decompression stop to allow the tissue compartment and the bubble size to adjust
\end{frame}

\begin{frame}{\insertsubsection}
	\textbf{Exemple:}\\
	Same diver as the before, ascending quickly to 20m:
	\begin{align*}
		P_{ambiant}&=3bar\\
		P_{Laplace}&=0.5bar\\
		P_{bubble}&=3.5bar
	\end{align*}
	\pause
	Pressure in the tissue compartment (don't have time to offgas):
	$$P_{tissue}=3.95bar$$
	\pause
	The bubble pressure is lower than the tissue tension, the bubble would then on-gas and grow
\end{frame}

\begin{frame}{\insertsubsection}
	\textbf{Bubble radius:}\\
	Another factor will affect the growth of a bubble: its radius\\
	\pause \vfill
	The Laplace pressure is directly linked to the bubble radius. As a bubble grows, its "skin" becomes thiner, and the Laplace pressure decreases\\
	\pause \vfill
	The bubble pressure will also decrease, and the bubble will grow faster!\\
	\pause \vfill
	Large bubbles grow, small bubbles shrink
\end{frame}

\begin{frame}{\insertsubsection}
	\textbf{Critical radius:}\\ 
	Maximum radius a bubble can have to maintain an adequate Laplace pressure\\
	\pause\vfill
	Bubbles models try to control the size of the bubbles\\
	\pause\vfill
	Under a critical radius, bubble will shrink as the diver ascent
\end{frame}

%---------------------------------------------------------------------
%										VPM
%---------------------------------------------------------------------
\subsection{VPM: Variable Permeability Model}
	\begin{frame}
		\resizebox{!}{18pt}{\strut\textbf{\insertsubsection}\strut}\\
		\vfill
   \end{frame}

\begin{frame}{\insertsubsection}  
	\textbf{VPM: Variable Permeability Model}
	\begin{itemize}
		\item Same idea as Buhlamnn dissolved gas model 
		\item 16 compartment tracking dissolved inert gases
	 	\item Add a single bubble to each compartment
		\item 16 bubbles in those compartments
		\item Dissolved gas and free gases (bubbles) interacting
	 \end{itemize}
\end{frame}

\begin{frame}{\insertsubsection}  
	\textbf{Specific properties:}
	\begin{itemize}
		\item Specific size indicated by its radius
		\item Varying permeability
		\item Defined skin
		\item Eleasticity and memory
		\item Gas pressure inside the bubble
	\end{itemize}	
\end{frame}

\begin{frame}{\insertsubsection}
	\textbf{Variable Permeability:}
	\begin{itemize}
		\item "Skin" made of tiny balls (molecules)
		\item Dissolved gases from the tissue compartment diffuse through the skin (around the tiny balls)
		\item When the ambiant pressure increase, the bubble is compressed and the skin becomes impermeable
		\item Further crushing becomes more difficult
		\item Hence its name: Variable Permeablilty Model
	\end{itemize}
\end{frame}

\begin{frame}{\insertsubsection}
	\textbf{Bubble crushing:}\\
	During the descent:
	\begin{itemize}
		\item Ambiant pressure increase
		\item Compartments on-gas follwing Buhlmann model
		\item Tissue tension in the compartment increase
		\item Bubbles are crushed
		\item Bubbles in slow compartement crush faster than in fast compartment 
	\end{itemize}
	\pause \vfill
	Remark:\\
	The bubble's size is linked to difference of pressure between the ambiant pressure and the tissue tension
\end{frame}

\begin{frame}{\insertsubsection}  
	\textbf{VPM Algorithm:}
	\begin{itemize}
		\item Buhlmann track dissolved gases in the tissues compartment (tissu tension)
		\item VPM track the bubbles' radius
		\item VPM calculates bubbles' pressures during the ascent
		\item Bubbles grow only when pressure of the bubble exceed the tissue tension (VPM vs Buhlmann)
		\item Decompression stop before a bubble starts to grow
	\end{itemize}
\end{frame}

\begin{frame}{\insertsubsection}
	Original VPM algorithm\\
	\begin{itemize}
		\item Too conservative
		\item "no-bubble-growth model"
	\end{itemize}
	\pause \vfill
	Modifications made:
	\begin{itemize}
		\item Body can handle a certain amount of free gas
		\item DCS over a critical volume
		\item Allow bubbles from a certain size (critical radius) to grow
		\item "not-too-much-bubble-growth model"
	\end{itemize}
	\pause \vfill
	Bubble population:
		\begin{itemize}
			\item Numerous small bubble, few large bubbles
		\end{itemize}
\end{frame}

\begin{frame}{\insertsubsection}  
	Latest modifications:\\
	VPM-B:
	\begin{itemize}
		\item Incorporates Boyle's law to the bubbles behaviour
		\item Note bubbles' size at the first stop
		\item Increase the radius of each bubbles for next stop according to Boyle's law
		\item Tissue tension allowed for the next stop is reduced due to the larger bubble radius
		\item Add conservatism to the profile
	\end{itemize}
	VPM-B/E:
	\begin{itemize}
		\item Add conservatism for extreme dives
	\end{itemize}
\end{frame}

%---------------------------------------------------------------------
%										KEY POINTS
%---------------------------------------------------------------------
\section{Summary}
\subsection{Key points}
\begin{frame}{\insertsubsection}
	Key Points:
	\begin{itemize}
		\item Buhlmann dissolved model
		\item Bubbles behaviour
		\begin{itemize}
			\item Diffusion with tissue compartment
			\item Laplace pressure
			\item Grow or shrink
			\item Critical radius
		\end{itemize}
		\item Principle of VPM model
		\begin{itemize}
			\item How the model is made
			\item Variable permeability
			\item Bubble crushing
			\item Algorithm
			\item Variants
		\end{itemize}
	\end{itemize}
\end{frame}

%---------------------------------------------------------------------
%										IMPORTANCE
%---------------------------------------------------------------------
\subsection{Importance to understand VPM}
\begin{frame}{\insertsubsection}
	What is that VPM-B and why is it important to know it?
	\begin{itemize}
		\item Popular decompression model
		\item Alternative to Buhlmann ZHL16
		\item Way to double check your schedule
		\item Understand your dive profile
	\end{itemize}
\end{frame}

%---------------------------------------------------------------------
%										QUIZZ
%---------------------------------------------------------------------
\subsection{Quizz}
\begin{frame}{\insertsubsection}  
	\textbf{Question 1:}\\
	How many tissue compartment does VPM use for it's Buhlmann model?
	\begin{enumerate}[(a)]
		\item 8
		\item 12
		\item 16
		\item 24
	\end{enumerate}
	\pause
	\textbf{Answer:}(c)\\
	The model in that presentation is the Buhlmann ZHL16
\end{frame}

\begin{frame}{\insertsubsection}  
	\textbf{Question 2:}\\
	To what kind of model does the Buhlmann ZHL belongs to?
	\begin{enumerate}[(a)]
		\item Dissolved Model
		\item Bubble Model
		\item Reduced Gradient Model
		\item Variable Permeability Model
	\end{enumerate}
	\pause
	\textbf{Answer:}(a)
\end{frame}

\begin{frame}{\insertsubsection}  
	\textbf{Question 3:}\\
	Why would a bubble grow
	\begin{enumerate}[(a)]
		\item Pressure pushing in greater than pressure pushing out
		\item Tissue tension greater than bubble pressure
		\item Skin becomes impermeable
		\item Ambiant pressure increase
	\end{enumerate}
	\pause
	\textbf{Answer:}(b )
\end{frame}

\begin{frame}{\insertsubsection}  
	\textbf{Question 4:}\\
	Why does a bubble becomes impermeable?
	\begin{enumerate}[(a)]
		\item High pressure squeezes molecules against each other
		\item The thickness of the skin is too small
		\item Tissue tension exceed the M-Value
		\item Bubble pressure in null
	\end{enumerate}
	\pause
	\textbf{Answer:}(a)
\end{frame}

\begin{frame}{\insertsubsection}  
	\textbf{Question 5:}\\
	What is the main reason of DCS according to VPM model?
	\begin{enumerate}[(a)]
		\item Tissue tension exceeding the M-Value
		\item Amount of free gas in the body
		\item Supersaturation of the tissue compartment
		\item The size of the bubble in the slow compartment
	\end{enumerate}
	\pause
	\textbf{Answer:}(b)
\end{frame}

\end{document}